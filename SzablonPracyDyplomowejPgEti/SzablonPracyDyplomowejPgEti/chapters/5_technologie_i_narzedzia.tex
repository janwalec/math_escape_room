\chapter{Technologie i narzędzia}
Realizacja projektu aplikacji edukacyjnej typu escape room w środowisku rzeczywistości wirtualnej wymagała doboru odpowiednich narzędzi oraz technologii, które umożliwiłyby stworzenie interaktywnej, atrakcyjnej wizualnie i funkcjonalnej aplikacji kompatybilnej z systemami dostępnymi w Laboratorium Zanurzonej Wizualizacji Przestrzennej.
\label{chap:research}

\section{Silnik gry (Konrad Czarnecki)}
Do stworzenia aplikacji wirtualnego escape roomu zdecydowano się na wykorzystanie silnika Unreal Engine 5, który jest jednym z najpopularniejszych środowisk do tworzenia gier komputerowych oraz aplikacji wirtualnej rzeczywistości. Unreal Engine, rozwijany przez firmę Epic Games, umożliwia tworzenie zaawansowanych wizualnie, interaktywnych projektów 3D oraz VR dzięki nowoczesnemu systemowi renderowania, rozbudowanemu edytorowi oraz wsparciu dla technologii immersyjnych.

Silnik ten oferuje użytkownikom możliwość programowania logiki aplikacji zarówno w języku C++, jak i z wykorzystaniem wizualnego systemu skryptowego Blueprint, co znacząco przyspiesza proces tworzenia prototypów i ułatwia implementację interakcji w środowisku VR. Unreal Engine zapewnia również bogaty zestaw narzędzi do tworzenia animacji, efektów specjalnych, obsługi dźwięku oraz integracji z zewnętrznymi bibliotekami.

Podczas analizy możliwych rozwiązań rozważano również wykorzystanie silnika Unity, który podobnie jak Unreal Engine jest szeroko stosowany w branży gier i aplikacji VR. Unity charakteryzuje się dużą elastycznością, wsparciem dla wielu platform oraz dostępnością licznych wtyczek i rozszerzeń. W porównaniu z Unreal Engine, Unity posiada mniej rozbudowany natywny system graficzny oraz wymaga większego nakładu pracy przy tworzeniu zaawansowanych efektów wizualnych.

Ostatecznym argumentem przemawiającym za wyborem Unreal Engine była kwestia kompatybilności — Laboratorium Zanurzonej Wizualizacji Przestrzennej, w którym aplikacja miała zostać wdrożona, nie wspiera nowszych wersji Unity, natomiast Unreal Engine zapewniał pełną zgodność z istniejącą infrastrukturą sprzętową i programową laboratorium.





\section{Laboratorium Zanurzonej Wizualizacji Przestrzennej (Konrad Czarnecki)}
Laboratorium Zanurzonej Wizualizacji Przestrzennej (LZWP) to specjalistyczne środowisko badawczo-edukacyjne wyposażone w systemy rzeczywistości wirtualnej, umożliwiające tworzenie i testowanie aplikacji immersyjnych w warunkach kontrolowanych. W skład laboratorium wchodzą między innymi systemy typu CAVE, czyli pomieszczenia projekcyjne z ekranami ściennymi i podłogowymi, które otaczają użytkownika obrazem 3D wyświetlanym z kilku projektorów.

LZWP wyposażone jest również w systemy śledzenia pozycji i ruchu użytkownika oraz kontrolery umożliwiające interakcję z wirtualnym środowiskiem. Dzięki temu laboratorium stanowi doskonałe zaplecze do testowania aplikacji edukacyjnych VR oraz prowadzenia badań nad efektywnością i ergonomią rozwiązań immersyjnych.




\section{Środowisko 3D (Konrad Czarnecki)}
W procesie tworzenia aplikacji niezbędne było opracowanie modeli 3D reprezentujących obiekty, elementy wystroju oraz interaktywne przedmioty pojawiające się w wirtualnym escape roomie. Do tego celu wykorzystano Blender — darmowe oprogramowanie do modelowania trójwymiarowego, animacji, teksturowania oraz renderowania.

Blender oferuje szeroki zakres narzędzi umożliwiających tworzenie szczegółowych modeli 3D, generowanie animacji oraz przygotowywanie materiałów i tekstur kompatybilnych z silnikami do tworzenia gier. Dzięki wsparciu dla formatów eksportowych takich jak FBX i GLTF, modele stworzone w Blenderze mogły zostać bezproblemowo zaimportowane do Unreal Engine i wykorzystane w aplikacji VR.



\section{System kontroli wersji (Konrad Czarnecki)}
Dla zapewnienia bezpieczeństwa danych oraz efektywnego zarządzania projektem zastosowano system kontroli wersji Git wraz z usługą hostingową GitHub. GitHub umożliwia przechowywanie kodu źródłowego, modeli 3D, plików dźwiękowych oraz dokumentacji projektowej w repozytorium zdalnym z możliwością współdzielenia zasobów pomiędzy członkami zespołu.

Dzięki systemowi kontroli wersji możliwe było śledzenie historii zmian, zarządzanie gałęziami projektowymi oraz szybkie przywracanie poprzednich wersji w przypadku wystąpienia błędów. W projekcie GitHub pełnił również rolę platformy do przechowywania backupów.