\chapter*{Streszczenie (Jan Walczak)}

Zaangażowanie uczniów w szkołach średnich spada, a lekcje w standardowej formule mogą wydawać
się nieatrakcyjne. Matematyka jest jednym z trudniejszych przedmiotów w szkole, którego
uczniowie uczą się na każdym etapie edukacji. Rośnie więc potrzeba dywersyfikacji sposobów
nauczania i urozmaicania zajęć, tak aby stawały się one ciekawsze.

W ramach pracy zaprojektowano i wykonano aplikację matematycznego escape roomu dla uczniów szkół średnich,
która miałaby przedstawić im naukę matematyki jako coś interesującego i wartego ich zaangażowania.
Projekt obejmuje opracowanie 13 zagadek, z których każda dotyczy innego działu matematyki i porusza zagadnienia
obowiązujące w podstawie programowej szkół średnich. Implementacja została wykonana w Laboratorium 
Zanurzonej Wizualizacji Przestrzennej z wykorzystaniem technologii VR. Głęboka immersja i interaktywność
przedstawionego rozwiązania mają pozwolić uczniom na efektywną i przyjemną naukę.

\textbf{Słowa kluczowe}: matematyczny escape room, Laboratorium Zanurzonej Wizualizacji Przestrzennej, 
CAVE, Unreal Engine, VR, szkoły średnie.


%\addcontentsline{toc}{chapter}{Streszczenie}  

\chapter*{Abstract (Jan Walczak)}

Fewer and fewer students are engaged in school, and lessons in the standard format may seem unattractive
to them. Mathematics is one of the most difficult subjects that students study at every stage of their 
education. Because of this, the demand for diversifying teaching methods is increasing so that learning
becomes more interesting.

As part of an engineering project, a mathematical escape room application for high school students was 
designed and developed to present mathematics as something fresh and worth their effort. The project consists 
of 13 different puzzles, each of which introduces a different branch of mathematics and addresses topics 
covered in the curriculum. The project was implemented in the Immersive 3D Visualization Lab using VR technology. 
The deep immersion and interactivity of the solution are intended to enable students to study effectively and 
enjoyably.

\textbf{Keywords}: mathematical escape room, Immersive 3D Visualization Lab, CAVE, Unreal Engine, VR, middle school,
high school.

%\addcontentsline{toc}{chapter}{Abstract}  
