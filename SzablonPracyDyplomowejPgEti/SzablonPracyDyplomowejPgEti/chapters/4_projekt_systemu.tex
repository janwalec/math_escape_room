\chapter{Projekt systemu}
\label{chap:project}



\section{Wymagania funkcjonalne i niefunkcjonalne (Konrad Czarnecki)}
Przed przystąpieniem do implementacji aplikacji edukacyjnej w formie escape roomu w środowisku rzeczywistości wirtualnej konieczne było precyzyjne określenie wymagań funkcjonalnych i niefunkcjonalnych, jakie powinna spełniać projektowana aplikacja. Zdefiniowanie wymagań pozwala na kontrolowanie procesu realizacji projektu, zapewnia jego zgodność z oczekiwaniami użytkowników końcowych oraz umożliwia przeprowadzenie testów walidacyjnych w końcowej fazie prac.

\subsection{Wymagania funkcjonalne}

Wymagania funkcjonalne określają, jakie działania użytkownik może wykonać w aplikacji oraz jakie funkcje powinna ona realizować. W przypadku projektowanej aplikacji escape room VR wymagania te obejmują:
\begin{itemize}[left=1.5em, label=\textbullet, topsep=0pt, itemsep=0pt]
    \item Wyświetlanie wirtualnego środowiska escape roomu – Aplikacja powinna umożliwiać użytkownikowi poruszanie się po wirtualnym pokoju zagadek, obejmującym zestaw pomieszczeń związanych z różnymi działami matematyki.

    \item Realizacja 13 zagadek matematycznych – aplikacja musi zawierać 13 interaktywnych zagadek, po jednej dla każdego z wybranych działów matematyki: liczby rzeczywiste, wyrażenia algebraiczne, równania i nierówności, układy równań, funkcje, ciągi, trygonometria, planimetria, geometria analityczna, stereometria, kombinatoryka, rachunek prawdopodobieństwa i statystyka oraz optymalizacja i rachunek różniczkowy.

    \item Interaktywność zagadek – każda zagadka powinna wymagać od użytkownika aktywnej interakcji z otoczeniem wirtualnym, np. manipulowania obiektami, wpisywania odpowiedzi czy przestawiania elementów.

    \item System weryfikacji poprawności odpowiedzi – aplikacja powinna sprawdzać poprawność rozwiązań podawanych przez użytkownika oraz wyświetlać komunikaty informujące o sukcesie lub błędzie.

    \item Śledzenie postępu użytkownika – system powinien zapamiętywać, które zagadki zostały już rozwiązane, i odblokowywać dostęp do kolejnych pomieszczeń w ustalonej kolejności.

    \item Zakończenie rozgrywki – po rozwiązaniu wszystkich 13 zagadek aplikacja powinna wyświetlić ekran podsumowujący z informacją o ukończeniu escape roomu.

    \item Obsługa urządzeń VR i interfejsów w Laboratorium – aplikacja musi być kompatybilna z systemami śledzenia ruchu i projekcji wykorzystywanymi w Laboratorium Zanurzonej Wizualizacji Przestrzennej, w tym z systemami typu CAVE oraz kontrolerami ruchu.
\end{itemize}



\subsection{Wymagania niefunkcjonalne}

Wymagania niefunkcjonalne opisują cechy jakościowe systemu oraz warunki, jakie powinien spełniać, aby zapewnić poprawne działanie i pozytywne doświadczenia użytkowników. W kontekście projektowanej aplikacji escape room VR wymagania niefunkcjonalne obejmują:
\begin{itemize}[left=1.5em, label=\textbullet, topsep=0pt, itemsep=0pt]
    \item Kompatybilność z infrastrukturą LZWP – aplikacja musi działać poprawnie w warunkach technicznych Laboratorium Zanurzonej Wizualizacji Przestrzennej, współpracując z systemem projekcyjnym CAVE oraz wykorzystywanymi kontrolerami ruchu.

    \item Wydajność – aplikacja powinna działać płynnie, z minimalnym opóźnieniem renderowania obrazu i reakcji na ruchy użytkownika, zapewniając co najmniej 60 klatek na sekundę w środowisku VR.

    \item Jakość grafiki – wirtualne środowisko escape roomu powinno charakteryzować się realistyczną i spójną oprawą wizualną wspomagającą immersję.

    \item Łatwość rozbudowy i modyfikacji – struktura projektu powinna umożliwiać przyszłą rozbudowę aplikacji o nowe zagadki, pomieszczenia lub funkcjonalności bez konieczności przebudowy istniejącego kodu i modeli.

    \item Przenośność kodu i zasobów – wszystkie pliki źródłowe, modele oraz pliki graficzne powinny być przechowywane w repozytorium GitHub w sposób umożliwiający łatwe przenoszenie projektu pomiędzy stanowiskami roboczymi oraz serwerami laboratorium.

    \item Dokumentacja – projekt musi być opatrzony szczegółową dokumentacją techniczną i użytkową, opisującą strukturę aplikacji, sposób instalacji, obsługi oraz instrukcje dotyczące przyszłej rozbudowy.
\end{itemize}

\section{Scenariusz gry (Konrad Czarnecki)}

Po uruchomieniu aplikacji gracz przenosi się do pierwszego z 13 wirtualnych pomieszczeń. Każde pomieszczenie jest
utrzymane w odmiennym stylu wizualnym i zawiera jedną interaktywną zagadkę matematyczną z przypisanego działu.
Zagadki muszą być rozwiązywane w określonej kolejności, a przejście do kolejnego pokoju jest możliwe dopiero po
poprawnym rozwiązaniu bieżącej zagadki. Pomijanie zagadek lub powrót do poprzednich pokoi nie są możliwe z poziomu
użytkownika.

W każdym pokoju gracz ma możliwość poruszania się po przestrzeni VR, wchodzenia w interakcje z elementami zagadki
oraz przechodzenia do kolejnego pomieszczenia po poprawnym rozwiązaniu zagadki. 

Działanie każdej zagadki może wymagać różnych form interakcji — takich jak przeciąganie obiektów, wpisywanie wyników
na wirtualnej klawiaturze, wskazywanie elementów przestrzeni, czy manipulowanie wirtualnymi narzędziami (np. suwakami i
przyciskami).
Po rozwiązaniu ostatniej, trzynastej zagadki użytkownik zostaje przeniesiony do wirtualnego pokoju podsumowań, gdzie
gra informuje go o odniesionym sukcesie.

W przypadku przerwania rozgrywki przez użytkownika lub awarii systemu, aplikacja powinna zapewniać możliwość powrotu
do gry lub całkowitego zamknięcia gry bez ryzyka utraty integralności danych systemu. Rozwiązanie to
gwarantuje bezpieczeństwo użytkownika oraz stabilność aplikacji w środowisku CAVE.

Administrator systemu ma możliwość pomijania zagadek i bezpośredniego przeniesienia gracza do wybranego pokoju, co może
być wykorzystane na życzenie użytkownika w związku z jego preferencjami w zakresie zagadek i dziedzin matematyki lub w
przypadku awarii systemu. Funkcjonalność ta jest dostępna wyłącznie dla uprawnionych użytkowników i nie jest widoczna
dla zwykłych graczy, by uniemożliwić przypadkowe pominięcie zagadek.

Tak skonstruowany scenariusz rozgrywki umożliwia nie tylko wprowadzenie użytkownika w świat wirtualnej
rzeczywistości, ale także zapewnia dynamiczny, uporządkowany i logiczny przebieg interakcji. Dzięki systematycznemu
rozwiązywaniu zagadek i przemieszczaniu się między pokojami, aplikacja utrzymuje wysoki poziom zaangażowania i
motywacji użytkowników do ukończenia całej gry.







%\section{Zagadki (Konrad Czarnecki)}
%W ramach realizacji projektu opracowano trzynaście interaktywnych zagadek matematycznych, z których każda reprezentuje inny dział matematyki nauczanej na poziomie szkoły średniej. Zagadki zostały zaprojektowane tak, aby nie tylko sprawdzać wiedzę użytkowników, ale również aktywizować ich poprzez wykorzystanie mechanik charakterystycznych dla gier logicznych i escape roomów. Dzięki zastosowaniu technologii wirtualnej rzeczywistości, gracze mogą wchodzić w bezpośrednią interakcję z otoczeniem i rozwiązywać zadania w formie angażujących, przestrzennych łamigłówek. Każde z zadań posiada unikalny scenariusz i zasady działania, dostosowane do specyfiki danego działu matematyki oraz do możliwości sprzętowych Laboratorium Zanurzonej Wizualizacji Przestrzennej. Poniżej przedstawiono szczegółowy opis wszystkich trzynastu zagadek, wraz z określeniem ich celu, przebiegu oraz warunków zakończenia.
%
%
%
%\subsection{Wyrażenia algebraiczne}
%
%Pierwsze zadanie wirtualnego escape roomu polega na rozpoznawaniu oraz uzupełnianiu wzorów skróconego mnożenia. Po pojawieniu się gracza w wirtualnym pomieszczeniu widoczna jest tablica zawierająca początki wybranych wzorów algebraicznych. Każda formuła przedstawiona jest w postaci lewej strony równania wraz ze znakiem „=”, po którym pozostawiono puste miejsce przeznaczone na brakującą część wyrażenia.
%
%W przestrzeni wirtualnego pokoju rozmieszczone są interaktywne bloki zawierające możliwe zakończenia wzorów. Wśród dostępnych elementów znajdują się zarówno poprawne zakończenia odpowiadające właściwym wzorom skróconego mnożenia, jak i bloki z nieprawidłowymi zapisami, które pełnią funkcję elementów rozpraszających. Zadaniem gracza jest przeciąganie wybranych bloków oraz umieszczanie ich w odpowiednich miejscach na tablicy, tak aby utworzyć kompletne i matematycznie poprawne formuły.
%
%Po uzupełnieniu wszystkich pustych pól użytkownik ma możliwość uruchomienia przycisku sprawdzającego. Po jego użyciu system dokonuje weryfikacji poprawności skonstruowanych równań i wyświetla komunikat tekstowy informujący o wyniku zadania. Zagadkę uznaje się za rozwiązaną w momencie prawidłowego dopasowania wszystkich zakończeń do odpowiadających im początków wzorów.
%
%\subsection{Planimetria}
%
%Po ukończeniu pierwszego etapu, powierzchnia interaktywnego stołu zmienia swój wygląd, a na ekranie pojawia się plansza zawierająca dwanaście par figur geometrycznych. Wśród nich znajdują się takie figury jak kwadraty, romby, prostokąty, równoległoboki, trapezy oraz różne typy trójkątów. Zadanie ma na celu sprawdzenie znajomości zależności między figurami oraz umiejętności rozpoznawania ich cech charakterystycznych. Na planszy, pomiędzy każdą parą figur, znajduje się miejsce przeznaczone na wybór relacji. Gracze, analizując właściwości obu figur w każdej parze, muszą zdecydować, czy jedna z nich jest szczególnym przypadkiem drugiej, czy też nie. Do dyspozycji mają dwa symbole: strzałkę w prawo, oznaczającą, że pierwsza figura jest szczególnym przypadkiem drugiej, oraz przekreśloną strzałkę, oznaczającą brak takiej zależności. Po dokonaniu wszystkich wyborów system natychmiast weryfikuje poprawność ustawionych relacji. W przypadku błędnych odpowiedzi, gracze otrzymują stosowny komunikat i możliwość poprawienia wyłącznie tych relacji, które zostały oznaczone niepoprawnie. Zadanie kończy się w momencie, gdy wszystkie relacje zostaną poprawnie oznaczone. 
%
%\subsection{Równania i nierówności}
%
%W trzecim zadaniu gracze trafiają na scenę stylizowaną na film przygodowy, gdzie postać odkrywcy musi przejść przez wiszący most zbudowany z kamiennych płytek. Na każdej z płytek widnieje liczba rzeczywista, a na ścianie pojawia się układ dwóch nierówności. Zadaniem uczestników jest rozwiązanie układu i wybranie płytek z wartościami należącymi do przedziału spełniającego oba warunki. Gracze prowadzą postać, klikając po kolei właściwe płytki, a błędny wybór powoduje zapadnięcie się płytki i konieczność powrotu na początek. Po poprawnym przejściu przez most gracze odblokowują dostęp do kolejnej zagadki.
%
%\subsection{Funkcje}
%
%W czwartym zadaniu gracze pracują z dużą interaktywną tablicą, na której wyświetlany jest układ współrzędnych. Ich celem jest dostosowanie parametrów funkcji liniowej w taki sposób, aby jej wykres przechodził przez trzy wyznaczone punkty. Dzięki suwakom lub polom edycyjnym uczestnicy mogą dynamicznie zmieniać współczynnik kierunkowy oraz wyraz wolny funkcji, obserwując na bieżąco, jak zmienia się jej wykres. Zadanie polega na precyzyjnym dobraniu wartości parametrów, by wykres przeciął wszystkie wskazane punkty. 
%
%\subsection{Geometria analityczna na płaszczyźnie kartezjańskiej}
%
%W piątym zadaniu gracze ponownie pracują z interaktywną tablicą, na której tym razem pojawiają się dwa wykresy funkcji liniowych. Celem wyzwania jest odnalezienie współrzędnych punktu, w którym obie funkcje przecinają się na płaszczyźnie kartezjańskiej. Uczestnicy mogą odczytać ten punkt bezpośrednio z wykresu lub rozwiązać układ równań opisujących obie proste. Po ustaleniu wartości współrzędnych gracze wpisują je w specjalnie przygotowane pole. System weryfikuje poprawność odpowiedzi i odblokowuje przejście do kolejnego etapu gry.
%
%\subsection{Kombinatoryka}
%
%W szóstym zadaniu gracze natrafiają na metalowy sejf wyposażony w podświetlany panel numeryczny. Na wyświetlaczu obok pojawia się zagadka z zakresu kombinatoryki dotycząca liczby możliwych ustawień trzycyfrowego kodu PIN, w którym żadna z cyfr się nie powtarza. Uczestnicy muszą przypomnieć sobie zasady permutacji i wyliczyć, ile jest możliwych kombinacji, wybierając trzy różne cyfry spośród dziesięciu dostępnych. Po obliczeniu właściwej liczby gracze wpisują wynik na klawiaturze sejfu.
%
%\subsection{Liczby rzeczywiste}
%
%W siódmym zadaniu gracze pracują z dużą drewnianą skrzynią oznaczoną liczbami i symbolami zbiorów liczbowych. Pierwszym etapem jest przyporządkowanie wszystkich liczb nadrukowanych na ściankach skrzyni do odpowiednich zbiorów: liczb naturalnych, całkowitych, wymiernych i niewymiernych. Na podstawie tego podziału gracze ustalają czterocyfrowy kod otwierający mechaniczny zamek. Po poprawnym wprowadzeniu kombinacji, wewnątrz skrzyni odnajdują płaską planszę ilustrującą przecięcia zbiorów w formie diagramu Venna. Obok leżą kafelki z symbolami działań na zbiorach, które należy odpowiednio rozmieścić na planszy, uzupełniając brakujące miejsca. Po wykonaniu tego zadania, gracze odkrywają kolejny mechanizm, gdzie muszą ułożyć zależności między zbiorami liczbowymi za pomocą ruchomych symboli i strzałek, odzwierciedlając relacje zawierania się zbiorów. Poprawne wykonanie wszystkich trzech etapów aktywuje mechanizm w skrzyni i otwiera dostęp do kolejnego etapu gry.
%
%\subsection{Trygonometria}
%
%W ósmym zadaniu gracze stają przed ścianą, na której umieszczone są cztery duże okręgi, symbolizujące funkcje trygonometryczne: sinus, cosinus, tangens i cotangens. Każdy okrąg podzielony jest na cztery ćwiartki układu współrzędnych oznaczone numerami I–IV. Zadaniem uczestników jest uzupełnienie pustych pól przy każdej ćwiartce odpowiednim znakiem „+” lub „−”, wskazującym, czy dana funkcja przyjmuje w tej ćwiartce wartości dodatnie czy ujemne. Poprawne przypisanie znaków wymaga znajomości własności funkcji trygonometrycznych w poszczególnych ćwiartkach.
%
%\subsection{Ciągi}
%
%W dziewiątym zadaniu gracze przenoszą się na interaktywną platformę stylizowaną na arenę gry rytmicznej, gdzie głównym wyzwaniem jest rozpoznawanie kolejnych wyrazów ciągów liczbowych. Na ekranie wyświetlana jest formuła ciągu arytmetycznego lub geometrycznego, a z góry spadają kostki z różnymi wartościami liczbowymi. Zadaniem uczestników jest przecięcie mieczem świetlnym tylko tych kostek, które zawierają poprawne elementy podanego ciągu. Jeśli kostka nie pasuje do ciągu, gracze powinni ją pozostawić nienaruszoną. Za każdą poprawną akcję przyznawany jest punkt, natomiast błędne cięcie skutkuje odebraniem punktu. Tempo gry rośnie wraz z postępem, zwiększając wymagania dotyczące refleksu i precyzji. Gra kończy się sukcesem po zdobyciu 50 punktów i składa się z dwóch etapów: rozpoznawania ciągu arytmetycznego, a następnie geometrycznego.
%
%\subsection{Rachunek prawdopodobieństwa i statystyka}
%
%W tym zadaniu gracze stają przed wyzwaniem probabilistycznym, reprezentowanym przez interaktywny stół z postacią Morfeusza oraz dwoma pojemnikami i zestawem 100 tabletek – po 50 czerwonych i niebieskich. Celem jest rozdzielenie tabletek między pojemniki w taki sposób, aby maksymalizować szansę wybrania czerwonej tabletki przez Morfeusza. Gracze mogą dowolnie przesuwać tabletki, testując różne układy i obserwując wyświetlany na bieżąco procent prawdopodobieństwa sukcesu. Optymalnym rozwiązaniem jest umieszczenie jednej czerwonej tabletki w pierwszym pojemniku, a pozostałych 49 czerwonych oraz 50 niebieskich w drugim, co daje maksymalną szansę około 74,75\%. Zadanie wymaga od uczestników zrozumienia i wykorzystania zasad prawdopodobieństwa do podejmowania decyzji.
%
%\subsection{Optymalizacja i rachunek różniczkowy}
%
%Gracze stają przed wyzwaniem trafienia w cele na symulowanym polu bitwy. Sterują kątem nachylenia armaty, regulując go na pokrętle, aby zmodyfikować trajektorię lotu pocisku wyświetlaną na ekranie. Ich zadaniem jest precyzyjne ustawienie kąta, tak aby pocisk trafił w pojawiający się w losowej pozycji cel. Po każdym strzale pojawia się nowy cel, a gracz musi trafić łącznie 15 razy. W trakcie rozgrywki uczestnicy poznają związek między parametrami funkcji kwadratowej opisującej tor lotu, a jej maksimum, ucząc się podstaw optymalizacji i rachunku różniczkowego.
%
%\subsection{Układy równań}
%
%Gracze muszą umieścić odpowiednią liczbę jednostek energii w trzech kolorowych pojemnikach: czerwonym, zielonym i niebieskim, tak aby spełnić podane warunki dotyczące ich wydajności i sumy energii. Poprawne ustawienie 12 jednostek energii, z uwzględnieniem zależności między pojemnikami, spowoduje zapalenie się dużej żarówki nad stołem, sygnalizując sukces. System na bieżąco weryfikuje poprawność i informuje o błędach, dając możliwość kolejnych prób.
%
%\subsection{Stereometria}
%
%Gracze w półmroku obracają na obrotowym podeście pojawiającą się bryłę przestrzenną, taką jak graniastosłup, ostrosłup, walec czy stożek. Muszą rozpoznać jej nazwę oraz wskazać właściwości, takie jak liczba ścian, wierzchołków, krawędzi, rodzaj podstawy czy kąty między elementami.

\section{Model przypadków użycia(Andrii Demyshyn)}
Model przypadków użycia opisuje sposób interakcji użytkowników z projektowaną aplikacją edukacyjną typu escape room w środowisku rzeczywistości wirtualnej. W systemie wyróżniono dwóch aktorów: gracza oraz administratora systemu.

Administrator systemu jest osobą nadzorującą przebieg rozgrywki z poziomu komputera sterującego, znajdującego się poza środowiskiem wirtualnym. Do zadań administratora należy uruchamianie i kończenie aplikacji, monitorowanie przebiegu gry oraz reagowanie na sytuacje awaryjne. Aby uniknąć jakichkolwiek problemów w aplikacji, administratorowi dodano możliwość pomijania zagadek w trakcie gry. Po otrzymaniu prośby lub wystąpieniu jakiegokolwiek błędu administrator może uznać poziom za zaliczony i przenieść gracza do następnej zagadki.
To pozwala na zachowanie stabilności działania aplikacji w środowisku CAVE.

Gracz przebywa fizycznie w przestrzeni Laboratorium Zanurzonej Wizualizacji Przestrzennej i korzysta z aplikacji w środowisku VR. Podczas rozgrywki porusza się po wirtualnym pomieszczeniu, wchodzi w interakcję z obiektami poprzez ich naciskanie, przemieszczanie czy wybieranie. Gracz odczytuje informacje tekstowe i komunikaty systemowe wyświetlane w przestrzeni trójwymiarowej oraz rozwiązuje zagadki matematyczne. Także dla wygody użytkowania gracz za pomocą wybranego przycisku może obracać pokój i wszystkie  obiekty w nim na 90 stopni, co pozwala dopasować pokój dla wygodniejszego korzystania.Na każdym etapie system na bieżąco informuje gracza o poprawności wprowadzanych rozwiązań, wyświetlając odpowiednie komunikaty w przestrzeni wirtualnej. Interakcja z aplikacją odbywa się bez wykorzystania klasycznego interfejsu graficznego, a wszystkie działania realizowane są bezpośrednio w przestrzeni trójwymiarowej. Po poprawnym rozwiązaniu zagadki system automatycznie przechodzi do kolejnego etapu rozgrywki, aż do ukończenia wszystkich trzynastu zagadek.

Zaprojektowany model przypadków użycia odpowiada rzeczywistym funkcjonalnościom zaimplementowanym w aplikacji i zapewnia czytelny podział ról pomiędzy użytkownikiem końcowym a osobą nadzorującą system.
\section{Projekt architektury systemu(Andrii Demyshyn)}
Architektura projektowanej aplikacji została zaprojektowana z myślą o realizacji edukacyjnej gry typu escape room w środowisku rzeczywistości wirtualnej. System został zaimplementowany z wykorzystaniem silnika Unreal Engine i przystosowany do działania w środowisku CAVE, wykorzystywanym w Laboratorium Zanurzonej Wizualizacji Przestrzennej.
Struktura aplikacji opiera się na kilku głównych komponentach odpowiedzialnych za poszczególne aspekty działania systemu. Centralnym elementem architektury jest moduł zarządzania rozgrywką zaimplementowanej jako obiekt typu Blueprint Actor MergeBP. MergeBP kontroluje aktualny etap gry, kolejność zagadek oraz postęp gracza. Moduł ten odpowiada również za przejścia pomiędzy kolejnymi etapami po poprawnym rozwiązaniu zagadek matematycznych.
Logika zagadek matematycznych została podzielona na trzynaście niezależnych poziomów gry, każdy z których odpowiada jednej zagadce i jednemu działowi matematyki. Każda zagadka posiada własny mechanizm interakcji oraz weryfikacji poprawności rozwiązania, co umożliwia ich łatwą modyfikację lub rozbudowę bez ingerencji w pozostałą część systemu. Takie podejście upraszcza również proces testowania poszczególnych etapów gry.
Istotnym elementem architektury aplikacji jest sposób realizacji środowiska gry. Cała rozgrywka odbywa się w jednej wspólnej przestrzeni wirtualnej, która dynamicznie zmienia się w zależności od aktualnego etapu gry. Poszczególne elementy pomieszczenia, obiekty zagadek oraz elementy interaktywne są ukrywane, usuwane, aktywowane lub teleportowane w odpowiednie miejsca po ukończeniu danego poziomu. Rozwiązanie to pozwala na zachowanie spójności środowiska oraz ograniczenie konieczności ładowania nowych scen.
Moduł interakcji odpowiada za obsługę kontrolerów ruchu oraz systemów śledzenia pozycji użytkownika w przestrzeni CAVE. Umożliwia on graczowi poruszanie się po wirtualnym pomieszczeniu, manipulowanie obiektami oraz wykonywanie akcji wymaganych do rozwiązania zagadek. Równolegle system obsługuje tryb administratora, który działa z poziomu komputera sterującego i umożliwia nadzorowanie przebiegu rozgrywki oraz ingerencję w jej przebieg w sytuacjach awaryjnych.
Zaprojektowana architektura systemu zapewnia stabilne działanie aplikacji w środowisku rzeczywistości wirtualnej oraz umożliwia jej dalszą rozbudowę, na przykład poprzez dodanie nowych zagadek matematycznych lub rozszerzenie scenariusza gry.


\section{Projekt interfejsu użytkownika i środowiska gry(Andrii Demyshyn)}
Projekt interfejsu użytkownika oraz środowiska gry został opracowany z myślą o specyfice rzeczywistości wirtualnej oraz warunkach pracy w systemie CAVE. Głównym założeniem było zapewnienie wysokiego poziomu immersji oraz intuicyjnej obsługi aplikacji przy jednoczesnym ograniczeniu elementów mogących rozpraszać uwagę gracza podczas rozwiązywania zagadek matematycznych.
W projektowanej aplikacji zrezygnowano z klasycznego, dwuwymiarowego interfejsu graficznego w postaci menu czy stałych elementów HUD. Wszystkie informacje przekazywane użytkownikowi, takie jak komunikaty systemowe, treści zadań czy informacja o poprawności rozwiązania, prezentowane są bezpośrednio w przestrzeni trójwymiarowej jako elementy świata gry. Takie rozwiązanie pozwala na zachowanie spójności wizualnej oraz zwiększa poczucie obecności w środowisku wirtualnym.
Środowisko gry zostało zaprojektowane jako jedna wspólna przestrzeń wirtualna, której wygląd i zawartość zmieniają się w zależności od aktualnego etapu rozgrywki. Poszczególne zagadki matematyczne realizowane są poprzez dynamiczne pojawianie się, ukrywanie lub modyfikowanie obiektów znajdujących się w pomieszczeniu. Dzięki temu możliwe było zachowanie jednolitej przestrzeni przy jednoczesnym wyraźnym rozróżnieniu kolejnych etapów gry.
Interakcja gracza z otoczeniem realizowana jest poprzez kontrolery ruchu, umożliwiające wybieranie, naciskanie oraz przemieszczanie obiektów. Dodatkowo w aplikacji zaimplementowano możliwość obrotu całego pomieszczenia wraz z obiektami o 90 stopni, co pozwala na dostosowanie orientacji przestrzeni do preferencji użytkownika. Rozwiązanie to zwiększa komfort użytkowania aplikacji i ułatwia rozwiązywanie zagadek.
Projekt interfejsu oraz środowiska gry został przetestowany w rzeczywistych warunkach Laboratorium Zanurzonej Wizualizacji Przestrzennej. Przeprowadzone testy pozwoliły na dopasowanie skali obiektów, czytelności komunikatów oraz sposobu interakcji do potrzeb użytkowników, zapewniając płynne i komfortowe korzystanie z aplikacji w środowisku VR.