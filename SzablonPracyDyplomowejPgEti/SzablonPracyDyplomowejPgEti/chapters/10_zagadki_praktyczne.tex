\chapter{Implementacja zagadek}
\label{chap:development}

\section{Zadanie 1 – wzory skróconego mnożenia (Autor)}
a
\section{Zadanie 2 – Planimetria (Jan Walczak)}

\subsection{Problemy i różnice w realizacji zadania w praktyce}
Podczas implementacji zadania, zgodnie z teorią opisaną w podrozdziale \ref{subsec:planimetria_teoria}
napotkałem się z kilkoma problemami. Przede wszystkim zauważyłem, że odpowiedzi są zero--jedynkowe.
Przykładowo: uczeń ma do uzupełniania relację typu \glqq każdy kwadrat ... prostokątem\grqq. Jeśli 
odpowie niepoprawnie tj. zaznaczy odpowiedź \glqq nie jest\grqq, a zostanie od razu zapytany ponownie o tę
samą relację, to od razu wyklucza jedną z odpowiedzi. Tym samym zadanie zatraca swoją wartość edukacyjną
-- uczeń może rozwiązać całe zadanie stosując jedynie metodę eliminacji.

Rozwiązanie tego problemu, które zostało zaimplementowane, to zdefiniowanie puli takich relacji i po udzieleniu
przez ucznia odpowiedzi, każdorazowe losowanie relacji innej niż ta poprzednia.
W ten sposób uniemożliwia się uczniowi stosowania zasady eliminacji i wymusza na nim prawidłowe podejście.

Kolejnym problemem była prezentacja zadania. Zwykła tabela, którą uczeń miałby uzupełniać mogłaby wydać mu się
mało ciekawa. Tym samym postanowiłem wizualnie usprawnić zagadkę. Na środku zostały umieszczone dwa przyciski:
\begin{itemize}
    \item każdy
    \item nie każdy
\end{itemize}
Uczeń zostaje poinstruowany, że po obu ścianach pokoju zostaną wyświetlone różne figury geometryczne. Po lewej stronie,
patrząc od przycisków -- figury oznaczone kolorem czerwonym, po prawej stronie -- figury oznaczone kolorem zielonym.
Liczba figur jest stała, każdorazowo typ wyświetlanych figur jest wybierany z określonej puli i wyświetlany w pseudolosowej 
konfiguracji -- losowane jest ich położenie, obrót oraz rozmiar. Zadaniem ucznia jest uzupełnianie kolejnych relacji poprzez
wybieranie odpowiednich przycisków. Relacja wyświetlana na ścianie pokoju przedstawia się jako:
\glqq każdy typ figury, narysowany kolorem czerwonym, jest równocześnie typem figury oznaczonym kolorem zielonym\grqq.
Dodatkowo, tekst jest odpowiednio pokolorowany, tak aby uczeń nie miał wątpliwości, że chodzi o typ figury, wyświetlane tymże kolorem
na ścianach.

\subsection{Implementacja struktury danych przechowującej relację}
Na początku pracy należało zdefiniować strukturę, przechowującą relację, czyli innymi słowami, pytanie na które uczeń
będzie odpowiadał. Relacja taka została zdefiniowana jako aktor. Zawiera pola:

\begin{itemize}
    \item \verb|Every| -- wartość logiczna
    \item \verb|FigureA| -- ciąg tekstowy, reprezentujący pierwszą figurę w relacji
    \item \verb|FigureB| -- ciąg tekstowy, reprezentujący drugą figurę w relacji
\end{itemize}

Wartość zmiennej \verb|Every| odpowiada na pytanie, czy każda figura typu pierwszego (\verb|FigureA|) jest równocześnie
figurą typu drugiego (\verb|FigureB|). Struktura zawierała również funkcję słownikową, czyli taką, która tłumaczy wartości
tekstowe na liczbowe.

\subsection{Implementacja kontrolerów}
Aby zarządzać zagadką została zaimplementowana seria kontrolerów i menadżerów.
\begin{itemize}
    \item \verb|ControlerFigures| (główny kontroler)
    \item \verb|FiguresTextManager|
    \item \verb|WallFigures|
\end{itemize}

\verb|WallFigures| to najprostszy z kontrolerów. Wykonuje polecenia głównego kontrolera. Jego zadaniem jest 
wyświetlanie żądanych figur i ich losowe ustawianie (obracanie, skalowanie, rozmieszczanie). W projekcie występują
jego dwie instancje: kontroler lewy i prawy. Odpowiadają za odpowiednie ściany, na których wyświetlane są figury.

\verb|FiguresTextManager| jest odpowiedzialny za zarządzanie tekstem wyświetlanym na ekranie. Wykonuje polecenia
głównego kontrolera. Jego zadaniem jest odpowiednie wyświetlanie, skalowanie i kolorowanie tekstu.

\verb|ControlerFigures| czyli kontroler główny jest najważniejszym elementem zagadki. Zawiera referencję do pozostałych
kontrolerów, zarządza obiektami wyświetlanymi na scenie i kontroluje przebieg zadania.

\subsection{Przebieg zadania}


\section{Zadanie 3 – Nierówności (Autor)}
\section{Zadanie 4 – Funkcje (Autor)}
\section{Zadanie 5 – Geometria analityczna (Autor)}
\section{Zadanie 6 – Kombinatoryka (Autor)}
\section{Zadanie 7 – Liczby rzeczywiste i działania na zbiorach liczbowych 
\texorpdfstring{\\}{ } (Jan Walczak)}
\section{Zadanie 8 – Znaki funkcji trygonometrycznych (Autor)}
\section{Zadanie 9 – Ciągi liczbowe (Jan Walczak)}
\section{Zadanie 10 – Prawdopodobieństwo (Autor)}
\section{Zadanie 11 – Optymalizacja i rachunek różniczkowy (Autor)}
\section{Zadanie 12 – Układy równań (Autor)}
\section{Zadanie 13 – Stereometria (Jan Walczak)}
