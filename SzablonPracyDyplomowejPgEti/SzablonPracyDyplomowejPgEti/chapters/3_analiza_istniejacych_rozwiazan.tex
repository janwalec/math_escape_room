\chapter{Analiza istniejących rozwiązań (Jan Walczak)}
\label{chap:algs}

\section{\glqq Gravity Sketch\grqq}

\subsection{Implementacja rozwiązania}
Badanie, przeprowadzone na uczelni Tecnologico de Monterrey, miało charakter eksperymentalny i było przeprowadzone na studentach pierwszego roku \cite{Campos2022}.
Opierało się o~zainstalowanie oprogramowania \glqq Gravity Sketch\grqq~na urządzeniu VR. Narzędzie służy w praktyce do trójwymiarowego projektowania 
różnych obiektów w~wirtualnej przestrzeni. Aspektem, który wyróżnia oprogramowanie \glqq Gravity Sketch\grqq, jest
umożliwienie rysowania obiektów opierających się o siatkę współrzędnych, a co za tym idzie, użyto go do 
wizualizacji wektorów w przestrzeni trójwymiarowej \cite{Campos2022}. 

Możliwość samodzielnego rysowania wektorów powinna ułatwić studentom wizualizację omawianego zagadnienia przez
responsywność narzędzia i możliwość zmiany perspektywy (przez ruch użytkownika).


\subsection{Przeprowadzone badania}
Głównym celem badania było sprawdzenie jak technologia VR pomaga w nauce abstrakcyjnych pojęć matematycznych, które wymagają od ucznia wyobraźni przestrzennej. 
Uczestnikami było 94 studentów, którzy zostali podzieleni na dwie grupy: kontrolną, która uczestniczyła w~standardowych
zajęciach i eksperymentalną, która brała udział w zajęciach z użyciem technologii VR~\cite{Campos2022}.

Obie grupy rozwiązywały dwa testy: przed rozpoczęciem i po zakończeniu zajęć. Obejmowały one zagadnienia związane z wektorami i dotykał problemów
powiązanych z ich wizualizacją, takich jak obliczanie kątów. Dodatkowo grupa eksperymentalna wypełniała ankietę zawierającą pytania o~ich
doznania przy podczas używania wirtualnej rzeczywistości \cite{Campos2022}. 

\subsection{Wyniki}
W badaniu opracowanym przez Esmeraldę Campos, Irvinga Hidrogo i Genaro Zavala w \cite{Campos2022} możemy przeczytać:
\textit{\glqq We found that on those items in which the visualization was important, students in the experimental group, i.e., using
VR, did better than those who did not use VR. We have evidence that VR can help students visualize angles and components that 
help them solve problems better.\grqq}. Warto zauważyć, że w pytaniach dotyczących wizualizacji grupa, która korzystała z VR uzyskała znacznie lepszy przyrost wiedzy niż grupa
kontrolna \cite{Campos2022}.

Można więc założyć, że zagadnienia wymagające od ucznia wyobraźni przestrzennej mogą wymagać dodatkowego wsparcia 
przez rozszerzenie standardowej formuły lekcji o aplikację opartą o technologię VR.



\section{Wirtualny pokój zagadek z zakresu matematyki}

\subsection{Projekt i implementacja}
Wirtualny pokój zagadek z zakresu matematyki został opracowany na potrzeby publikacji
w~\glqq Zeszytach Naukowych Wydziału Elektroniki i Automatyki Politechniki Gdańskiej\grqq~\cite{lebiedz1}.

Artykuł opisuje innowacyjny projekt edukacyjny, polegający na stworzeniu matematycznego pokoju zagadek
(escape room) na poziomie studiów inżynierskich. Jak możemy przeczytać we wskazanym artykule,
celem pracy było stworzenie aplikacji na platformie Unity, która miałaby działać w Laboratorium
Zanurzonej Wizualizacji Przestrzennej i uatrakcyjnić studentom naukę matematyki \cite{lebiedz1}.

Escape room został podzielony na 4 pokoje, z czego pierwszy z nich był pokojem wprowadzającym,
prowadził on do 3 kolejnych pokoi zawierających łącznie 13 zagadek. Pokoje były podzielone nie tylko
ze względu na tematykę zagadnień, ale także ze względu na wystrój \cite{lebiedz1}. Podział ten wyglądał następująco:

\begin{enumerate}
    \item Pokój w stylu nowoczesnym: zawierał zadania z zakresu m.in. wyznaczników macierzy (metoda Sarrusa), równań płaszczyzny oraz wykresów funkcji.
    \item Pokój warsztatowy: skupiał się na ciągach liczbowych (np. ciągu Fibonacciego), liczbach zespolonych, schemacie Hornera oraz systemie binarnym.
    \item Pokój w stylu egipskim: oferował zadania dotyczące działań na liczbach zespolonych oraz układów równań liniowych rozwiązywanych metodą Gaussa-Jordana.
\end{enumerate}

\subsection{Przeprowadzone badania}

Przy użyciu wcześniej omawianego rozwiązania został przeprowadzony eksperyment obejmujący
grupę 54 studentów. Badanie wykazało, że nauka poprzez rozwiązywanie zagadek w matematycznym
escape roomie przynosi wymierne korzyści \cite{lebiedz1}.

Badanie zostało rozszerzone i opisane w artykule \glqq Educational values of a virtual escape room in mathematics\grqq~\cite{lebiedz2}. Badanie polegało na podzieleniu studentów na 2 równe grupy, 
na których zostały zastosowane dwa oddzielne podejścia przeprowadzane w różnej kolejności. Pierwsze podejście polegało na uczestnictwie grupy w tradycyjnej lekcji matematyki,
a~drugie na przejściu matematycznego escape roomu w małych zespołach. Grupy brały udział w sesjach opartych o~oba podejścia,
z~czego jedna najpierw uczestniczyła w tradycyjnej lekcji, a druga najpierw uczestniczyła w~sesji w escape roomie.

Wyniki zaprezentowane w artykule wskazują, że studenci uczestniczący w sesjach odbywających się w~wirtualnym escape
roomie wykazują się większym skupieniem i lepszym samopoczuciem. Największy wzrost wiedzy
odnotowano w grupie, która najpierw uczestniczyła w tradycyjnej lekcji matematyki, a po kilku dniach
wzięła udział w zajęciach przeprowadzanych przy użyciu omawianej aplikacji \cite{lebiedz2}.

\section{\glqq Empiriusz 2.0\grqq}

\subsection{Wykorzystane narzędzie}
\glqq Empiriusz 2.0\grqq~\cite{nowa-era-empiriusz} to narzędzie opracowane przez wydawnictwo \glqq Nowa Era\grqq, które wykorzystuje technologię 
VR do wsparcia nauki w szkołach podstawowych i ponadpodstawowych. Platforma, która jest udostępniana razem z~narzędziem,
oferuje kilka aplikacji zatytułowanych kolejno:
\begin{itemize}
    \item \glqq Wirtualne laboratorium chemiczne\grqq
    \item \glqq Wirtualny atlas anatomiczny\grqq
    \item \glqq Geometria przestrzenną\grqq
    \item \glqq Magiczna komnata\grqq
    \item \glqq Pierwsza pomoc -- 4 HELP VR\grqq
    \item \glqq Ziemia i Wszechświat PRO\grqq
\end{itemize}
Jak deklaruje wydawnictwo -- aplikacje są zgodne z podstawą programową, obowiązującą w szkołach podstawowych
i~ponadpodstawowych \cite{nowa-era-empiriusz}.

Urządzenia, na których działa platforma, to bezprzewodowe gogle VR i dwa kontrolery. Gracz wkracza do 
wirtualnej przestrzeni roboczej, a jego działania są na bieżąco obserwowane przez nauczyciela oraz 
pozostałych uczniów znajdujących się w klasie, dzięki wyświetlaniu obrazu z~perspektywy gracza, 
np. na tablicy interaktywnej lub monitorze. Wydawnictwo dostarcza również materiały edukacyjne 
i~karty pracy oraz oferuje szkolenia dla nauczycieli z obsługi platformy. Podkreślony jest również pozytywny
charakter edukacyjny tego rozwiązania omawiany dalej w \ref{subsec:empiriusz-wartosc}. 

\subsection{Aplikacja do nauki geometrii przestrzennej}
Geometria przestrzenna jest jednym z tematów opracowanych na potrzeby platformy \glqq Empiriusz~2.0\grqq.~Oferuje pomoc dydaktyczną przez 
rozszerzenie standardowej lekcji matematyki o~interaktywną aplikację z zadaniami dla szkół podstawowych i ponadpodstawowych \cite{nowa-era-geometria}.
Na stronie internetowej wydawnictwa możemy przeczytać kilkukrotne wzmianki, o tym że: \textit{\glqq aplikacja zwiększa motywację młodych ludzi do nauki i~ułatwia 
im przyswajanie zagadnień z zakresu geometrii\grqq}~\cite{nowa-era-geometria}. Teza ta pokrywa się z celami tej pracy omówionymi w~rozdziale
numer \ref{chap:introduction} i jest dalej omawiana w~\ref{subsec:empiriusz-wartosc}.

Na rysunku \ref{fig:geometria_empiriusz} możemy zobaczyć zakres tematyki, który obejmuje aplikacja. Dla szkół ponadpodstawowych
wydawnictwo przewiduje szerszą tematykę i więcej funkcjonalności, w skład których wchodzą m.in. możliwość nauki o siatkach brył
przestrzennych, ich przekroju, kątach w~bryłach i dostęp do interaktywnych wzorów.

\begin{figure}[htbp]
    \centering
    \includegraphics[width=0.6\textwidth]{images/EMPIRIUSZ.jpg}
    \caption{Skład zagadnień omawianych dla aplikacji o geometrii przestrzennej dostępnej na platformie \glqq Empiriusz 2.0\grqq~\cite{nowa-era-geometria}}
    \label{fig:geometria_empiriusz}
\end{figure}
\FloatBarrier

\subsection{Wartość edukacyjna narzędzia \glqq Empiriusz 2.0\grqq}
\label{subsec:empiriusz-wartosc}
Atrakcyjność i pozytywny wpływ aplikacji VR, w kontekście zastosowania narzędzia \glqq Empiriusz~2.0\grqq~są podkreślane
wielokrotnie przez wydawnictwo. Platforma ma oferować nowoczesne narzędzie dydaktyczne, uatrakcyjniające
naukę \cite{nowa-era-artykul}. Elementem, który wyróżnia to rozwiązanie jest ciągła obecność nauczyciela podczas lekcji, która
odbywa się przy użyciu wirtualnej rzeczywistości. Może on komentować i omawiać poszczególne posunięcia ucznia, który wykonuje zadania.
Dodatkowo nauczyciel jest wyposażony w materiały dydaktyczne, bezpośrednio powiązane z omawianymi zagadnieniami.

\section{Wnioski z analizy dostępnych rozwiązań}

Omawiane rozwiązania pokazują, że technologia VR może pozytywnie wpływać nie tylko na efekty edukacyjne, ale także
na samopoczucie i nastrój uczniów.

Zgodnie z rozdziałem numer \ref{chap:introduction}. celem tej pracy jest stworzenie narzędzia, które będzie
wspierać proces nauczania, a nie go zastępować. Ma zwiększyć zainteresowanie matematyką wśród uczniów szkół średnich
poprzez ich samodzielne uczestnictwo w~angażującym środowisku.