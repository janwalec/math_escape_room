\chapter{Analiza istniejących rozwiązań (Jan Walczak)}
\label{chap:algs}

\section{Wirtualny pokój zagadek z zakresu matematyki}

\subsection{Projekt i implementacja}
Wirtualny pokój zagadek z zakresu matematyki został opracowany na potrzeby publikacji
w~\glqq Zeszytach Naukowych Wydziału Elektroniki i Automatyki Politechniki Gdańskiej\grqq~\cite{lebiedz1}.

Artykuł opisuje innowacyjny projekt edukacyjny, polegający na stworzeniu matematycznego pokoju zagadek
(escape room) na poziomie studiów inżynierskich. Jak możemy przeczytać we wskazanym artykule,
celem pracy było stworzenie aplikacji na platformie Unity, która miałaby działać w Laboratorium
Zanurzonej Wizualizacji Przestrzennej i uatrakcyjnić studentom naukę matematyki \cite{lebiedz1}.

Zagadka została podzielona na 4 pokoje, z czego pierwszy z nich był pokojem wprowadzającym,
prowadził on do 3 kolejnych pokoi zawierających łącznie 13 zagadek. Pokoje były podzielone nie tylko
ze względu na tematykę zagadnień, ale także ze względu na wystrój \cite{lebiedz1}. Podział ten wyglądał następująco:

\begin{enumerate}
    \item Pokój w stylu nowoczesnym: zawierał zadania z zakresu m.in. wyznaczników macierzy (metoda Sarrusa), równań płaszczyzny oraz wykresów funkcji.
    \item Pokój warsztatowy: skupiał się na ciągach liczbowych (np. ciągu Fibonacciego), liczbach zespolonych, schemacie Hornera oraz systemie binarnym.
    \item Pokój w stylu egipskim: oferował zadania dotyczące działań na liczbach zespolonych oraz układów równań liniowych rozwiązywanych metodą Gaussa-Jordana.
\end{enumerate}

\subsection{Przeprowadzone badania}

Przy użyciu wcześniej omawianego rozwiązania został przeprowadzony eksperyment obejmujący
grupę 54 studentów. Badanie wykazało, że nauka poprzez rozwiązywanie zagadek w matematycznym
escape roomie przynosi wymierne korzyści \cite{lebiedz1}.

Badanie zostało rozszerzone i opisane w artykule \glqq Educational values of a virtual escape room in mathematics\grqq~\cite{lebiedz2}. Badanie polegało na podzieleniu studentów na 2 równe grupy, na których zostały zastosowane dwa
oddzielne podejścia stosowane w różnej kolejności. Pierwsze podejście polegało na uczestnictwie grupy w tradycyjnej lekcji matematyki,
a~drugie na przejściu matematycznego escape roomu w małych zespołach. Grupy brały udział w sesjach opartych o oba podejścia,
z~czego jedna najpierw uczestniczyła w tradycyjnej lekcji, a druga najpierw uczestniczyła w~sesji w escape roomie.

Wyniki zaprezentowane w artykule wskazują, że studenci uczestniczący w sesjach odbywających się w~wirtualnym escape
roomie wykazują się większym skupieniem i lepszym samopoczuciem. Największy wzrost wiedzy
odnotowano w grupie, która najpierw uczestniczyła w tradycyjnej lekcji matematyki, a po kilku dniach
wzięła udział w zajęciach przeprowadzanych przy użyciu omawianej aplikacji \cite{lebiedz2}.

\section{Wnioski z analizy dostępnych rozwiązań}

Rozwiązanie omawiane w poprzednim podrozdziale pokazuje, że głęboka imersja, którą oferuje
Laboratorium Zanurzonej Wizualizacji Przestrzennej, poprawia samopoczucie oraz nastrój studentów \cite{lebiedz2}.
Zgodnie z rozdziałem numer \ref{chap:introduction}. celem tej pracy jest stworzenie narzędzia, które będzie
wspierać proces nauczania, a nie go zastępować. Ma zwiększyć zainteresowanie matematyką wśród uczniów szkół średnich
poprzez ich samodzielne uczestnictwo w~angażującym środowisku.