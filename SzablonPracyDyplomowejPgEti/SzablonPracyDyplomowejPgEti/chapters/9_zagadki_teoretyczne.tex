\chapter{Opis teoretyczny zagadek}
\label{chap:theory}

Dzień dobry
Tu będą zagadki

\section{Wzory skróconego mnożenia}

tekst tekst tekst

\section{Planimetria (Jan Walczak)}
Po ukończeniu pierwszego zadania plansza z kafelkami znika, a na jej miejscu pojawia się tabela z trzema kolumnami.
Dwie skrajne kolumny zawierają nazwy figur geometrycznych. Środkowa kolumna jest pusta i będzie uzupełniana przez gracza 
odpowiednimi symbolami.  
\subsection{Cel zadania}
Celem gracza jest poprawne ułożenie relacji między figurami w kolejnych rzędach. 
Przykładowo, relacją taką jest to, że „każdy kwadrat jest rombem” lub „każdy kwadrat jest prostokątem”.
\subsection{Zasady działania}
Gracze przechodzą przez kolejne rzędy sekwencyjnie. Dopóki nie ułożą pierwszej relacji poprawnie,
to nie mogą ułożyć kolejnej. Po poprawnym ułożeniu rzędu (wstawieniu odpowiedniego symbolu) podświetla się on na zielono, 
sygnalizując graczowi, że może przejść do kolejnego punktu. Symbole, które układają gracze są następujące:
\begin{figure}[htbp]
    \centering
    \includegraphics[width=0.7\textwidth]{images/zad2.1.png}
    \caption{Przykładowy wygląd tabeli}
\end{figure}
\\Gracze kierują się swoją wiedzą matematyczną oraz następującymi własnościami figur:
\begin{itemize}
    \item liczba boków,
    \item długości boków,
    \item kąty (proste lub nie),
    \item cechy charakterystyczne danej figury.
\end{itemize}
Po zaznaczeniu wszystkich relacji zadanie uznaje się za wykonane.

\subsection{Zakończenie zadania}

Poprawne rozwiązanie powoduje rozsunięcie się blatu stołu, który odsłania kolejne wyzwanie.

