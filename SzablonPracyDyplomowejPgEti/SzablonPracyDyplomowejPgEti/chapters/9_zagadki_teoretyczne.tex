\chapter{Wstępny projekt zagadek}
\label{chap:theory}


W poniższym rozdziale znajduje się teoretyczny opis zagadek, który stanowi punkt wejściowy do implementacji
projektu inżynierskiego. Stanowi on podstawę teoretyczną i nakreśla charakter pracy. Opisy poszczególnych zagadek
zostały zredagowane na podstawie ogólnodostępnej podstawy programowej dla szkół średnich z przedmiotu matematyka.
Jest to kluczowy aspekt całego projektu inżynierskiego, gdyż aplikacja końcowa ma być skierowana do uczniów
szkół średnich, w związku z czym wymagany zakres wiedzy nie może wykraczać poza podstawę programową.











\section{Zadanie 1 – Wzory skróconego mnożenia (Andrii Demyshyn)}
\subsection{Cel zadania}
Celem zadania jest utrwalenie wiedzy z zakresu wzorów skróconego mnożenia oraz rozwijanie umiejętności rozpoznawania poprawnych zależności algebraicznych. Zadanie ma charakter wprowadzający i pozwala uczestnikowi na przypomnienie  sobie podstawowych wzorów omawianych w programie nauczania matematyki na poziomie szkoły średniej. Dodatkowo zadanie wspiera rozwój logicznego myślenia oraz umiejętność analizy struktury wyrażeń algebraicznych.
\subsection{Zasady działania}
Zadanie polega na dopasowaniu początków wzorów skróconego mnożenia do ich poprawnych zakończeń. Uczestnik otrzymuje zestaw rozpoczętych wyrażeń algebraicznych, w których brakują prawe strony równań. Równocześnie dostępny jest zbiór możliwych zakończeń wzorów, wśród których znajdują się zarówno poprawne, jak i niepoprawne zapisy algebraiczne (rys.~\ref{fig:wzory}).
\begin{figure}[H]
    \centering
    \includegraphics[width=0.8\textwidth]{images/wzory.png}
    \caption{Przykładowy wygląd pierwszego zadania – wzory skróconego mnożenia}
    \label{fig:wzory}
\end{figure}
Celem użytkownika jest wybranie właściwych elementów i połączenie ich w taki sposób, aby utworzyć kompletne i matematycznie poprawne wzory skróconego mnożenia. Zadanie wymaga znajomości podstawowych własności działań algebraicznych oraz umiejętności odróżniania poprawnych wzorów od błędnych.
\subsection{Zakończenie zadania}
Zadanie uznaje się za zakończone w momencie poprawnego uzupełnienia wszystkich wzorów skróconego mnożenia. Prawidłowe rozwiązanie potwierdza, że uczestnik posiada wymaganą wiedzę teoretyczną oraz potrafi ją zastosować w praktyce poprzez rozpoznawanie i kompletowanie wyrażeń algebraicznych.














\section{Zadanie 2 – Planimetria (Jan Walczak)}
\label{subsec:planimetria_teoria}
Po ukończeniu pierwszego zadania pojawia się tabela z trzema kolumnami.
Dwie skrajne kolumny zawierają nazwy figur geometrycznych. Środkowa kolumna jest pusta i będzie uzupełniana przez gracza 
odpowiednimi symbolami.  
\subsection{Cel zadania}
Celem gracza jest poprawne ułożenie relacji między figurami w kolejnych rzędach. 
Przykładowo, relacją taką jest to, że „każdy kwadrat jest rombem” lub „każdy kwadrat jest prostokątem”.
\subsection{Zasady działania}
Gracze przechodzą przez kolejne rzędy sekwencyjnie (rys. \ref{fig:przyklad}). Dopóki nie ułożą pierwszej relacji poprawnie,
to nie mogą ułożyć kolejnej. Po poprawnym ułożeniu rzędu (wstawieniu odpowiedniego symbolu) podświetla się on na zielono, 
sygnalizując graczowi, że może przejść do kolejnego punktu. Symbole, które układają gracze są następujące:
\begin{itemize}
    \item > to <
    \item > to nie <
\end{itemize}
\begin{figure}[htbp]
    \centering
    \begin{tabular}{|c|c|c|}
        \hline
        Kwadrat & > to < & Prostokąt \\ \hline
        Prostokąt & > to nie < & Kwadrat \\ \hline
        Romb & [      ] & Kwadrat \\ \hline
        ... & ... & ... \\ \hline
    \end{tabular}
    \caption{Przykładowy diagram zależności między figurami geometrycznymi.}
    \label{fig:przyklad}
\end{figure}

\medskip
  
Gracze kierują się swoją wiedzą matematyczną oraz następującymi własnościami figur:

\begin{itemize}[left=1.5em, label=\textbullet, topsep=0pt, itemsep=0pt]
    \item liczba boków,
    \item długości boków,
    \item kąty (proste lub nie),
    \item cechy charakterystyczne danej figury.
\end{itemize}
\subsection{Zakończenie zadania}
Po poprawnym uzupełnieniu wszystkich relacji zadanie uznaje się za zakończone. Gracz może przejść
do kolejnego zadania.













\section{Zadanie 3 – Nierówności (Konrad Czarnecki)}
\label{subsec:nierownosci_teoria}
Po wejściu do kolejnego pomieszczenia gracz widzi przed sobą most zawieszony nad przepaścią.
Po drugiej stronie znajduje się zamknięte przejście, do którego gracz musi się dostać.
Na bocznych ścianach wypisany jest układ nierówności:
    $2x - 5 < 9$;
    $x + 1 \ge 4$.
Na płytkach mostu umieszczone są różne liczby z zakresu liczb całkowitych (np.: $1$, $2$, $3$, $4$, $5$, $6$, $8$, $7$, $10$).
\subsection{Cel zadania}
Gracz musi rozwiązać układ nierówności i wyznaczyć przedział, który spełnia oba warunki. Następnie powinien przeprowadzić
niewielką figurkę przez most, odpowiednio nią poruszając, tak aby przesuwała się wyłącznie po płytkach z wartościami należącymi
do tego przedziału. Poruszanie figurką odbywa się za pomocą czterech przycisków (w prawo, w lewo, do przodu i do tyłu) znajdujących
się na jednej ze ścian.
\medskip
Rozwiązanie układu
\begin{enumerate}[left=1.5em, topsep=0pt, itemsep=0pt]
    \item Rozwiązanie pierwszej nierówności:
    \begin{itemize}[left=1.5em, label=\textbullet, topsep=0pt, itemsep=0pt]
        \item $2x - 5 < 9$
        \item $2x < 14$
        \item $x < 7$
    \end{itemize}
    \item Rozwiązanie drugiej nierówności:
    \begin{itemize}[left=1.5em, label=\textbullet, topsep=0pt, itemsep=0pt]
        \item $x + 1 \ge 4$
        \item $x \ge 3$
    \end{itemize}
    \item Wspólny przedział:
    \begin{itemize}[left=1.5em, label=\textbullet, topsep=0pt, itemsep=0pt]
        \item \textit{$x \in [3; 7)$}
    \end{itemize}
\end{enumerate}
\medskip
Gracz musi wybrać wyłącznie płytki z wartościami większymi lub równymi $3$ i mniejszymi od $7$, np. $3$, $4$, $5$, $6$.
\subsection{Zasady działania}
\begin{itemize}[left=1.5em, label=\textbullet, topsep=0pt, itemsep=0pt]
    \item Gracz poruszając postacią, przechodzi przez kolejne płytki.
    \item Poprawna płytka – postać przechodzi dalej.
    \item Błędna płytka – płytka zapada się lub podświetla na czerwono, a postać wraca na początek mostu. Gracz może próbować dowolną liczbę razy, aż do skutecznego przejścia na drugą stronę.
\end{itemize}
\subsection{Zakończenie zadania}
Zadanie zostaje uznane za zakończone, gdy graczowi uda się przejść na drugą stronę mostu.



















\section{Zadanie 4 – Funkcje (Konrad Czarnecki)}
\label{subsec:funkcje_teoria}
Zadania 4 i 5 są realizowane w jednym pomieszczeniu. Gracz widzi dwie duże tablice, umieszczone na ścianach. Na obu naniesiona jest siatka układu współrzędnych. W układzie współrzędnych pojawiają się trzy punkty, np.:
\begin{itemize}[left=1.5em, label=\textbullet, topsep=0pt, itemsep=0pt]
    \item Punkt $A (1, 2)$
    \item Punkt $B (3, 4)$
    \item Punkt $C (5, 6)$
\end{itemize}
oraz wzór funkcji kwadratowej ze wszystkimi współczynnikami domyślnie ustawionymi na $0$.
Poniżej widoczne są elementy służące do sterowania wykresem funkcji. Dają one możliwość zmiany współczynników wylosowanego wzoru.
\subsection{Cel zadania}
Celem gracza jest dobranie odpowiednich współczynników funkcji krawdatowej za pomocą wirtualnego suwaka tak, aby przeszła ona przez wszystkie wyświetlone punkty. 
\subsection{Zasady działania}
Podczas wybierania współczynników funkcji przez gracza, jej wykres na tablicy zmienia się na bieżąco zgodnie z ustawionym wzorem. Wykres funkcji powinien być losowany z wcześniej zdefiniowanej puli par typu funkcja – punkty, tak aby po każdorazowym uruchomieniu zadania gracz czuł, że ma przed sobą nowe wyzwanie.
\subsection{Zakończenie zadania}
Rozwiązanie jest sprawdzane na bieżąco i gdy gracz dobierze współczynniki funkcji kwadratowej poprawnie, tj. przetnie ona wszystkie wybrane punkty, zadanie zostaje uznane za rozwiązane. W takim przypadku tablica się blokuje i podświetla na zielono.

















\section{Zadanie 5 – Geometria analityczna (Konrad Czarnecki)}
\label{subsec:geometria_analityczna_teoria}
W układzie współrzędnych pojawiają się dwa wykresy funkcji liniowych. Są one losowane z pewnej określonej puli, podobnie jak w zadaniu 4. 
\subsection{Cel zadania}
Zadaniem gracza jest znalezienie punktu przecięcia dwóch funkcji liniowych wyświetlanych w układzie współrzędnych. Określenie puli wyklucza możliwość prostych równoległych, które nie mają ze sobą żadnych punktów wspólnych lub w całości się pokrywają. W tych przypadkach zadanie byłoby niemożliwe do rozwiązania. Gracz będzie wpisywał swoją odpowiedź (współrzędne punktu przecięcia) na klawiaturze znajdującej się pod tablicą.
\subsection{Zasady działania}
Skala osi współrzędnych jest dobrana tak, by gracz nie mógł odczytać z niej rozwiązania; musi rozwiązać układ równań dysponując dwoma wzorami funkcji. Przykładowe dwa wykresy przecinających się funkcji (rys. \ref{fig:przykladowy_wykres_funkcji}): 
\begin{itemize}[left=1.5em, label=\textbullet, topsep=0pt, itemsep=0pt]
    \item $y = 3x + 4$ (czerwony)
    \item $y = 5x + 8$ (niebieski)
\end{itemize}
\begin{figure}[htbp]
    \centering
    \includegraphics{images/przykładowy_wykres_funkcji.png}
    \caption{Przykładowy wygląd tablicy z przecinającymi się wykresami}
    \label{fig:przykladowy_wykres_funkcji}
\end{figure}
Gracz dysponuje tymi wzorami i na ich podstawie musi rozwiązać układ równań. Wykresy powinny być dobrane tak, aby obie współrzędne punktu przecięcia były liczbami całkowitymi. 
\subsection{Zakończenie zadania}
Rozwiązanie jest sprawdzane na bieżąco i gdy gracz dobierze współrzędne punktu przecięcia funkcji poprawnie, zadanie zostaje uznane za rozwiązane. W takim przypadku tablica się blokuje i podświetla na zielono. Jeśli gracz wykonał wcześniej zadanie czwarte (znajdujące się w tym samym pokoju), przechodzi do następnego zadania.
















\section{Zadanie 6 – Kombinatoryka (Andrii Demyshyn)}

\subsection{Cel zadania}
Gracz zostaje poinformowany, że musi wyliczyć, ile jest możliwych do ułożenia trzycyfrowych kodów do sejfu bez powtarzających się cyfr.

\subsection{Zasady działania}
Na środku pokoju znajduje się zamknięty sejf (rys.~\ref{fig:sejf}). Rozwiązaniem jest kod, a nie liczba – oznacza to, że początkową cyfrą w trzycyfrowym kodzie może być zero. Informacja ta powinna być jasno zakomunikowana graczowi, np. na tabliczce umieszczonej nad sejfem.

Rozwiązaniem zadania jest:
\[
10 \cdot 9 \cdot 8 = 720
\]

\subsection{Zakończenie zadania}
Po poprawnym wpisaniu kodu sejf automatycznie się otwiera, co utwierdza gracza w przekonaniu, że poprawnie wykonał zadanie. W środku znajdują się elementy potrzebne do wykonania następnego zadania.

\begin{figure}[H]
    \centering
    \includegraphics[width=0.8\textwidth]{images/sejf.png}
    \caption{Przykładowy wygląd sejfu występującego w zadaniu}
    \label{fig:sejf}
\end{figure}















\section{Zadanie 7 – Liczby rzeczywiste i działania na zbiorach liczbowych \\(Jan Walczak)}
\label{subsec:liczby_rzeczywiste}

W następnym zadaniu gracze znajdują skrzynię, na której powierzchni – z każdej strony – narysowane są liczby:
\textit{-3; 7; 12; 2; 5; 0; 10; 3,75; √2}.

\subsection{Cel zadania}
Skrzynia wyposażona jest w mechaniczny zamek z pięcioma polami na cyfry, obok których znajdują się symbole zbiorów:
\begin{itemize}
    \item \textit{N} – liczby naturalne,
    \item \textit{Z} – liczby całkowite,
    \item \textit{R} – liczby rzeczywiste,
    \item \textit{Q} – liczby wymierne,
    \item \textit{R\textbackslash Q} – liczby niewymierne,
\end{itemize}
Gracz musi uważnie obejrzeć skrzynię i policzyć ile liczb, zapisanych na skrzyni, należy do jakiego zbioru. 
Po wpisaniu odpowiednich liczb, obok symboli zbiorów, skrzynia otwiera się, a gracz otrzymuje kilka symboli: 
\begin{itemize}
    \item zawieranie się zbiorów – $\subset$,
    \item zbiór pusty – $\varnothing$,
    \item iloczyn zbiorów – $\cap$.
\end{itemize}
Tym samym gracz przechodzi do drugiego etapu zadania. W drugim etapie w pokoju ukazuje się plansza z symbolami zbiorów, takimi jak na skrzyni. 
Celem gracza jest ułożenie poprawnej relacji między nimi tj. zbiór liczb naturalnych zawiera się w zbiorze liczb całkowitych, zbiór liczb całkowitych zawiera się w zbiorze liczb rzeczywistych itd.

\subsection{Zasady działania}
W pierwszym etapie zadania gracz powinien policzyć, ile liczb ze skrzyni należy do danego zbioru 
i wpisać odpowiednią odpowiedź na kłódce. Przykładowo:
\begin{itemize}
    \item \textit{N} (naturalne): \textit{ 7; 12; 2; 5; 10} – 5 liczb,
    \item \textit{C} (całkowite):  \textit{-3; 7; 12; 2; 5; 0; 10} – 7 liczb,
    \item \textit{R} (rzeczywiste): wszystkie liczby – 9 liczb,
    \item \textit{Q} (wymierne): \textit{-3; 7; 12; 2; 5; 0; 10; 3,75} – 8 liczb,
    \item \textit{R\textbackslash Q} (niewymierne): \textit{√2} – 1 liczba.
\end{itemize}
Kombinacja do ustawienia na kłódce: \textit{5, 7, 9, 8, 1}.

W drugim etapie gracz powinien ustawić posiadane symbole między literami symbolizującymi kolejne zbiory i odpowiednio je obrócić, tak aby powstała między nimi poprawna relacja tj.

\begin{itemize}
    \item $N \subset Z \subset R$,
    \item $R\textbackslash Q \cap Q = \varnothing$,
    \item $R\textbackslash Q \cap R = R\textbackslash Q$.
\end{itemize}

\subsection{Zakończenie zadania}
Zadanie zostaje uznane za zakończone, gdy gracz poprawnie przejdzie przez oba etapy. 
Gracz nie może przejść do etapu drugiego, bez zakończenia etapu pierwszego – jest to wymuszone 
przez wymaganie otwarcia przez niego skrzyni.
















\section{Zadanie 8 – Znaki funkcji trygonometrycznych (Konrad Czarnecki)}
\label{subsec:znaki_funkcji_trygonometrycznych_teoria}
Po ukończeniu zadania ze zbiorami gracz kieruje się do ściany z czterema dużymi okręgami jednostkowymi, oznaczonymi nazwami funkcji:
\begin{itemize}[left=1.5em, label=\textbullet, topsep=0pt, itemsep=0pt]
    \item $sin(\alpha)$
    \item $cos(\alpha)$
    \item $tg(\alpha)$
    \item $ctg(\alpha)$
\end{itemize}
Każdy z okręgów podzielony jest na cztery ćwiartki, oznaczone jako $I$, $II$, $III$, $IV$.
Przy każdej ćwiartce znajduje się puste miejsce, które gracz musi wypełnić odpowiednim znakiem (rys. \ref{fig:cwiartki_ukladow_jednostkowych}):
\begin{itemize}[left=1.5em, label=\textbullet, topsep=0pt, itemsep=0pt]
    \item „+” (plus) — funkcja przyjmuje wartości dodatnie
    \item „−” (minus) — funkcja przyjmuje wartości ujemne
\end{itemize}
Gracz zmienia znak naciskając na niego. Domyślnie wszystkie znaki są ustawione na puste, dopiero po pierwszym naciśnięciu zmieniają się na znak „+”, po kolejnym na „−”, następnie znów na „+”, itd.
\begin{figure}[htbp]
    \centering
    \includegraphics[width=\textwidth]{images/cwiartki_ukladow_jednostkowych.png}
    \caption{Ćwiartki układow jednostkowych ze znakami funkcji trygonometrycznych}
    \label{fig:cwiartki_ukladow_jednostkowych}
\end{figure}
\subsection{Cel zadania}
Celem gracza jest poprawne ustawienie znaków funkcji trygonometrycznych w każdej ćwiartce układu współrzędnych.

\subsection{Zasady działania}
Gracz modyfikuje znaki „+” i „−” w odpowiednich miejscach na planszy naciskając na nie. Może dowolnie poprawiać swój wybór, dopóki nie zatwierdzi odpowiedzi.

\subsection{Zakończenie zadania}
Po poprawnym ułożeniu wszystkich znaków ściana rozświetla się na zielono.















\section{Zadanie 9 – Ciągi liczbowe (Jan Walczak)}
\label{sec:sequence}
Poniżej opisane zadanie będzie podobne do gry wyprodukowanej przez Nintendo na platformę Pegasus (rys. \ref{duckhunt}).

\subsection{Cel zadania}
Zadaniem gracza będzie odpowiednio szybko obliczyć kolejne wyrazy ciągu arytmetycznego 
z podanego wzoru. Będzie on wyposażony w wirtualny, laserowy pistolet, którym będzie musiał strzelać
w odpowiednio oznaczone kaczki.
\subsection{Zasady działania}
Na jednej ze ścian pokoju pojawia się formuła ciągu arytmetycznego, np.: $a_n = 2n + 1$ wybranego z predefiniowanej puli. 
Gracz będzie musiał wybrać i strzelić do kaczki oznaczonej odpowiednią wartością kolejnych wyrazów ciągu. Gra przyspiesza (kaczki lecą coraz szybciej) razem z postępem w zadaniu. 
Aby ułatwić graczowi zadanie, na górze jednej ze ścian pokoju będzie podany nie tylko wzór ciągu, ale też aktualny numer wyrazu tego ciągu. 
Przykładowo, dla wzoru $a_n = 2n + 1$:
\begin{itemize}
    \item $n = 1$ – gracz musi trafić w kaczkę z liczbą 3,
    \item $n = 2$ – gracz musi trafić w kaczkę z liczbą 5
\end{itemize}

\subsection{Zakończenie zadania}
Gracz będzie zdobywał punkty za każdy poprawnie wybrany wyraz ciągu. 
Gra zakończy się po upływie określonego czasu lub jeśli gracz pomyli się trzy razy.

\begin{figure}[htbp]
    \centering
    \includegraphics{images/duckhunt.jpg}
    \caption{gra Duck Hunt}
    \label{duckhunt}
\end{figure}







\section{Zadanie 10 – Prawdopodobieństwo (Andrii Demyshyn)}
Gracze podchodzą do stołu ustawionego w rogu pokoju, gdzie siedzi postać Morfeusza. Na stole stoją dwa identyczne pojemniki, a obok leży 100 tabletek — 50 czerwonych i 50 niebieskich (rys.~\ref{fig:Morfeus}).

Pojawia się komunikat:
„Pomóż Morfeuszowi zwiększyć jego szanse na powrót do rzeczywistości.
Podziel tabletki między dwa pojemniki tak, aby miał jak największą szansę na wybranie czerwonej.”

\subsection{Cel zadania}
Gracze muszą znaleźć optymalny układ, czyli:
w pierwszym pojemniku umieścić jedną czerwoną tabletkę,
w drugim pojemniku umieścić 49 czerwonych i 50 niebieskich tabletek (lub odwrotnie. Rozwiązanie działa niezależnie od tego, który pojemnik traktujemy jako „pierwszy”, a który jako „drugi”).
Taki układ osiągnie maksymalne prawdopodobieństwo wylosowania czerwonej tabletki – 74,75\%.

\subsection{Zasady działania}
Gracze mogą:
przeciągać tabletki do pojemników w dowolny sposób,
testować różne rozkłady,
sprawdzać procent szans na wygraną, który wyświetla się po każdym rozłożeniu.

\begin{figure}[H]
    \centering
    \includegraphics[width=0.8\textwidth]{images/morfeus.png}
    \caption{Przykładowy wygląd w zadaniu}
    \label{fig:Morfeus}
\end{figure}

Po każdym rozkładzie system oblicza i wyświetla szansę na wygraną.
Jeśli gracz nie osiągnie 74,75\%, pojawia się komunikat:
„Możesz to zrobić lepiej! Spróbuj jeszcze raz.”
Jeśli gracz osiągnie 74,75\% system gratuluje i zalicza zadanie.

\subsection{Zakończenie zadania}
Po poprawnym rozłożeniu tabletek i osiągnięciu maksymalnej szansy 74,75\%, Morfeusz wstaje od stołu, uśmiecha się i mówi:
„Dziękuję. Dzięki Wam mam szansę wrócić do rzeczywistego świata”.
Po chwili Morfeusz znika, rozpuszczając się w powietrzu niczym hologram lub efekt teleportacji, symbolizujące jego powrót do rzeczywistości.










\section{Zadanie 11 – Optymalizacja i rachunek różniczkowy (Autor)}
a
\subsection{Cel zadania}
a
\subsection{Zasady działania}
a
\subsection{Zakończenie zadania}


\section{Zadanie 12 – Układy równań (Andrii Demyshyn)}

\subsection{Cel zadania}
Na środku pomieszczenia znajduje się stół z trzema przezroczystymi pojemnikami, oznaczonymi kolorami: czerwonym (x), zielonym (y) oraz niebieskim (z). Bezpośrednio nad stołem, na ścianie, umieszczona jest duża żarówka, która zapala się, gdy gracze prawidłowo uzupełnią pojemniki (rys.~\ref{fig:Lamp}). Obok znajduje się instrukcja techniczna:
„Aby uruchomić instalację świetlną, należy załadować dokładnie 12 kulek. Wszystkie kulki razem dają 12 jednostek energii. Czerwona kulka jest dwa razy bardziej wydajna niż zielona. Czerwona i dwie zielone razem dają o 9 jednostek energii więcej niż niebieska”.

\subsection{Zasady działania}
\begin{itemize}
    \item Gracze wkładają kolorowe kulki do pojemników.
    \item System na bieżąco sprawdza poprawność ustawienia.
    \item W razie błędu wyświetlany jest komunikat: „Niepoprawne ustawienie. Spróbuj ponownie”.
\end{itemize}

\[
\begin{cases}
x = 2y \\
x + 2y - z = 9 \\
x + y + z = 12
\end{cases}
\quad
\begin{aligned}
&x &&\text{– energia generowana przez kulki czerwone}, \\
&y &&\text{– energia generowana przez kulki zielone}, \\
&z &&\text{– energia generowana przez kulki niebieskie}.
\end{aligned}
\]

\begin{itemize}
    \item Czerwony (x) = 6,
    \item Zielony (y) = 3,
    \item Niebieski (z) = 3.
\end{itemize}

\begin{figure}[H]
    \centering
    \includegraphics[width=0.4\textwidth]{images/Lamp.png}
    \caption{Przykładowy wygląd zadania}
    \label{fig:Lamp}
\end{figure}

\subsection{Zakończenie zadania}
Po poprawnym ułożeniu żarówka zapala się nad stołem, sygnalizując zakończenie zadania.









\section{Zadanie 13 – Stereometria (Jan Walczak)}
\label{stereometria:teoria}
Bryły przestrzenne mogą sprawić uczniom trudność, ponieważ wymagają przejścia od rysunku 
dwuwymiarowego tj. takiego na kartce papieru, do wyobrażenia sobie ich w przestrzeni trójwymiarowej.
\subsection{Cel zadania}
Celem zadania jest zapoznać uczniów z podstawowymi własnościami brył geometrycznych poprzez ich samodzielne odkrywanie.
\subsection{Zasady działania}
Gracz wchodzi do ciemnego pokoju i jest wyposażony w źródło światła np. pochodnię lub latarkę. 
Na środku pokoju znajduje się jedna z brył, wylosowanych z danej puli: prostopadłościan, ostrosłup, graniastosłup, kula lub sześcian.
Na ścianie pokoju znajduje się panel z pytaniami np.:
\begin{itemize}
    \item Jak nazywa się ta bryła?
    \item Jaki jest wzór na obliczenie jej pola powierzchni?
    \item Jaki jest wzór na obliczenie jej objętości?
\end{itemize}
Gracz obchodząc bryłę dookoła i rozświetlając ją latarką powinien móc udzielić odpowiedzi na te pytania. 
\subsection{Zakończenie zadania}
Gdy gracz udziela poprawnych odpowiedzi na kolejne pytania zostają one podświetlone na zielono, sygnalizując graczowi, że wykonał zadanie poprawnie. 
Jeżeli się pomyli to losowana jest kolejna bryła, inna niż poprzednia. Gra kończy się w momencie kiedy gracz udzieli poprawnych
odpowiedzi na pytania dotyczące dwóch różnych brył.