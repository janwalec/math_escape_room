\chapter{Sprawozdanie z działania aplikacji (Andrii Demyshyn)}
\label{Sprawozdanie_z_działania_aplikacji}

W ramach realizacji projektu opracowano trzynaście interaktywnych zagadek matematycznych, z których każda reprezentuje inny dział matematyki nauczanej na poziomie szkoły średniej. Zagadki zostały zaprojektowane tak, aby nie tylko sprawdzać wiedzę użytkowników, ale również aktywizować ich poprzez wykorzystanie mechanik charakterystycznych dla gier logicznych i escape roomów. Dzięki zastosowaniu technologii wirtualnej rzeczywistości, gracze mogą wchodzić w bezpośrednią interakcję z otoczeniem i rozwiązywać zadania w formie angażujących, przestrzennych łamigłówek. Każde z zadań posiada unikalny scenariusz i zasady działania, dostosowane do specyfiki danego działu matematyki oraz do możliwości sprzętowych Laboratorium Zanurzonej Wizualizacji Przestrzennej. Poniżej przedstawiono szczegółowy opis wszystkich trzynastu zagadek, wraz z określeniem ich celu, przebiegu oraz warunków zakończenia.
\section{Wzory skróconego mnożenia}
Pierwsze zadanie wirtualnego escape roomu polega na rozpoznawaniu oraz uzupełnianiu wzorów skróconego mnożenia. Po pojawieniu się gracza w wirtualnym pomieszczeniu widoczna jest tablica zawierająca początki wybranych wzorów algebraicznych. Każda formuła przedstawiona jest w postaci lewej strony równania wraz ze znakiem „=”, po którym pozostawiono puste miejsce przeznaczone na brakującą część wyrażenia.
W przestrzeni wirtualnego pokoju rozmieszczone są interaktywne bloki zawierające możliwe zakończenia wzorów. Wśród dostępnych elementów znajdują się zarówno poprawne zakończenia odpowiadające właściwym wzorom skróconego mnożenia, jak i bloki z nieprawidłowymi zapisami, które pełnią funkcję elementów rozpraszających. Zadaniem gracza jest przeciąganie wybranych bloków oraz umieszczanie ich w odpowiednich miejscach na tablicy, tak aby utworzyć kompletne i matematycznie poprawne formuły. Widok sceny zadania przedstawiono na rysunku \ref{fig:Wzoryalg}.
\begin{figure}[H]
    \centering
    \includegraphics[width=0.7\textwidth]{images/wzoryalg.jpg}
    \caption{ Uzupełniania wzorów skróconego mnożenia }
    \label{fig:Wzoryalg}
\end{figure}
Po uzupełnieniu wszystkich pustych pól użytkownik ma możliwość uruchomienia przycisku sprawdzającego. Po jego użyciu system dokonuje weryfikacji poprawności skonstruowanych równań i wyświetla komunikat tekstowy informujący o wyniku zadania. Zagadkę uznaje się za rozwiązaną w momencie prawidłowego dopasowania wszystkich zakończeń do odpowiadających im początków wzorów.

\section{Planimetria}

Po ukończeniu pierwszego zadania na ścianach pokoju pojawiają się figury geometryczne. Wśród nich znajdują się m.in. kwadraty, romby, prostokąty, równoległoboki, trapezy oraz różne typy trójkątów. Zadanie ma na celu sprawdzenie znajomości zależności między figurami oraz umiejętności rozpoznawania ich cech charakterystycznych. Na ścianie centralnej, pomiędzy prezentowanymi zbiorami figur, znajdują się przyciski służące do wyboru relacji. Gracze analizują właściwości figur i decydują, czy każda figura z czerwonego zbioru jest jednocześnie figurą należącą do zielonego zbioru. Do dyspozycji mają dwa przyciski: \textit{Każdy} – oznaczający, że relacja jest prawdziwa, oraz \textit{Nie każdy } – oznaczający, że relacja jest fałszywa. Przykładowy układ figur oraz wybór relacji pokazano na rys.~\ref{fig:Planimetria23}. Po udzieleniu odpowiedzi system natychmiast weryfikuje poprawność wskazanej relacji. W przypadku błędnej odpowiedzi gracz otrzymuje komunikat zwrotny, a następnie wyświetlana jest kolejna relacja do oceny. Zadanie kończy się w momencie, gdy wszystkie relacje zostaną poprawnie oznaczone. 
\begin{figure}[H]
    \centering
    \includegraphics[width=0.7\textwidth]{images/planimetria23.jpg}
    \caption{ Zadanie planimetryczne – relacje między figurami}
    \label{fig:Planimetria23}
\end{figure}


\section{Nierówności}

W trzecim zadaniu gracze trafiają na scenę z mostem zbudowanym z kamiennych płytek oraz metaliczną kostką umieszczoną na jego początku. Kostka musi przejść przez wiszący most, poruszając się po płytkach. Na każdej z płytek widnieje liczba, natomiast na ścianie wyświetlany jest układ dwóch nierówności. Widok mostu z kafelkami liczbowymi oraz układem nierówności przedstawiono na rys.~\ref{fig:Most23}.
\begin{figure}[H]
    \centering
    \includegraphics[width=0.6\textwidth]{images/most23.jpg}
    \caption{Zadanie z nierównościami – przejście przez most}
    \label{fig:Most23}
\end{figure}

 Zadaniem uczestników jest rozwiązanie układu nierówności oraz wybór płytek z wartościami należącymi do przedziału spełniającego oba warunki. Gracze prowadzą kostkę, korzystając z przycisków strzałek, które przesuwają ją na wybraną płytkę, a błędny wybór powoduje zapadnięcie się płytki i konieczność powrotu na początek lub do punktu kontrolnego. Po poprawnym przejściu przez most zagadka zostaje uznana za rozwiązaną, a gracze uzyskują dostęp do kolejnego etapu gry.
\section{Funkcje}

W czwartym zadaniu gracze pracują z dużą interaktywną tablicą, na której wyświetlany jest układ współrzędnych. Ich celem jest dostosowanie parametrów funkcji w taki sposób, aby jej wykres przechodził przez wyznaczone punkty. Interaktywną tablicę z wykresem funkcji pokazano na rys.~\ref{fig:Funkcja23}. Dzięki suwakom uczestnicy mogą dynamicznie zmieniać współczynniki funkcji, obserwując na bieżąco, jak zmienia się jej wykres. Zadanie polega na precyzyjnym dobraniu wartości parametrów funkcji tak, aby wykres przeciął wszystkie wskazane punkty.
\begin{figure}[H]
    \centering
    \includegraphics[width=0.6\textwidth]{images/funkce23.jpg}
    \caption{ Interaktywna tablica do modyfikacji wykresu funkcji }
    \label{fig:Funkcja23}
\end{figure}


\section{Geometria analityczna}

W piątym zadaniu gracze ponownie pracują z interaktywną tablicą, na której tym razem pojawiają się dwa wykresy funkcji. Celem wyzwania jest wyznaczenie współrzędnych punktu przecięcia wykresów funkcji na płaszczyźnie kartezjańskiej. Uczestnicy mogą odczytać ten punkt bezpośrednio z wykresu lub wyznaczyć go na podstawie równań opisujących obie funkcje. Po ustaleniu wartości współrzędnych gracze wybierają je za pomocą dedykowanych przycisków. System weryfikuje poprawność wskazanych współrzędnych i odblokowuje przejście do kolejnego etapu gry. Przykład wyznaczania punktu przecięcia wykresów funkcji przedstawiono na rys.~\ref{fig:Geometrija23}.
\begin{figure}[H]
    \centering
    \includegraphics[width=0.8\textwidth]{images/geometria23.jpg}
    \caption{ Interaktywna tablica do modyfikacji wykresu funkcji }
    \label{fig:Geometrija23}
\end{figure}


\section{Kombinatoryka}

W szóstym zadaniu gracze natrafiają na metalowy sejf wyposażony w podświetlany panel numeryczny. Na oddzielnym panelu informacyjnym wyświetlana jest zagadka z zakresu kombinatoryki dotycząca liczby możliwych ustawień kodu PIN. Uczestnicy muszą przypomnieć sobie zasady kombinatoryki i obliczyć liczbę możliwych kombinacji zgodnie z treścią wylosowanej zagadki. Po obliczeniu właściwej liczby gracze wprowadzają wynik za pomocą klawiatury sejfu. Widok sejfu z panelem numerycznym oraz treścią zagadki pokazano na rys.~\ref{fig:Sejf23}.
\begin{figure}[H]
    \centering
    \includegraphics[width=0.6\textwidth]{images/sejf23.jpg}
    \caption{Sejf z zagadką kombinatoryczną}
    \label{fig:Sejf23}
\end{figure}


\section{Liczby rzeczywiste i działania na zbiorach liczbowych}

W siódmym zadaniu gracze pracują z elektroniczną skrzynią wyposażoną w klawiaturę numeryczną oraz elementy związane ze zbiorami liczbowymi. Zadanie składa się z dwóch etapów i ma na celu sprawdzenie znajomości podstawowych własności zbiorów liczb rzeczywistych oraz relacji między nimi.
Pierwszy etap polega na analizie liczb wyświetlonych w pomieszczeniu oraz określeniu, ile z nich należy do poszczególnych zbiorów liczbowych, takich jak liczby naturalne, całkowite, wymierne, niewymierne oraz zbiór liczb rzeczywistych. Na podstawie uzyskanych wyników gracze ustalają kod otwierający elektroniczny zamek skrzyni, zgodnie z kolejnością wskazaną na panelu informacyjnym umieszczonym na ścianie.
Po poprawnym wprowadzeniu kodu uruchamiany jest drugi etap zadania. Gracze otrzymują zestaw ruchomych bloczków z symbolami relacji zbiorów, takich jak zawieranie się zbiorów, część wspólna oraz zbiór pusty. Ich zadaniem jest uzupełnienie wyświetlonych relacji pomiędzy zbiorami liczbowymi poprzez umieszczenie odpowiednich symboli we właściwych miejscach. Drugi etap zadania przedstawiono na rys.~\ref{fig:Liczby23}.
\begin{figure}[H]
    \centering
    \includegraphics[width=0.6\textwidth]{images/liczby23.jpg}
    \caption{Zadanie dotyczące zbiorów liczb rzeczywistych}
    \label{fig:Liczby23}
\end{figure}
Poprawne wykonanie obu etapów zadania skutkuje zaliczeniem zagadki oraz odblokowaniem przejścia do kolejnego etapu gry.
\section{Znaki funkcji trygonometrycznych}

W ósmym zadaniu gracze stają przed układem jednostkowym reprezentującym jedną z funkcji trygonometrycznych, takich jak sinus, cosinus, tangens lub cotangens. Każdy okrąg podzielony jest na cztery ćwiartki układu współrzędnych oznaczone numerami I–IV. Zadaniem uczestników jest uzupełnienie pustych pól w każdej ćwiartce odpowiednim znakiem „+” lub „−” za pomocą interaktywnych elementów, wskazujących, czy dana funkcja przyjmuje w tej ćwiartce wartości dodatnie czy ujemne. Poprawne przypisanie znaków wymaga znajomości własności funkcji trygonometrycznych w poszczególnych ćwiartkach. Układ jednostkowy z znakami funkcji trygonometrycznej pokazano na rys.~\ref{fig:Sin23}.
\begin{figure}[H]
    \centering
    \includegraphics[width=0.7\textwidth]{images/sin23.jpg}
    \caption{ Układ jednostkowy w zadaniu trygonometrycznym }
    \label{fig:Sin23}
\end{figure}


\section{Ciągi liczbowe}

W dziewiątym zadaniu gracze przenoszą się na scenę stylizowaną na grę „Duck Hunt”, gdzie głównym wyzwaniem jest rozpoznawanie kolejnych wyrazów ciągów liczbowych. Na ekranie wyświetlana jest formuła ciągu arytmetycznego, a z góry lecą kaczki z różnymi wartościami liczbowymi. Zadaniem uczestników jest strzelanie laserem do tych kaczek, które zawierają poprawny wyraz podanego ciągu. Jeśli kaczka nie pasuje do ciągu, gracze powinni ją pozostawić. Scenę stylizowaną na grę „Duck Hunt” z wyrazami ciągu pokazano na rys.~\ref{fig:Kaczki23}.
\begin{figure}[H]
    \centering
    \includegraphics[width=0.57\textwidth]{images/kaczki23.jpg}
    \caption{ Rozpoznawanie wyrazów ciągu w dynamicznej rozgrywce }
    \label{fig:Kaczki23}
\end{figure}

 Poprawne wskazanie wyrazu ciągu skutkuje zaliczeniem punktu, natomiast błędne działanie prowadzi do utraty jednego z dostępnych żyć. Tempo gry rośnie wraz z postępem, zwiększając wymagania dotyczące refleksu i precyzji. Gra kończy się w momencie utraty wszystkich czterech żyć. Po zakończeniu zadania gracze uzyskują swój wynik i przechodzą do kolejnego etapu gry.

\section{Prawdopodobieństwo}

W tym zadaniu gracze stają przed wyzwaniem probabilistycznym, realizowanym w formie interaktywnej sceny z postacią Morfeusza, dwoma pojemnikami oraz zestawem 100 tabletek – po 50 czerwonych i niebieskich. Celem jest rozdzielenie tabletek między pojemniki w taki sposób, aby maksymalizować szansę wybrania czerwonej tabletki przez Morfeusza. Gracze mogą dowolnie przesuwać tabletki, testując różne układy i sprawdzając uzyskane prawdopodobieństwo za pomocą przycisku weryfikacji. Zadanie posiada rozwiązanie optymalne, które pozwala osiągnąć maksymalne możliwe prawdopodobieństwo sukcesu, wynoszące około 74{,}75\%. Zadanie wymaga od uczestników zrozumienia i wykorzystania zasad prawdopodobieństwa do podejmowania decyzji. Końcowy stan zadania po poprawnym rozwiązaniu przedstawiono na rysunku \ref{fig:Matrix23}.
\begin{figure}[H]
    \centering
    \includegraphics[width=0.57\textwidth]{images/matrix23.jpg}
    \caption{ Zadanie z zakresu rachunku prawdopodobieństwa }
    \label{fig:Matrix23}
\end{figure}


\section{Optymalizacja i rachunek różniczkowy}
W jedenastym zadaniu gracze mierzą się z zagadką z zakresu rachunku różniczkowego, polegającą na łączeniu funkcji z ich pochodnymi. W pomieszczeniu rozmieszczone są interaktywne tabliczki, które początkowo nie ujawniają swojej zawartości. Po ich aktywacji gracze odkrywają zapisane na nich wzory matematyczne. Widok pokoju przedstawiono na rys.~\ref{fig:Pochodna23}.
\begin{figure}[H]
    \centering
    \includegraphics[width=0.8\textwidth]{images/pochodna23.jpg}
    \caption{ Łączenie funkcji z odpowiadającymi im pochodnymi}
    \label{fig:Pochodna23}
\end{figure}
Zadaniem uczestników jest odnalezienie par składających się z funkcji oraz odpowiadającej jej pochodnej. W danym momencie możliwe jest odkrycie maksymalnie dwóch tabliczek. Jeżeli wybrane wzory tworzą poprawną parę, tabliczki znikają z planszy, a licznik postępu zostaje zwiększony. W przypadku błędnego dopasowania tabliczki wracają do stanu początkowego.
Zagadka zostaje uznana za rozwiązaną po poprawnym połączeniu wszystkich dostępnych par. Zadanie wymaga od graczy znajomości podstawowych reguł obliczania pochodnych oraz umiejętności kojarzenia zależności pomiędzy funkcją a jej pochodną.

\section{Układy równań}
W dwunastym zadaniu gracze rozwiązują zagadkę opartą na układach równań, polegającą na dobraniu odpowiedniej liczby kulek o różnych wartościach energetycznych. W pomieszczeniu znajdują się trzy kolorowe pojemniki: czerwony, zielony oraz niebieski, z których każdy przyjmuje wyłącznie kulki odpowiadającego mu koloru. Zadanie z kulkami energetycznymi oraz żarówką przedstawiono na rys.~\ref{fig:Kulki23}.
\begin{figure}[H]
    \centering
    \includegraphics[width=0.6\textwidth]{images/kulki23.jpg}
    \caption{ Rozwiązywanie układów równań przy użyciu kulek}
    \label{fig:Kulki23}
\end{figure}
Zadaniem uczestników jest umieszczenie w pojemnikach takiej liczby kulek, aby łączna wartość generowanej energii była dokładnie równa wartości wyświetlanej nad żarówką. Przy doborze kulek należy uwzględnić zależności energetyczne pomiędzy poszczególnymi kolorami, co wymaga logicznego myślenia oraz poprawnego modelowania zależności w postaci układu równań.
Poprawne dobranie liczby kulek powoduje zapalenie się żarówki nad stołem, sygnalizując rozwiązanie zagadki. W przypadku niepoprawnego ustawienia gracze mogą dowolnie modyfikować rozmieszczenie kulek i podejmować kolejne próby aż do uzyskania właściwego wyniku.

\section{Stereometria}
W trzynastym zadaniu gracze obserwują wyświetlaną w półmroku bryłę przestrzenną, prezentowaną w formie trójwymiarowego modelu. Wśród pojawiających się obiektów znajdują się m.in. kula, stożek, walec, sześcian oraz ostrosłup prawidłowy czworokątny. Przykład zadania pokazano na rys.~\ref{fig:Bryla23}.
\begin{figure}[H]
    \centering
    \includegraphics[width=0.65\textwidth]{images/bryla23.jpg}
    \caption{ Zadanie stereometryczne – rozpoznawanie brył}
    \label{fig:Bryla23}
\end{figure}

Zadaniem uczestników jest rozpoznanie rodzaju bryły oraz wskazanie poprawnych wzorów na obliczenie jej pola powierzchni i objętości, wybierając jedną z dostępnych odpowiedzi. Pytania zadawane są sekwencyjnie, a błędna odpowiedź powoduje zmianę wyświetlanej bryły.
Zadanie zostaje zaliczone po poprawnym rozwiązaniu zestawu pytań dotyczących kilku różnych brył geometrycznych.
