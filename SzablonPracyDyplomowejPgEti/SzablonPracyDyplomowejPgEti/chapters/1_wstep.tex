\chapter{Wstęp i cel pracy}
\label{chap:introduction}

Współczesny rozwój technologii informatycznych i multimedialnych umożliwia tworzenie nowoczesnych, interaktywnych narzędzi wspomagających proces nauczania, które pozwalają uczniom nie tylko na bierne przyswajanie wiedzy, ale również na aktywne uczestnictwo w zajęciach oraz samodzielne rozwiązywanie problemów w angażującym środowisku. W szczególności dynamicznie rozwijające się technologie wirtualnej i rozszerzonej rzeczywistości znajdują coraz szersze zastosowanie nie tylko w przemyśle i rozrywce, lecz również w edukacji na różnych poziomach nauczania. Zastosowanie immersyjnych środowisk w nauczaniu matematyki może znacząco wpłynąć na wzrost zaangażowania uczniów oraz efektywność przyswajania materiału dydaktycznego.

Matematyka, jako jedna z podstawowych dziedzin nauki, często bywa postrzegana przez uczniów szkół średnich jako przedmiot trudny i mało atrakcyjny. Jednym ze sposobów przełamywania tego stereotypu jest wprowadzenie do procesu dydaktycznego elementów gier edukacyjnych, które poprzez zabawę i rywalizację angażują uczniów do aktywnego rozwiązywania problemów. W ostatnich latach coraz większym zainteresowaniem cieszą się tzw. escape roomy — gry logiczne, w których uczestnicy muszą w określonym czasie rozwiązać serię zagadek i łamigłówek, aby „wydostać się” z wirtualnego lub rzeczywistego pomieszczenia. Włączenie tego typu rozwiązań do nauczania matematyki stwarza szansę na połączenie nauki z zabawą oraz zastosowanie wiedzy teoretycznej w praktycznych sytuacjach problemowych.

Przedmiotem niniejszej pracy jest opracowanie i wykonanie aplikacji stanowiącej wirtualny pokój zagadek matematycznych (ang. escape room), przeznaczonej do działania w systemie jaskiń rzeczywistości wirtualnej dostępnych w Laboratorium Zanurzonej Wizualizacji Przestrzennej (LZWP). Laboratorium to wyposażone jest w nowoczesne rozwiązania umożliwiające projekcję obrazów w technologii VR na ścianach specjalnie przystosowanych pomieszczeń, dając użytkownikowi wrażenie całkowitego zanurzenia w trójwymiarowym środowisku. Aplikacja powinna umożliwiać uczestnikom interakcję z otoczeniem za pomocą kontrolerów ruchu wykorzystywanych w LZWP.

Celem głównym projektu jest stworzenie interaktywnej aplikacji edukacyjnej zawierającej 13 zagadek matematycznych, odpowiadających poszczególnym działom matematyki nauczanym w szkołach średnich. Działy te obejmują: liczby rzeczywiste, wyrażenia algebraiczne, równania i nierówności, układy równań, funkcje, ciągi, trygonometrię, planimetrię, geometrię analityczną na płaszczyźnie kartezjańskiej, stereometrię, kombinatorykę, rachunek prawdopodobieństwa i statystykę oraz optymalizację i rachunek różniczkowy.

W ramach realizacji projektu przewidziano następujące etapy prac:
\begin{quotation}
    1. Zapoznanie się z funkcjonowaniem systemu jaskiń rzeczywistości wirtualnej dostępnych w LZWP oraz metodami programowania aplikacji dla tego typu środowisk.

    2. Opracowanie koncepcji i projektów zagadek matematycznych reprezentujących wymienione działy matematyki, z uwzględnieniem ich formy, poziomu trudności oraz możliwych metod interakcji użytkownika z aplikacją.

    3. Konsultacja opracowanych projektów zagadek z dydaktykami matematyki w celu dostosowania ich treści oraz sposobu prezentacji do wymagań programowych i poziomu uczniów szkół średnich.

    4. Implementacja aplikacji integrującej wszystkie zaprojektowane i zatwierdzone zagadki w ramach jednego spójnego środowiska wirtualnego escape roomu.

    5. Przeprowadzenie testów funkcjonalnych, wydajnościowych oraz badań pilotażowych z udziałem grupy uczniów i dydaktyków w celu weryfikacji poprawności działania aplikacji oraz jej efektywności dydaktycznej.
\end{quotation}
Ostatecznym rezultatem pracy będzie kompletna, działająca aplikacja edukacyjna przeznaczona do uruchomienia w jaskiniach rzeczywistości wirtualnej, umożliwiająca użytkownikom rozwijanie kompetencji matematycznych w angażującej i nowoczesnej formie. Wnioski płynące z przeprowadzonych badań pilotażowych pozwolą na ocenę przydatności tego typu rozwiązań w praktyce dydaktycznej oraz wskazanie możliwości ich dalszego rozwoju i wykorzystania w nauczaniu przedmiotów ścisłych.