\chapter{Podsumowanie (Andrii Demyshyn)}
\label{chap:summary}

\section{Ocena realizacji celów}
Głównym celem  pracy dyplomowej było opracowanie aplikacji edukacyjnej w formacie matematycznego escape roomu, funkcjonującej w środowisku wirtualnej rzeczywistości. Głównym zadaniem projektu było stworzenie interaktywnej gry VR, łączącej elementy nauki matematyki z mechaniką gry, a także jej dostosowanie do warunków pracy w systemie CAVE.

W ramach realizacji projektu udało się stworzyć środowisko wirtualne, w którym użytkownik kolejno rozwiązuje trzynaście zagadek matematycznych obejmujących różne dziedziny matematyki. Każda zagadka została zrealizowana w formie interaktywnej i wymaga aktywnej interakcji użytkownika z obiektami wirtualnego otoczenia. Aplikacja została zaimplementowana przy użyciu silnika Unreal Engine i przystosowana do działania w środowisku Laboratorium Zanurzonej Wizualizacji Przestrzennej.

Określone cele funkcjonalne, takie jak zapewnienie immersyjnego środowiska VR, obsługa interakcji użytkownika, weryfikacja poprawności rozwiązań oraz kontrola przebiegu rozgrywki, zostały osiągnięte. System umożliwia zarówno samodzielne przechodzenie gry przez użytkownika, jak i nadzór ze strony administratora, co zwiększa stabilność działania aplikacji oraz bezpieczeństwo jej użytkowania.

Zrealizowany projekt spełnia założenia edukacyjne, umożliwiając naukę matematyki w atrakcyjnej i angażującej formie. Dodatkowo podczas realizacji pracy osiągnięto cele osobiste autora, związane z nabyciem praktycznych umiejętności w zakresie projektowania aplikacji VR, pracy z silnikiem Unreal Engine oraz integracji systemu z infrastrukturą CAVE.

Podsumowując, można stwierdzić, że wszystkie główne cele pracy dyplomowej zostały zrealizowane zgodnie z przyjętymi założeniami, a otrzymany rezultat stanowi w pełni funkcjonalne i wartościowe rozwiązanie edukacyjne.

\section{Propozycje rozwoju projektu}

Aplikacja opracowana w ramach pracy dyplomowej może być w przyszłości rozbudowana i udoskonalona zarówno pod względem funkcjonalnym, jak i dydaktycznym. Zrealizowany prototyp matematycznego escape roomu w środowisku wirtualnej rzeczywistości stanowi solidną podstawę do dalszych usprawnień i rozwoju projektu.

W pierwszej kolejności można by zwiększyć różnorodność zagadek. Na przykład obecnie na poziomie „Układy równań” przy każdym uruchomieniu zmienia się tylko moc żarówki, ale układ trzech równań pozostaje ten sam. Można by dodać jeszcze kilka układów równań do zadania, ale należy również wziąć pod uwagę, aby można było osiągnąć każdą wygenerowaną moc żarówki. W poziomie „Prawdopodobieństwo” obecnie zawsze tworzy się 100 tabletek i 2 pojemniki. Można sprawić, aby generowano różną liczbę tabletek, na przykład od 26 do 200, oraz generować liczbę pojemników od 2 do 4. Podczas realizacji warto również zwrócić uwagę, aby liczba tabletek i pojemników była zależna od osiągnięcia celów i interesującej interakcji. Takie zmiany zmienności można dodać również do innych poziomów.

W aplikacji można dodać wybór poziomu trudności, na przykład tak, aby na początku przed graczem pojawiły się 3 przyciski wyboru trudności: łatwy, średni, wysoki. Każda zagadka mogłaby być trudniejsza, a po wybraniu poziomu trudności gracz przechodziłby do innego typu zagadki. Jako przykład weźmy zagadkę „Wzory skróconego mnożenia”. Dla łatwego poziomu trudności pozostawimy tę, która jest. Dla średniego poziomu trudności można zrealizować, aby zamiast całych prawych części gracz zbierał części np. „2ab”, „a²” lub „ab²”, a przy dopasowywaniu takich części między sobą pojawiałby się znak plus, który po naciśnięciu zmieniałby się na minus i odwrotnie. Dla wysokiego poziomu trudności można by stworzyć grę przypominającą logiczną grę „Piętnastka”. Na planszy rozmieszczone byłyby losowo małe kafelki, z których każdy zawierałby jeden symbol matematyczny lub wyrażenie algebraiczne. Zadaniem gracza w takim zadaniu jest również poprawne dokończenie siedmiu wzorów skróconego mnożenia na planszy. Początek każdego wzoru byłby podany w pierwszej kolumnie, a gracze musieliby przesuwać kafelki tak, aby w odpowiednim wierszu powstało pełne i poprawne wyrażenie algebraiczne. Podobnie, na przykład, w poziomie „Ciągi liczbowe” można zrobić tak, że aby przejść do następnego poziomu, trzeba osiągnąć określony wynik, a im wyższy poziom trudności tematu, tym wyższy wynik trzeba osiągnąć. Tak że w „Ciągi liczbowe” można w zależności od poziomu trudności zwiększać lub zmniejszać liczbę żyć. Takie zmiany dla poziomów trudności można wprowadzić w każdej zagadce.

Można by zrealizować całą aplikację w jednym stylu questowym, co nadałoby jej bardziej klimatyczny charakter. Na przykład zadania „Nierówności” i „Układy równań” można by zrealizować w stylu Indiany Jonesa. Gracz znajdowałby się w pokoju, w którym na ścianach napisane są nierówności, a gracz chodziłby po płytkach mostu, na których napisane są liczby, które pasują lub nie pasują do przedziału, co przypominałoby scenę z filmu „Indiana Jones -- W drodze do Graala”. Zadanie „Układy równań” można zrealizować w stylu „Indiana Jones -- Poszukiwacze Zaginionej Arki”. W pokoju byłyby rozrzucone różnokolorowe kamienie o różnej wadze, byłby piedestał, na którym napisane układy równań, a po ich rozwiązaniu gracz dowiadywałby się o wadze kamieni i artefakt na piedestale z liczbą jego wagi. Zadaniem gracza byłoby zebranie do worka kamieni o wadze odpowiadającej artefaktowi i zastąpienie ich na piedestale, tak jak w scenie otwierającej film. W takiej stylistyce można by zrealizować wszystkie zadania. Można również zrealizować wszystko w stylu filmu „Matrix” lub innego filmu, serialu, książki, gry. Albo stworzyć przejście w formie detektywa w jakimś laboratorium, gdzie coś się wydarzyło, a gracz musi dowiedzieć się, co się stało, gdzie po każdym etapie gracz otrzymywałby nowe szczegóły. Przy takiej realizacji quest byłby bardzo klimatyczny, co zwiększyłoby zainteresowanie i zachwyt z przejścia.

Jednym z możliwych kierunków byłoby również rozszerzenie bazy gier. Na przykład można by dać inne interpretacje gier w zależności od poziomu trudności lub wyboru gracza. Na przykład w zadaniu „Ciągi liczbowe” zamiast gry „DuckHunt”, w której gracz strzela do kaczek z liczbami, można by zrealizować również grę „BeatSaber”, w której gracz musiałby rozbijać kostki z liczbami mieczem świetlnym.

Można by również wprowadzić system rankingowy oraz limit czasowy, w którym gracze musieliby walczyć o osiągnięcie najlepszego wyniku.

Podsumowując, zaproponowane kierunki rozwoju pokazują, że stworzona aplikacja posiada duży potencjał rozbudowy i może w przyszłości stanowić kompleksowe narzędzie wspomagające nauczanie matematyki z wykorzystaniem technologii wirtualnej rzeczywistości.

\section{Wnioski końcowe}
W ramach niniejszej pracy dyplomowej opracowano i zrealizowano projekt edukacyjny w formie matematycznego escape roomu funkcjonującego w środowisku wirtualnej rzeczywistości. Głównym celem pracy było stworzenie interaktywnej aplikacji VR, łączącej naukę matematyki z elementami gry, oraz jej adaptacja do warunków pracy w systemie CAVE. Wyznaczone cele zostały osiągnięte, a zaprojektowana aplikacja stanowi w pełni funkcjonalne rozwiązanie edukacyjne.

W trakcie realizacji projektu opracowano wirtualne środowisko, w którym użytkownik rozwiązuje trzynaście zróżnicowanych zagadek matematycznych, wymagających aktywnej interakcji z obiektami świata wirtualnego. Zastosowane mechaniki pozwalają na angażujące i intuicyjne przechodzenie gry, a brak klasycznego interfejsu użytkownika sprzyja zwiększeniu immersji w środowisku VR.

Realizacja pracy umożliwiła autorom zdobycie praktycznego doświadczenia w zakresie projektowania aplikacji w silniku Unreal Engine, pracy z technologiami wirtualnej rzeczywistości oraz integracji systemu z infrastrukturą Laboratorium Zanurzonej Wizualizacji Przestrzennej. Projekt pozwolił również na pogłębienie wiedzy z zakresu projektowania interaktywnych systemów edukacyjnych.

Podsumowując, przeprowadzona praca potwierdza, że wykorzystanie technologii wirtualnej rzeczywistości w procesie nauczania matematyki może stanowić skuteczne i atrakcyjne narzędzie dydaktyczne. Opracowana aplikacja spełnia założenia pracy dyplomowej i może stanowić podstawę do dalszego rozwoju oraz zastosowań edukacyjnych.
