\chapter{Organizacja pracy (Jan Walczak)}
\label{organizacja_pracy}

Zarządzanie zespołem i dobra organizacja pracy to dwa kluczowe aspekty podczas pracy grupowej.
Metodyczne działanie pozwala usprawnić i uporządkować pracę projektową i komunikację 
w zespole \cite{Zarzadzanie_Projektami_IT_Przewodnik_po_Metodykach}. Zespół, który implementował omawiane w tej pracy rozwiązanie, składał się 
z~trzech osób: Jana Walczaka, Konrada Czarneckiego i Andriego Demyshyna.

\section{System kontroli wersji (Jan Walczak)}

\subsection{Zastosowane rozwiązanie}
System kontroli wersji pomaga śledzić historię zmian dokonywanych przez uczestników projektu.
Ułatwia zarządzanie procesem jego tworzenia, zapewnia integralność danych oraz gwarantuje,
że dokonywane zmiany są na bieżąco zapisywane i udostępniane uczestnikom oraz użytkownikom \cite{github_about_git}. 

Do wdrożenia systemu kontroli wersji w omawianym projekcie użyto internetowej platformy GitHub.
Organizacja pracy sprowadza się do opracowania zasad dotyczących tzw. \glqq Flow\grqq.
\glqq Flow \grqq pochodzi od angielskiego słowa \glqq workflow \grqq (\textit{ang. przepływu pracy}).
Taki przepływ ma na celu opisanie i~wdrażanie zmian zachodzących w danym projekcie tak, aby ułatwić 
ich zrozumienie nie tylko deweloperom, ale także użytkownikom, którzy śledzą zmiany zachodzące w repozytorium \cite{github_about_git}.  

\subsection{Kontrola przepływu}

Przepływ zaczyna się od zdefiniowania problemu, poprzez opisanie problemu lub funkcjonalności,
których będzie dotyczyć implementowana zmiana. Na platformie GitHub odbywa się to poprzez tworzenie
\verb|Issues| (\textit{zagadnień}). Pozwalają one również na planowanie i prowadzenie dyskusji
na temat danego zagadnienia z innymi uczestnikami projektu lub użytkownikami \cite{github_about_issues}.
Dodatkową zaletą jest możliwość bezpośredniego powiązania zmian dokonanych w kodzie z unikalnym numerem,
który jest nadawany każdej takiej dyskusji przez omawianą platformę.
W omawianym projekcie zdecydowano się na wprowadzanie zasad dotyczących formułowania tytułu i treści
\verb|Issue|. Tytuł powinien być napisany w języku angielskim, tak żeby każda osoba przeglądająca
repozytorium mogła powiązać problem i dyskusję ze zmianą w aplikacji. Dodatkowo powinien być zwięzły i dotyczyć
wyłącznie jednego zagadnienia. Opis i dyskusja mogą być napisane w języku polskim, tak żeby maksymalnie usprawnić komunikację zespołu.
\verb|Issue| może znajdować się w dwóch stanach: zamkniętym -- czyli stan, w którym praca została zakończona
i otwartym -- czyli stan, w którym praca nad danym zagadnieniem nadal trwa.

\begin{figure}[htbp]
    \centering
    \includegraphics[width=0.7\textwidth]{images/git_issues.png}
    \caption{Zrzut ekranu z platformy GitHub -- tablica z przykładowymi Issues do projektu.}
    \label{git:issues}
\end{figure}

\FloatBarrier

W repozytorium znajduje się tak zwana główna linia (\textit{ang. main line of development}).
Jest to liniowa historia zmian, które zostały wprowadzone do projektu. Deweloperzy mogą wprowadzać swoje zmiany,
które są definiowane tym, w jaki sposób (w którym punkcie w historii zmian) odłączyły się od tej linii.
W ten sposób można tworzyć rozgałęzienia (\textit{ang. branches}) względem głównej linii \cite{git_branches}. 
Główna gałąź (\textit{ang. main branch}), może umożliwić deweloperom zorientowanie się czy zmiana, którą chcą wprowadzić, 
będzie mogła być połączona z aktualnie istniejącą wersją aplikacji. Ważnym elementem przepływu jest więc zdefiniowanie 
tego w jaki sposób każda zmiana będzie dołączana do głównej gałęzi. Podczas implementacji projektu zdecydowano się na
bezpośrednie powiązanie tworzonych gałęzi z \verb|Issues|, tj. nazywanie ich tymi samymi tytułami oraz wiązanie ich przez linkowanie
unikalnego identyfikatora nadawanego każdemu \verb|Issue|.

\subsection{Łączenie gałęzi}

Domyślnie gałęzie mogą być wiązane bez żadnych ograniczeń. GitHub umożliwia tworzenie zbioru zasad (\textit{ang. rulesets})
dotyczących wiązania gałęzi deweloperów z innymi gałęziami. Zasady te pozwalają na określenie, które grupy są upoważnione
do jakich czynności związanych z daną gałęzią oraz definiują jakie zasady muszą spełnić, aby powiązać swoją gałąź,
czyli wprowadzaną zmianę, do danej linii. \cite{git_rulesets}. W projekcie zastosowano następujące zasady,
dotyczące głównej linii:
\begin{itemize}
    \item \verb|Restrict deletions| -- nie pozwalaj na usuwanie powiązanych referencji.
    \item \verb|Require a pull request before merging| -- powiąż z gałęzią przez \glqq pull request\grqq.
    \item \verb|Block force pushes| -- nie pozwól na wymuszenie nadpisania zmian.
\end{itemize}

W celu spełnienia zasady \verb|Require a pull request before merging|
każda wprowadzana zmiana musi zostać przygotowana do połączenia w odpowiedni sposób. Odbywa się to przez mechanizm
\verb|Pull request|. Umożliwia on wgląd we wprowadzane modyfikacje innym uczestnikom projektu przed ich finalnym zatwierdzeniem i wdrożeniem \cite{git_about_pull_requests}.
W projekcie określono, że żeby zmiana została połączona z główną gałęzią, musi być zatwierdzona przez minimalnie 
jednego uczestnika projektu, niebędącego autorem. Dodatkowo, zmiana musi być możliwa do powiązania bez występowania
konfliktów, czyli fragmentów plików, których system kontroli wersji nie mógł samodzielnie scalić ze zmianami \cite{git_about_pull_requests}.

Zmiany wprowadzane przez deweloperów składają się z serii migawek \verb|Commit|. \verb|Commit| zawiera wiadomość,
unikalny identyfikator oraz listę plików i zmian, które są z nim powiązane. Pozwala na
wielokrotny zapis zmian podczas pracy w ramach jednej gałęzi. W projekcie zostało nałożone ograniczenie, wymuszające
na uczestnikach scalanie migawek w jedność za pomocą mechanizmu \verb|rebase| i \verb|squash| \cite{git_about_pull_requests}.

Po poprawnym utworzeniu i scaleniu \verb|Pull request| utworzona wcześniej gałąź może zostać usunięta. Jest to bezpieczne,
ponieważ gałąź zostaje dołączona do głównej linii jako \verb|Commit| i historia zachowuje swój liniowy charakter. Deweloper
powinien oznaczyć \verb|Issue| jako zamknięte. \verb|Pull request| otrzyma status \verb|Merged| (scalone) automatycznie.

\begin{figure}[htbp]
    \centering
    \includegraphics[width=0.7\textwidth]{images/git_merged.png}
    \caption{Zrzut ekranu z platformy GitHub -- tablica z przykładowym Pull Request}
    \label{git:merged}
\end{figure}

\FloatBarrier

\subsection{Przechowywanie dużych plików -- Git LFS}
Projekt był realizowany w środowisku Unreal Engine, przez co w projekcie znajdowało się dużo plików o dużym rozmiarze.
Co do zasady Git nie jest przystosowany do przechowywania dużych plików (takich jak wideo, audio, grafiki), a jedynie
małych plików binarnych takich jak tekst czy pliki z kodem.

W projekcie zastosowano mechanizm \verb|Git LFS| czyli \glqq Git Large File Storage\grqq.  Jest to system, który
umożliwia tworzenie wskaźników kierujących do danego pliku, który jest umieszczany poza repozytorium (na zewnętrznym
serwerze) \cite{git_lfs}. Repozytorium wymaga wyspecyfikowania, które pliki powinny być przechowywane nie jako pliki
binarne w repozytorium, a jako dowiązania. W~tym celu należało zastosować odpowiednie filtry wskazujące na duże
obiekty tworzone w Unreal Engine: rozszerzenia \verb|*.uasset| oraz \verb|*.umap|.
W przypadku omawianej platformy zagwarantowany jest dysk o pojemności \textit{10~GB} znajdujący
się bezpośrednio po stronie platformy.

\section{Tablica Kanban (Jan Walczak)}
Aby dobrze zarządzać projektem potrzebna jest jasna komunikacja i wyznaczanie celów już od
najwcześniejszych etapów rozwoju projektu. Praktyki definiujące wymagania i przydzielające uczestników
projektu do poszczególnych zadań znacząco zwiększają ich zaangażowanie i gwarantują, że aplikacja będzie spełniać
potrzeby zdefiniowane przez interesariuszy \cite{rational-unified-process}.  

Żeby zapewnić powyższe wymagania zdecydowano się na zastosowanie zwinnej metodyki Kanban. Jednym z jej głównych zadań jest
wizualizacja działań podejmowanych w projekcie \cite{kanban}. Metodyka jest często stosowana w formie fizycznej i występuje
wtedy jako tablica podzielona na odpowiednie kolumny. Standardowy, trzykolumnowy podział prezentuje się następująco:
\begin{itemize}
    \item ToDo -- zadania do zrobienia,
    \item Doing -- zadania, nad którymi trwa praca,
    \item Done -- zadania zakończone.
\end{itemize}
Uczestnicy umieszczają samoprzylepne karteczki z krótko opisanymi zadaniami. W opisywanym projekcie zastosowano narzędzie
internetowe, służące do wizualizacji wykonywanych zadań. Każdy uczestnik otrzymał przydział zadań według deklarowanych przez niego
preferencji. Dodano dodatkową kolumnę \glqq testing\grqq, w której były umieszczane zadania, które były w danej chwili
testowane.

\begin{figure}[htbp]
    \centering
    \includegraphics[width=0.7\textwidth]{images/kanban_nasze.png}
    \caption{Zrzut ekranu z platformy Trello -- wizualizacja tablicy Kanban}
    \label{git:merged}
\end{figure}

\FloatBarrier

Kolejnym, charakterystycznym elementem metodyki Kanban jest kontrola przepływem \cite{kanban}. Dzięki sekwencyjnemu
przemieszczaniu karteczek z zadaniami na tablicy, zgodnie z zasadą, że można przesunąć tylko jedną karteczkę na raz
(od lewej do prawej), uczestnicy mają możliwość śledzenia postępu prac. Aby kontrola przepływem działała sprawnie
konieczna jest samodyscyplina uczestników w uaktualnianiu postępów wykonywanych zadań. 

Kontrola przepływem wiąże się bezpośrednio z zasadą limitu WIP (\textit{ang. Work In Progress}) określoną dla metodyki, 
która odróżnia ją od innych, zwinnych metodyk \cite{kanban}. W przypadku opisanym w projekcie określała ona nad iloma zagadkami
na raz może pracować każdy uczestnik projektu. Zdecydowano się na ustalenie tego limitu na jedną zagadkę na raz. W takim przypadku
każdy uczestnik projektu mógł umieścić tylko jedną karteczkę, przypisaną do niego, w kolumnie \glqq Doing\grqq.