% Wczytanie szablonu
\documentclass[nostrict]{Szablon}

\usepackage[polish]{babel}

\usepackage{placeins}
\usepackage{float}
\addbibresource{bibliography/bibliography.bib}

\titlespacing*{\subsubsection}{0pt}{0pt}{0pt}

% Definicja dokumentu
\usepackage[unicode=true]{hyperref}
\newcommand\PDFtitle{Tytuł pracy}
\newcommand\PDFauthors{Imie Nazwisko}
\hypersetup{
  pdftitle={\PDFtitle},
  pdfauthor={\PDFauthors},
}
\raggedbottom
% Zmiana czcionki dla symulacji maszynopisu (verbatim)
\makeatletter
\renewcommand{\verbatim@font}{\ttfamily\small}
\makeatother

% Część właściwa pracy
\begin{document}
\chapter*{Streszczenie (Jan Walczak)}

Zaangażowanie uczniów w szkołach średnich spada, a lekcje w standardowej formule mogą wydawać
się nieatrakcyjne. Matematyka jest jednym z trudniejszych przedmiotów w szkole, którego
uczniowie uczą się na każdym etapie edukacji. Rośnie więc potrzeba dywersyfikacji sposobów
nauczania i urozmaicania zajęć, tak aby stawały się one ciekawsze.

W ramach pracy zaprojektowano i wykonano aplikację matematycznego escape roomu dla uczniów szkół średnich,
która miałaby przedstawić im naukę matematyki jako coś interesującego i wartego ich zaangażowania.
Projekt obejmuje opracowanie 13 zagadek, z których każda dotyczy innego działu matematyki i porusza zagadnienia
obowiązujące w podstawie programowej szkół średnich. Implementacja została wykonana w Laboratorium 
Zanurzonej Wizualizacji Przestrzennej z wykorzystaniem technologii VR. Głęboka immersja i interaktywność
przedstawionego rozwiązania mają pozwolić uczniom na efektywną i przyjemną naukę.

\textbf{Słowa kluczowe}: matematyczny escape room, Laboratorium Zanurzonej Wizualizacji Przestrzennej, 
CAVE, Unreal Engine, VR, szkoły średnie.


%\addcontentsline{toc}{chapter}{Streszczenie}  

\chapter*{Abstract (Jan Walczak)}

Fewer and fewer students are engaged in school, and lessons in the standard format may seem unattractive
to them. Mathematics is one of the most difficult subjects that students study at every stage of their 
education. Because of this, the demand for diversifying teaching methods is increasing so that learning
becomes more interesting.

As part of an engineering project, a mathematical escape room application for high school students was 
designed and developed to present mathematics as something fresh and worth their effort. The project consists 
of 13 different puzzles, each of which introduces a different branch of mathematics and addresses topics 
covered in the curriculum. The project was implemented in the Immersive 3D Visualization Lab using VR technology. 
The deep immersion and interactivity of the solution are intended to enable students to study effectively and 
enjoyably.

\textbf{Keywords}: mathematical escape room, Immersive 3D Visualization Lab, CAVE, Unreal Engine, VR, middle school,
high school.

%\addcontentsline{toc}{chapter}{Abstract}  

\include{chapters/0b_spis_tresci}
\chapter*{Wykaz ważniejszych oznaczeń i skrótów}
%\addcontentsline{toc}{chapter}{Wykaz ważniejszych oznaczeń i skrótów}

Escape room – interaktywna gra, w której gracze rozwiązują zagadki, by „uciec” z zamkniętego pokoju lub przejść do kolejnego etapu.

Laboratorium Zanurzonej Wizualizacji Przestrzennej – specjalne środowisko lub przestrzeń, gdzie łączy się elementy rzeczywistości wirtualnej i fizycznej, często wykorzystywane do edukacji i interaktywnych doświadczeń.

Wirtualna rzeczywistość (VR) – technologia pozwalająca na zanurzenie się w cyfrowym świecie za pomocą gogli i kontrolerów.

\chapter{Wstęp i cel pracy}
\label{chap:introduction}

Współczesny rozwój technologii informatycznych i multimedialnych umożliwia tworzenie nowoczesnych, interaktywnych narzędzi wspomagających proces nauczania, które pozwalają uczniom nie tylko na bierne przyswajanie wiedzy, ale również na aktywne uczestnictwo w zajęciach oraz samodzielne rozwiązywanie problemów w angażującym środowisku. W szczególności dynamicznie rozwijające się technologie wirtualnej i rozszerzonej rzeczywistości znajdują coraz szersze zastosowanie nie tylko w przemyśle i rozrywce, lecz również w edukacji na różnych poziomach nauczania. Zastosowanie immersyjnych środowisk w nauczaniu matematyki może znacząco wpłynąć na wzrost zaangażowania uczniów oraz efektywność przyswajania materiału dydaktycznego.

Matematyka, jako jedna z podstawowych dziedzin nauki, często bywa postrzegana przez uczniów szkół średnich jako przedmiot trudny i mało atrakcyjny. Jednym ze sposobów przełamywania tego stereotypu jest wprowadzenie do procesu dydaktycznego elementów gier edukacyjnych, które poprzez zabawę i rywalizację angażują uczniów do aktywnego rozwiązywania problemów. W ostatnich latach coraz większym zainteresowaniem cieszą się tzw. escape roomy — gry logiczne, w których uczestnicy muszą w określonym czasie rozwiązać serię zagadek i łamigłówek, aby „wydostać się” z wirtualnego lub rzeczywistego pomieszczenia. Włączenie tego typu rozwiązań do nauczania matematyki stwarza szansę na połączenie nauki z zabawą oraz zastosowanie wiedzy teoretycznej w praktycznych sytuacjach problemowych.

Przedmiotem niniejszej pracy jest opracowanie i wykonanie aplikacji stanowiącej wirtualny pokój zagadek matematycznych (ang. escape room), przeznaczonej do działania w systemie jaskiń rzeczywistości wirtualnej dostępnych w Laboratorium Zanurzonej Wizualizacji Przestrzennej (LZWP). Laboratorium to wyposażone jest w nowoczesne rozwiązania umożliwiające projekcję obrazów w technologii VR na ścianach specjalnie przystosowanych pomieszczeń, dając użytkownikowi wrażenie całkowitego zanurzenia w trójwymiarowym środowisku. Aplikacja powinna umożliwiać uczestnikom interakcję z otoczeniem za pomocą kontrolerów ruchu wykorzystywanych w LZWP.

Celem głównym projektu jest stworzenie interaktywnej aplikacji edukacyjnej zawierającej 13 zagadek matematycznych, odpowiadających poszczególnym działom matematyki nauczanym w szkołach średnich. Działy te obejmują: liczby rzeczywiste, wyrażenia algebraiczne, równania i nierówności, układy równań, funkcje, ciągi, trygonometrię, planimetrię, geometrię analityczną na płaszczyźnie kartezjańskiej, stereometrię, kombinatorykę, rachunek prawdopodobieństwa i statystykę oraz optymalizację i rachunek różniczkowy.

W ramach realizacji projektu przewidziano następujące etapy prac:
\begin{quotation}
    1. Zapoznanie się z funkcjonowaniem systemu jaskiń rzeczywistości wirtualnej dostępnych w LZWP oraz metodami programowania aplikacji dla tego typu środowisk.

    2. Opracowanie koncepcji i projektów zagadek matematycznych reprezentujących wymienione działy matematyki, z uwzględnieniem ich formy, poziomu trudności oraz możliwych metod interakcji użytkownika z aplikacją.

    3. Konsultacja opracowanych projektów zagadek z dydaktykami matematyki w celu dostosowania ich treści oraz sposobu prezentacji do wymagań programowych i poziomu uczniów szkół średnich.

    4. Implementacja aplikacji integrującej wszystkie zaprojektowane i zatwierdzone zagadki w ramach jednego spójnego środowiska wirtualnego escape roomu.

    5. Przeprowadzenie testów funkcjonalnych, wydajnościowych oraz badań pilotażowych z udziałem grupy uczniów i dydaktyków w celu weryfikacji poprawności działania aplikacji oraz jej efektywności dydaktycznej.
\end{quotation}
Ostatecznym rezultatem pracy będzie kompletna, działająca aplikacja edukacyjna przeznaczona do uruchomienia w jaskiniach rzeczywistości wirtualnej, umożliwiająca użytkownikom rozwijanie kompetencji matematycznych w angażującej i nowoczesnej formie. Wnioski płynące z przeprowadzonych badań pilotażowych pozwolą na ocenę przydatności tego typu rozwiązań w praktyce dydaktycznej oraz wskazanie możliwości ich dalszego rozwoju i wykorzystania w nauczaniu przedmiotów ścisłych.
\chapter{Wprowadzenie do dziedziny}
\label{chap:field}




\section{Rola technologii w edukacji (Konrad Czarnecki)}
Rozwój technologii cyfrowych wywarł ogromny wpływ na niemal wszystkie obszary współczesnego życia, w tym również na edukację. Tradycyjne metody nauczania, oparte w dużej mierze na wykładzie i pracy z podręcznikiem, są coraz częściej wzbogacane lub nawet zastępowane przez nowoczesne narzędzia dydaktyczne, które wspomagają i uatrakcyjniają proces kształcenia. W szczególności technologie informacyjno-komunikacyjne oraz środowiska immersyjne, takie jak rzeczywistość wirtualna, rzeczywistość rozszerzona czy rzeczywistość mieszana, stają się istotnym elementem nowoczesnej edukacji.

Współczesne pokolenia uczniów i studentów od najmłodszych lat funkcjonują w środowisku przesyconym technologią i są przyzwyczajeni do interaktywnego oraz multimedialnego przekazu treści. Tradycyjny model nauczania nie zawsze jest w stanie zaspokoić potrzeby poznawcze młodych ludzi, co może prowadzić do spadku motywacji oraz trudności w przyswajaniu wiedzy. Odpowiedzią na to wyzwanie jest integracja narzędzi technologicznych z procesem dydaktycznym, co może przyczynić się do zwiększenia efektywności nauczania, ułatwienia zrozumienia trudnych zagadnień oraz podniesienia ogólnego poziomu zaangażowania uczniów.

Zastosowanie nowoczesnych technologii w edukacji nie tylko wpływa na sposób przekazywania wiedzy, ale również umożliwia tworzenie zupełnie nowych, interaktywnych form kształcenia. Szczególnie istotne znaczenie mają tutaj rozwiązania wykorzystujące elementy gamifikacji oraz immersyjnych doświadczeń edukacyjnych, które pozwalają uczniom wchodzić w bezpośrednią interakcję z materiałem dydaktycznym. W kontekście nauczania matematyki, będącej często postrzeganą jako przedmiot trudny i abstrakcyjny, technologie te mogą odegrać kluczową rolę w budowaniu pozytywnego nastawienia do nauki oraz lepszego zrozumienia prezentowanych treści.

\subsection{Nowoczesne metody nauczania matematyki}
Matematyka, jako dziedzina wymagająca myślenia analitycznego, logicznego rozumowania oraz umiejętności abstrakcyjnego operowania symbolami, od zawsze stawiała przed uczniami szczególne wyzwania. Z tego powodu nauczanie matematyki wymaga nieustannego poszukiwania skutecznych metod dydaktycznych, które nie tylko umożliwią efektywne przekazanie wiedzy, ale również wzbudzą w uczniach zainteresowanie i motywację do nauki.

Współczesne podejścia do nauczania matematyki coraz częściej odchodzą od modelu opartego wyłącznie na wykładzie i ćwiczeniach przy tablicy, na rzecz metod aktywizujących, w których uczniowie samodzielnie odkrywają zależności matematyczne, rozwiązują problemy oraz współpracują w grupie. Szczególne miejsce wśród nowoczesnych metod zajmują rozwiązania oparte na technologiach komputerowych — aplikacje i platformy edukacyjne, programy do wizualizacji danych matematycznych, a także symulacje i środowiska interaktywne.

Zastosowanie narzędzi interaktywnych pozwala uczniom na dynamiczne eksplorowanie pojęć matematycznych, eksperymentowanie z danymi i obserwowanie skutków wprowadzanych zmian w czasie rzeczywistym. Programy takie jak GeoGebra, Desmos, czy MATLAB wspierają nauczanie geometrii, analizy matematycznej i algebry w sposób znacznie bardziej przystępny i angażujący niż tradycyjne metody.

Również technologie immersyjne, takie jak VR, zyskują coraz większe znaczenie w edukacji matematycznej. Umożliwiają one prezentację skomplikowanych struktur geometrycznych w przestrzeni trójwymiarowej, co szczególnie sprzyja nauczaniu stereometrii czy geometrii analitycznej. Dzięki temu uczniowie mogą lepiej zrozumieć zależności przestrzenne oraz intuicyjnie postrzegać abstrakcyjne pojęcia matematyczne.

\subsection{Gamifinacka w edukacji}
Gamifikacja (\textit{ang. gamification}) to zastosowanie mechanizmów znanych z gier komputerowych i planszowych w kontekście niezwiązanym bezpośrednio z grami, takim jak edukacja, zarządzanie czy marketing. W praktyce oznacza to wprowadzanie elementów takich jak punkty, poziomy trudności, nagrody, rankingi czy wyzwania do tradycyjnych zadań edukacyjnych, co ma na celu zwiększenie motywacji, zaangażowania i satysfakcji uczestników procesu nauczania.

W edukacji gamifikacja znajduje szerokie zastosowanie, zwłaszcza w pracy z uczniami szkół podstawowych i średnich. Wprowadzenie grywalizacji do lekcji pozwala uczniom uczestniczyć w nauce w sposób bardziej aktywny i emocjonalnie zaangażowany. Zamiast biernego słuchania wykładu czy rozwiązywania zadań z podręcznika, uczniowie mogą wcielać się w bohaterów gier, zdobywać osiągnięcia i rywalizować z rówieśnikami w przyjazny sposób.

W kontekście nauczania matematyki, gamifikacja może znacząco ułatwić przyswajanie skomplikowanych treści. Zagadki logiczne, łamigłówki, quizy punktowane czy interaktywne escape roomy są przykładami narzędzi, które pozwalają uczniom na wykorzystanie wiedzy matematycznej w kontekście gry. Dzięki temu uczniowie nie tylko uczą się rozwiązywać konkretne typy zadań, ale także rozwijają umiejętności analitycznego myślenia, pracy zespołowej oraz podejmowania decyzji.

W szczególności w środowiskach immersyjnych, takich jak wirtualna rzeczywistość, gamifikacja osiąga nowy wymiar. Połączenie mechanizmów gry z wciągającym, interaktywnym środowiskiem pozwala użytkownikom na budowanie trwałych, pozytywnych skojarzeń z procesem nauki. Tego typu rozwiązania nie tylko zwiększają atrakcyjność zajęć, ale również mogą pozytywnie wpływać na wyniki edukacyjne i długofalowe postawy wobec nauki matematyki.









\section{Edukacyjne zastosowania escape roomów (Konrad Czarnecki)}
W ostatnich latach obserwuje się dynamiczny rozwój innowacyjnych metod dydaktycznych, które mają na celu zwiększenie zaangażowania uczniów oraz poprawę efektywności procesu nauczania. Jedną z takich metod są escape roomy, które pierwotnie funkcjonowały jako forma rozrywki, a z czasem zyskały również uznanie w środowisku edukacyjnym. Dzięki swojej interaktywnej i zespołowej formie, pokoje zagadek umożliwiają uczniom zdobywanie wiedzy oraz rozwijanie kompetencji miękkich w sposób atrakcyjny i emocjonujący.

Escape roomy w edukacji mogą przybierać różnorodne formy — od tradycyjnych wersji stacjonarnych, przez mobilne zestawy dydaktyczne, aż po aplikacje komputerowe i środowiska wirtualnej rzeczywistości. Ich celem jest nie tylko przekazywanie wiedzy merytorycznej, lecz także kształtowanie umiejętności pracy zespołowej, logicznego myślenia, zarządzania czasem oraz podejmowania decyzji pod presją. Dlatego coraz częściej wykorzystywane są one w szkołach, na uczelniach wyższych oraz podczas szkoleń i warsztatów.

W kontekście nauczania matematyki escape roomy stanowią wyjątkowo wartościowe narzędzie. Zagadki oparte na treściach matematycznych wymagają od uczestników nie tylko opanowania materiału, ale również zastosowania wiedzy w praktyce, co sprzyja utrwalaniu wiadomości oraz rozwijaniu umiejętności rozwiązywania problemów. Połączenie elementów gry z edukacją umożliwia uczniom naukę w przyjaznej atmosferze i sprzyja budowaniu pozytywnego nastawienia do przedmiotów ścisłych.



\subsection{Historia i definicja escape roomu}
Escape room, znany również jako pokój zagadek lub gra typu „ucieczka z pokoju”, to interaktywna forma rozrywki polegająca na rozwiązaniu serii łamigłówek i zadań logicznych w określonym czasie, aby wydostać się z zamkniętego pomieszczenia lub osiągnąć inny wyznaczony cel fabularny.

Pierwszy fizyczny escape room powstał w 2007 roku w Kioto w Japonii, a jego twórcą był Takao Kato, który postanowił przenieść ideę wirtualnej gry do świata rzeczywistego. Projekt szybko zyskał popularność, a w kolejnych latach podobne pokoje zaczęły powstawać w innych krajach azjatyckich, a następnie w Europie i Ameryce Północnej. Do Polski pierwsze escape roomy dotarły w 2014 roku i od tego czasu cieszą się dużym zainteresowaniem zarówno wśród młodzieży, jak i dorosłych.

Z czasem, obok komercyjnych pokoi zagadek, zaczęły pojawiać się również wersje edukacyjne, dostosowane do potrzeb szkół i uczelni. Edukacyjne escape roomy różnią się od wersji rozrywkowych tym, że ich głównym celem nie jest zabawa, lecz przekazanie wiedzy oraz rozwijanie określonych kompetencji. Zagadki w tego typu pokojach są projektowane w taki sposób, aby uczestnicy mogli przyswajać treści z wybranych dziedzin nauki podczas rozwiązywania interaktywnych zadań. Coraz częściej wykorzystywane są one także w środowiskach cyfrowych oraz w wirtualnej rzeczywistości, co dodatkowo zwiększa ich dostępność i atrakcyjność.

\subsection{Escape roomy jako metoda aktywizacji uczniów}
Jednym z największych wyzwań współczesnej edukacji jest utrzymanie wysokiego poziomu zaangażowania uczniów oraz motywowanie ich do aktywnego uczestnictwa w zajęciach. W tym kontekście escape roomy stanowią skuteczną metodę dydaktyczną, która pozwala na połączenie nauki z emocjonującą formą zabawy. Dzięki swojej interaktywnej strukturze i zespołowemu charakterowi, pokoje zagadek mobilizują uczestników do współpracy, komunikacji oraz kreatywnego rozwiązywania problemów.

Z punktu widzenia dydaktyki escape roomy wspierają rozwój kompetencji kluczowych, takich jak logiczne myślenie, umiejętność analizowania i syntetyzowania informacji, podejmowanie decyzji pod presją czasu oraz zarządzanie zadaniami w zespole. Dodatkowo umożliwiają uczniom zastosowanie zdobytej wiedzy teoretycznej w praktyce, co znacząco wpływa na trwałość zapamiętywania materiału i lepsze zrozumienie omawianych treści.

W szczególności istotne znaczenie mają escape roomy realizowane w środowisku rzeczywistości wirtualnej. Dzięki immersyjnemu charakterowi takich aplikacji uczestnicy mogą zanurzyć się w wirtualnym świecie, w którym zagadki matematyczne są częścią spójnej fabuły. Tego typu doświadczenie nie tylko zwiększa atrakcyjność nauki, ale również wzmacnia motywację wewnętrzną uczniów, którzy postrzegają naukę jako przygodę i wyzwanie, a nie obowiązek.





\section{Wirtualna rzeczywistość (Konrad Czarnecki)}
Współczesna edukacja coraz śmielej sięga po nowoczesne technologie, które pozwalają nie tylko wzbogacić proces nauczania, ale również znacząco zwiększyć zaangażowanie uczniów. Jednym z najbardziej dynamicznie rozwijających się obszarów w tym zakresie są technologie immersyjne, do których zalicza się wirtualną rzeczywistość, rzeczywistość rozszerzoną oraz rzeczywistość mieszaną. Ich wspólną cechą jest zdolność do tworzenia interaktywnych środowisk, w których użytkownik ma wrażenie pełnego zanurzenia w generowanym komputerowo świecie.

W kontekście edukacyjnym szczególnie istotne jest połączenie immersji z interaktywnością, czyli możliwością bezpośredniego wpływania na elementy wirtualnego środowiska. Interaktywne aplikacje VR sprzyjają aktywnej nauce poprzez angażowanie użytkownika w proces rozwiązywania problemów, podejmowania decyzji i wykonywania zadań w czasie rzeczywistym. Taka forma edukacji nie tylko zwiększa efektywność przyswajania wiedzy, ale również pozytywnie wpływa na motywację i postawy uczniów wobec nauki.


\subsection{Wirtualna rzeczywistość i jej zastosowanie w nauce}
Wirtualna rzeczywistość to technologia umożliwiająca tworzenie komputerowo generowanych środowisk trójwymiarowych, z którymi użytkownik może wchodzić w interakcję w czasie rzeczywistym. Dzięki zastosowaniu gogli VR oraz kontrolerów ruchu możliwe jest odwzorowanie ruchów użytkownika w przestrzeni wirtualnej, co pozwala na pełne zanurzenie w symulowanym środowisku. Technologie VR znajdują zastosowanie nie tylko w rozrywce, ale również w przemyśle, medycynie, wojsku oraz, coraz częściej, w edukacji.

W środowisku edukacyjnym wirtualna rzeczywistość oferuje szerokie możliwości w zakresie tworzenia interaktywnych laboratoriów, symulacji zjawisk fizycznych, rekonstrukcji historycznych czy wirtualnych wycieczek. Uczniowie mogą dzięki niej eksplorować trudno dostępne miejsca, takie jak wnętrze ludzkiego organizmu, przestrzeń kosmiczną, odległe zakątki świata lub historyczne budowle, co znacząco wzbogaca tradycyjny proces dydaktyczny.

W przypadku dydaktyki matematyki VR pozwala na wizualizację skomplikowanych zagadnień geometrycznych, przestrzennych oraz analitycznych w atrakcyjnej i przystępnej formie. Uczniowie mogą w wirtualnym środowisku manipulować bryłami, obserwować zmiany funkcji w czasie rzeczywistym czy uczestniczyć w interaktywnych grach logicznych i escape roomach matematycznych. Dzięki temu abstrakcyjne treści stają się bardziej zrozumiałe i łatwiejsze do przyswojenia, co przekłada się na lepsze wyniki w nauce oraz pozytywnie wpływa na rozwój umiejętności analitycznego myślenia.


\subsection{Wpływ interaktywności na zaangażowanie gracza}

Interaktywność stanowi jeden z kluczowych elementów nowoczesnych metod dydaktycznych, w tym również aplikacji wykorzystujących technologię wirtualnej rzeczywistości. Oznacza ona możliwość aktywnego uczestniczenia ucznia w procesie dydaktycznym poprzez bezpośrednie oddziaływanie na środowisko edukacyjne, podejmowanie decyzji oraz realizację zadań w czasie rzeczywistym. W przeciwieństwie do pasywnego odbioru treści w tradycyjnych formach nauczania, interaktywne środowiska angażują uczniów na wielu płaszczyznach — poznawczej, emocjonalnej i motorycznej.

Z perspektywy pedagogicznej interaktywność sprzyja rozwijaniu kompetencji poznawczych, takich jak krytyczne myślenie, umiejętność analizy danych, wyciągania wniosków czy rozwiązywania problemów. Uczniowie uczestniczący w interaktywnych lekcjach częściej angażują się w zadania, są bardziej zmotywowani do podejmowania wyzwań i wykazują większą samodzielność w poszukiwaniu rozwiązań.

W przypadku zastosowań wirtualnej rzeczywistości, interaktywność przybiera różnorodne formy — od prostych gestów i ruchów wykonywanych za pomocą kontrolerów, przez manipulowanie obiektami w przestrzeni wirtualnej, aż po rozwiązywanie zagadek i wykonywanie eksperymentów. Szczególnie efektywne okazują się aplikacje edukacyjne, które łączą elementy gry z nauką, umożliwiając użytkownikom rywalizację, zdobywanie punktów, odblokowywanie kolejnych poziomów czy rozwiązywanie zagadek fabularnych.

Środowiska edukacyjne o wysokim stopniu interaktywności znacząco zwiększają motywację wewnętrzną oraz podnoszą poziom satysfakcji z nauki. Uczniowie mają również większą łatwość w przyswajaniu wiedzy oraz chętniej uczestniczą w zajęciach, co pozytywnie wpływa na ogólne efekty dydaktyczne.

W kontekście projektowanej aplikacji typu escape room dla jaskini rzeczywistości wirtualnej, wysoki poziom interaktywności będzie kluczowym elementem wpływającym na atrakcyjność i skuteczność dydaktyczną opracowanego rozwiązania. Możliwość bezpośredniego wpływania na otoczenie, rozwiązywania zagadek matematycznych w przestrzeni wirtualnej oraz współpracy z innymi uczestnikami w czasie rzeczywistym stworzy warunki sprzyjające aktywnej, angażującej nauce i rozwijaniu kompetencji.
\chapter{Analiza istniejących rozwiązań (Jan Walczak)}
\label{chap:algs}

\section{Wirtualny pokój zagadek z zakresu matematyki}

\subsection{Projekt i implementacja}
Wirtualny pokój zagadek z zakresu matematyki został opracowany na potrzeby publikacji
w~\glqq Zeszytach Naukowych Wydziału Elektroniki i Automatyki Politechniki Gdańskiej\grqq~\cite{lebiedz1}.

Artykuł opisuje innowacyjny projekt edukacyjny, polegający na stworzeniu matematycznego pokoju zagadek
(escape room) na poziomie studiów inżynierskich. Jak możemy przeczytać we wskazanym artykule,
celem pracy było stworzenie aplikacji na platformie Unity, która miałaby działać w Laboratorium
Zanurzonej Wizualizacji Przestrzennej i uatrakcyjnić studentom naukę matematyki \cite{lebiedz1}.

Zagadka została podzielona na 4 pokoje, z czego pierwszy z nich był pokojem wprowadzającym,
prowadził on do 3 kolejnych pokoi zawierających łącznie 13 zagadek. Pokoje były podzielone nie tylko
ze względu na tematykę zagadnień, ale także ze względu na wystrój \cite{lebiedz1}. Podział ten wyglądał następująco:

\begin{enumerate}
    \item Pokój w stylu nowoczesnym: zawierał zadania z zakresu m.in. wyznaczników macierzy (metoda Sarrusa), równań płaszczyzny oraz wykresów funkcji.
    \item Pokój warsztatowy: skupiał się na ciągach liczbowych (np. ciągu Fibonacciego), liczbach zespolonych, schemacie Hornera oraz systemie binarnym.
    \item Pokój w stylu egipskim: oferował zadania dotyczące działań na liczbach zespolonych oraz układów równań liniowych rozwiązywanych metodą Gaussa-Jordana.
\end{enumerate}

\subsection{Przeprowadzone badania}

Przy użyciu wcześniej omawianego rozwiązania został przeprowadzony eksperyment obejmujący
grupę 54 studentów. Badanie wykazało, że nauka poprzez rozwiązywanie zagadek w matematycznym
escape roomie przynosi wymierne korzyści \cite{lebiedz1}.

Badanie zostało rozszerzone i opisane w artykule \glqq Educational values of a virtual escape room in mathematics\grqq~\cite{lebiedz2}. Badanie polegało na podzieleniu studentów na 2 równe grupy, na których zostały zastosowane dwa
oddzielne podejścia stosowane w różnej kolejności. Pierwsze podejście polegało na uczestnictwie grupy w tradycyjnej lekcji matematyki,
a~drugie na przejściu matematycznego escape roomu w małych zespołach. Grupy brały udział w sesjach opartych o oba podejścia,
z~czego jedna najpierw uczestniczyła w tradycyjnej lekcji, a druga najpierw uczestniczyła w~sesji w escape roomie.

Wyniki zaprezentowane w artykule wskazują, że studenci uczestniczący w sesjach odbywających się w~wirtualnym escape
roomie wykazują się większym skupieniem i lepszym samopoczuciem. Największy wzrost wiedzy
odnotowano w grupie, która najpierw uczestniczyła w tradycyjnej lekcji matematyki, a po kilku dniach
wzięła udział w zajęciach przeprowadzanych przy użyciu omawianej aplikacji \cite{lebiedz2}.

\section{Wnioski z analizy dostępnych rozwiązań}

Rozwiązanie omawiane w poprzednim podrozdziale pokazuje, że głęboka imersja, którą oferuje
Laboratorium Zanurzonej Wizualizacji Przestrzennej, poprawia samopoczucie oraz nastrój studentów \cite{lebiedz2}.
Zgodnie z rozdziałem numer \ref{chap:introduction}. celem tej pracy jest stworzenie narzędzia, które będzie
wspierać proces nauczania, a nie go zastępować. Ma zwiększyć zainteresowanie matematyką wśród uczniów szkół średnich
poprzez ich samodzielne uczestnictwo w~angażującym środowisku.
\chapter{Projekt systemu}
\label{chap:project}



\section{Wymagania funkcjonalne i niefunkcjonalne (Konrad Czarnecki)}
Przed przystąpieniem do implementacji aplikacji edukacyjnej w formie escape roomu w środowisku rzeczywistości wirtualnej konieczne było precyzyjne określenie wymagań funkcjonalnych i niefunkcjonalnych, jakie powinna spełniać projektowana aplikacja. Zdefiniowanie wymagań pozwala na kontrolowanie procesu realizacji projektu, zapewnia jego zgodność z oczekiwaniami użytkowników końcowych oraz umożliwia przeprowadzenie testów walidacyjnych w końcowej fazie prac.

\subsection{Wymagania funkcjonalne}

Wymagania funkcjonalne określają, jakie działania użytkownik może wykonać w aplikacji oraz jakie funkcje powinna ona realizować. W przypadku projektowanej aplikacji escape room VR wymagania te obejmują:
\begin{itemize}[left=1.5em, label=\textbullet, topsep=0pt, itemsep=0pt]
    \item Wyświetlanie wirtualnego środowiska escape roomu – Aplikacja powinna umożliwiać użytkownikowi poruszanie się po wirtualnym pokoju zagadek, obejmującym zestaw pomieszczeń związanych z różnymi działami matematyki.

    \item Realizacja 13 zagadek matematycznych – aplikacja musi zawierać 13 interaktywnych zagadek, po jednej dla każdego z wybranych działów matematyki: liczby rzeczywiste, wyrażenia algebraiczne, równania i nierówności, układy równań, funkcje, ciągi, trygonometria, planimetria, geometria analityczna, stereometria, kombinatoryka, rachunek prawdopodobieństwa i statystyka oraz optymalizacja i rachunek różniczkowy.

    \item Interaktywność zagadek – każda zagadka powinna wymagać od użytkownika aktywnej interakcji z otoczeniem wirtualnym, np. manipulowania obiektami, wpisywania odpowiedzi czy przestawiania elementów.

    \item System weryfikacji poprawności odpowiedzi – aplikacja powinna sprawdzać poprawność rozwiązań podawanych przez użytkownika oraz wyświetlać komunikaty informujące o sukcesie lub błędzie.

    \item Śledzenie postępu użytkownika – system powinien zapamiętywać, które zagadki zostały już rozwiązane, i odblokowywać dostęp do kolejnych pomieszczeń w ustalonej kolejności.

    \item Zakończenie rozgrywki – po rozwiązaniu wszystkich 13 zagadek aplikacja powinna wyświetlić ekran podsumowujący z informacją o ukończeniu escape roomu.

    \item Obsługa urządzeń VR i interfejsów w Laboratorium – aplikacja musi być kompatybilna z systemami śledzenia ruchu i projekcji wykorzystywanymi w Laboratorium Zanurzonej Wizualizacji Przestrzennej, w tym z systemami typu CAVE oraz kontrolerami ruchu.
\end{itemize}



\subsection{Wymagania niefunkcjonalne}

Wymagania niefunkcjonalne opisują cechy jakościowe systemu oraz warunki, jakie powinien spełniać, aby zapewnić poprawne działanie i pozytywne doświadczenia użytkowników. W kontekście projektowanej aplikacji escape room VR wymagania niefunkcjonalne obejmują:
\begin{itemize}[left=1.5em, label=\textbullet, topsep=0pt, itemsep=0pt]
    \item Kompatybilność z infrastrukturą LZWP – aplikacja musi działać poprawnie w warunkach technicznych Laboratorium Zanurzonej Wizualizacji Przestrzennej, współpracując z systemem projekcyjnym CAVE oraz wykorzystywanymi kontrolerami ruchu.

    \item Wydajność – aplikacja powinna działać płynnie, z minimalnym opóźnieniem renderowania obrazu i reakcji na ruchy użytkownika, zapewniając co najmniej 60 klatek na sekundę w środowisku VR.

    \item Jakość grafiki – wirtualne środowisko escape roomu powinno charakteryzować się realistyczną i spójną oprawą wizualną wspomagającą immersję.

    \item Łatwość rozbudowy i modyfikacji – struktura projektu powinna umożliwiać przyszłą rozbudowę aplikacji o nowe zagadki, pomieszczenia lub funkcjonalności bez konieczności przebudowy istniejącego kodu i modeli.

    \item Przenośność kodu i zasobów – wszystkie pliki źródłowe, modele oraz pliki graficzne powinny być przechowywane w repozytorium GitHub w sposób umożliwiający łatwe przenoszenie projektu pomiędzy stanowiskami roboczymi oraz serwerami laboratorium.

    \item Dokumentacja – projekt musi być opatrzony szczegółową dokumentacją techniczną i użytkową, opisującą strukturę aplikacji, sposób instalacji, obsługi oraz instrukcje dotyczące przyszłej rozbudowy.
\end{itemize}

\section{Scenariusz gry (Konrad Czarnecki)}

Po uruchomieniu aplikacji gracz przenosi się do pierwszego z 13 wirtualnych pomieszczeń. Każde pomieszczenie jest
utrzymane w odmiennym stylu wizualnym i zawiera jedną interaktywną zagadkę matematyczną z przypisanego działu.
Zagadki muszą być rozwiązywane w określonej kolejności, a przejście do kolejnego pokoju jest możliwe dopiero po
poprawnym rozwiązaniu bieżącej zagadki. Pomijanie zagadek lub powrót do poprzednich pokoi nie są możliwe z poziomu
użytkownika.

W każdym pokoju gracz ma możliwość poruszania się po przestrzeni VR, wchodzenia w interakcje z elementami zagadki
oraz przechodzenia do kolejnego pomieszczenia po poprawnym rozwiązaniu zagadki. 

Działanie każdej zagadki może wymagać różnych form interakcji — takich jak przeciąganie obiektów, wpisywanie wyników
na wirtualnej klawiaturze, wskazywanie elementów przestrzeni, czy manipulowanie wirtualnymi narzędziami (np. suwakami i
przyciskami).
Po rozwiązaniu ostatniej, trzynastej zagadki użytkownik zostaje przeniesiony do wirtualnego pokoju podsumowań, gdzie
gra informuje go o odniesionym sukcesie.

W przypadku przerwania rozgrywki przez użytkownika lub awarii systemu, aplikacja powinna zapewniać możliwość powrotu
do gry lub całkowitego zamknięcia gry bez ryzyka utraty integralności danych systemu. Rozwiązanie to
gwarantuje bezpieczeństwo użytkownika oraz stabilność aplikacji w środowisku CAVE.

Administrator systemu ma możliwość pomijania zagadek i bezpośredniego przeniesienia gracza do wybranego pokoju, co może
być wykorzystane na życzenie użytkownika w związku z jego preferencjami w zakresie zagadek i dziedzin matematyki lub w
przypadku awarii systemu. Funkcjonalność ta jest dostępna wyłącznie dla uprawnionych użytkowników i nie jest widoczna
dla zwykłych graczy, by uniemożliwić przypadkowe pominięcie zagadek.

Tak skonstruowany scenariusz rozgrywki umożliwia nie tylko wprowadzenie użytkownika w świat wirtualnej
rzeczywistości, ale także zapewnia dynamiczny, uporządkowany i logiczny przebieg interakcji. Dzięki systematycznemu
rozwiązywaniu zagadek i przemieszczaniu się między pokojami, aplikacja utrzymuje wysoki poziom zaangażowania i
motywacji użytkowników do ukończenia całej gry.







%\section{Zagadki (Konrad Czarnecki)}
%W ramach realizacji projektu opracowano trzynaście interaktywnych zagadek matematycznych, z których każda reprezentuje inny dział matematyki nauczanej na poziomie szkoły średniej. Zagadki zostały zaprojektowane tak, aby nie tylko sprawdzać wiedzę użytkowników, ale również aktywizować ich poprzez wykorzystanie mechanik charakterystycznych dla gier logicznych i escape roomów. Dzięki zastosowaniu technologii wirtualnej rzeczywistości, gracze mogą wchodzić w bezpośrednią interakcję z otoczeniem i rozwiązywać zadania w formie angażujących, przestrzennych łamigłówek. Każde z zadań posiada unikalny scenariusz i zasady działania, dostosowane do specyfiki danego działu matematyki oraz do możliwości sprzętowych Laboratorium Zanurzonej Wizualizacji Przestrzennej. Poniżej przedstawiono szczegółowy opis wszystkich trzynastu zagadek, wraz z określeniem ich celu, przebiegu oraz warunków zakończenia.
%
%
%
%\subsection{Wyrażenia algebraiczne}
%
%Pierwsze zadanie wirtualnego escape roomu polega na rozpoznawaniu oraz uzupełnianiu wzorów skróconego mnożenia. Po pojawieniu się gracza w wirtualnym pomieszczeniu widoczna jest tablica zawierająca początki wybranych wzorów algebraicznych. Każda formuła przedstawiona jest w postaci lewej strony równania wraz ze znakiem „=”, po którym pozostawiono puste miejsce przeznaczone na brakującą część wyrażenia.
%
%W przestrzeni wirtualnego pokoju rozmieszczone są interaktywne bloki zawierające możliwe zakończenia wzorów. Wśród dostępnych elementów znajdują się zarówno poprawne zakończenia odpowiadające właściwym wzorom skróconego mnożenia, jak i bloki z nieprawidłowymi zapisami, które pełnią funkcję elementów rozpraszających. Zadaniem gracza jest przeciąganie wybranych bloków oraz umieszczanie ich w odpowiednich miejscach na tablicy, tak aby utworzyć kompletne i matematycznie poprawne formuły.
%
%Po uzupełnieniu wszystkich pustych pól użytkownik ma możliwość uruchomienia przycisku sprawdzającego. Po jego użyciu system dokonuje weryfikacji poprawności skonstruowanych równań i wyświetla komunikat tekstowy informujący o wyniku zadania. Zagadkę uznaje się za rozwiązaną w momencie prawidłowego dopasowania wszystkich zakończeń do odpowiadających im początków wzorów.
%
%\subsection{Planimetria}
%
%Po ukończeniu pierwszego etapu, powierzchnia interaktywnego stołu zmienia swój wygląd, a na ekranie pojawia się plansza zawierająca dwanaście par figur geometrycznych. Wśród nich znajdują się takie figury jak kwadraty, romby, prostokąty, równoległoboki, trapezy oraz różne typy trójkątów. Zadanie ma na celu sprawdzenie znajomości zależności między figurami oraz umiejętności rozpoznawania ich cech charakterystycznych. Na planszy, pomiędzy każdą parą figur, znajduje się miejsce przeznaczone na wybór relacji. Gracze, analizując właściwości obu figur w każdej parze, muszą zdecydować, czy jedna z nich jest szczególnym przypadkiem drugiej, czy też nie. Do dyspozycji mają dwa symbole: strzałkę w prawo, oznaczającą, że pierwsza figura jest szczególnym przypadkiem drugiej, oraz przekreśloną strzałkę, oznaczającą brak takiej zależności. Po dokonaniu wszystkich wyborów system natychmiast weryfikuje poprawność ustawionych relacji. W przypadku błędnych odpowiedzi, gracze otrzymują stosowny komunikat i możliwość poprawienia wyłącznie tych relacji, które zostały oznaczone niepoprawnie. Zadanie kończy się w momencie, gdy wszystkie relacje zostaną poprawnie oznaczone. 
%
%\subsection{Równania i nierówności}
%
%W trzecim zadaniu gracze trafiają na scenę stylizowaną na film przygodowy, gdzie postać odkrywcy musi przejść przez wiszący most zbudowany z kamiennych płytek. Na każdej z płytek widnieje liczba rzeczywista, a na ścianie pojawia się układ dwóch nierówności. Zadaniem uczestników jest rozwiązanie układu i wybranie płytek z wartościami należącymi do przedziału spełniającego oba warunki. Gracze prowadzą postać, klikając po kolei właściwe płytki, a błędny wybór powoduje zapadnięcie się płytki i konieczność powrotu na początek. Po poprawnym przejściu przez most gracze odblokowują dostęp do kolejnej zagadki.
%
%\subsection{Funkcje}
%
%W czwartym zadaniu gracze pracują z dużą interaktywną tablicą, na której wyświetlany jest układ współrzędnych. Ich celem jest dostosowanie parametrów funkcji liniowej w taki sposób, aby jej wykres przechodził przez trzy wyznaczone punkty. Dzięki suwakom lub polom edycyjnym uczestnicy mogą dynamicznie zmieniać współczynnik kierunkowy oraz wyraz wolny funkcji, obserwując na bieżąco, jak zmienia się jej wykres. Zadanie polega na precyzyjnym dobraniu wartości parametrów, by wykres przeciął wszystkie wskazane punkty. 
%
%\subsection{Geometria analityczna na płaszczyźnie kartezjańskiej}
%
%W piątym zadaniu gracze ponownie pracują z interaktywną tablicą, na której tym razem pojawiają się dwa wykresy funkcji liniowych. Celem wyzwania jest odnalezienie współrzędnych punktu, w którym obie funkcje przecinają się na płaszczyźnie kartezjańskiej. Uczestnicy mogą odczytać ten punkt bezpośrednio z wykresu lub rozwiązać układ równań opisujących obie proste. Po ustaleniu wartości współrzędnych gracze wpisują je w specjalnie przygotowane pole. System weryfikuje poprawność odpowiedzi i odblokowuje przejście do kolejnego etapu gry.
%
%\subsection{Kombinatoryka}
%
%W szóstym zadaniu gracze natrafiają na metalowy sejf wyposażony w podświetlany panel numeryczny. Na wyświetlaczu obok pojawia się zagadka z zakresu kombinatoryki dotycząca liczby możliwych ustawień trzycyfrowego kodu PIN, w którym żadna z cyfr się nie powtarza. Uczestnicy muszą przypomnieć sobie zasady permutacji i wyliczyć, ile jest możliwych kombinacji, wybierając trzy różne cyfry spośród dziesięciu dostępnych. Po obliczeniu właściwej liczby gracze wpisują wynik na klawiaturze sejfu.
%
%\subsection{Liczby rzeczywiste}
%
%W siódmym zadaniu gracze pracują z dużą drewnianą skrzynią oznaczoną liczbami i symbolami zbiorów liczbowych. Pierwszym etapem jest przyporządkowanie wszystkich liczb nadrukowanych na ściankach skrzyni do odpowiednich zbiorów: liczb naturalnych, całkowitych, wymiernych i niewymiernych. Na podstawie tego podziału gracze ustalają czterocyfrowy kod otwierający mechaniczny zamek. Po poprawnym wprowadzeniu kombinacji, wewnątrz skrzyni odnajdują płaską planszę ilustrującą przecięcia zbiorów w formie diagramu Venna. Obok leżą kafelki z symbolami działań na zbiorach, które należy odpowiednio rozmieścić na planszy, uzupełniając brakujące miejsca. Po wykonaniu tego zadania, gracze odkrywają kolejny mechanizm, gdzie muszą ułożyć zależności między zbiorami liczbowymi za pomocą ruchomych symboli i strzałek, odzwierciedlając relacje zawierania się zbiorów. Poprawne wykonanie wszystkich trzech etapów aktywuje mechanizm w skrzyni i otwiera dostęp do kolejnego etapu gry.
%
%\subsection{Trygonometria}
%
%W ósmym zadaniu gracze stają przed ścianą, na której umieszczone są cztery duże okręgi, symbolizujące funkcje trygonometryczne: sinus, cosinus, tangens i cotangens. Każdy okrąg podzielony jest na cztery ćwiartki układu współrzędnych oznaczone numerami I–IV. Zadaniem uczestników jest uzupełnienie pustych pól przy każdej ćwiartce odpowiednim znakiem „+” lub „−”, wskazującym, czy dana funkcja przyjmuje w tej ćwiartce wartości dodatnie czy ujemne. Poprawne przypisanie znaków wymaga znajomości własności funkcji trygonometrycznych w poszczególnych ćwiartkach.
%
%\subsection{Ciągi}
%
%W dziewiątym zadaniu gracze przenoszą się na interaktywną platformę stylizowaną na arenę gry rytmicznej, gdzie głównym wyzwaniem jest rozpoznawanie kolejnych wyrazów ciągów liczbowych. Na ekranie wyświetlana jest formuła ciągu arytmetycznego lub geometrycznego, a z góry spadają kostki z różnymi wartościami liczbowymi. Zadaniem uczestników jest przecięcie mieczem świetlnym tylko tych kostek, które zawierają poprawne elementy podanego ciągu. Jeśli kostka nie pasuje do ciągu, gracze powinni ją pozostawić nienaruszoną. Za każdą poprawną akcję przyznawany jest punkt, natomiast błędne cięcie skutkuje odebraniem punktu. Tempo gry rośnie wraz z postępem, zwiększając wymagania dotyczące refleksu i precyzji. Gra kończy się sukcesem po zdobyciu 50 punktów i składa się z dwóch etapów: rozpoznawania ciągu arytmetycznego, a następnie geometrycznego.
%
%\subsection{Rachunek prawdopodobieństwa i statystyka}
%
%W tym zadaniu gracze stają przed wyzwaniem probabilistycznym, reprezentowanym przez interaktywny stół z postacią Morfeusza oraz dwoma pojemnikami i zestawem 100 tabletek – po 50 czerwonych i niebieskich. Celem jest rozdzielenie tabletek między pojemniki w taki sposób, aby maksymalizować szansę wybrania czerwonej tabletki przez Morfeusza. Gracze mogą dowolnie przesuwać tabletki, testując różne układy i obserwując wyświetlany na bieżąco procent prawdopodobieństwa sukcesu. Optymalnym rozwiązaniem jest umieszczenie jednej czerwonej tabletki w pierwszym pojemniku, a pozostałych 49 czerwonych oraz 50 niebieskich w drugim, co daje maksymalną szansę około 74,75\%. Zadanie wymaga od uczestników zrozumienia i wykorzystania zasad prawdopodobieństwa do podejmowania decyzji.
%
%\subsection{Optymalizacja i rachunek różniczkowy}
%
%Gracze stają przed wyzwaniem trafienia w cele na symulowanym polu bitwy. Sterują kątem nachylenia armaty, regulując go na pokrętle, aby zmodyfikować trajektorię lotu pocisku wyświetlaną na ekranie. Ich zadaniem jest precyzyjne ustawienie kąta, tak aby pocisk trafił w pojawiający się w losowej pozycji cel. Po każdym strzale pojawia się nowy cel, a gracz musi trafić łącznie 15 razy. W trakcie rozgrywki uczestnicy poznają związek między parametrami funkcji kwadratowej opisującej tor lotu, a jej maksimum, ucząc się podstaw optymalizacji i rachunku różniczkowego.
%
%\subsection{Układy równań}
%
%Gracze muszą umieścić odpowiednią liczbę jednostek energii w trzech kolorowych pojemnikach: czerwonym, zielonym i niebieskim, tak aby spełnić podane warunki dotyczące ich wydajności i sumy energii. Poprawne ustawienie 12 jednostek energii, z uwzględnieniem zależności między pojemnikami, spowoduje zapalenie się dużej żarówki nad stołem, sygnalizując sukces. System na bieżąco weryfikuje poprawność i informuje o błędach, dając możliwość kolejnych prób.
%
%\subsection{Stereometria}
%
%Gracze w półmroku obracają na obrotowym podeście pojawiającą się bryłę przestrzenną, taką jak graniastosłup, ostrosłup, walec czy stożek. Muszą rozpoznać jej nazwę oraz wskazać właściwości, takie jak liczba ścian, wierzchołków, krawędzi, rodzaj podstawy czy kąty między elementami.

\section{Model przypadków użycia(Andrii Demyshyn)}
Model przypadków użycia opisuje sposób interakcji użytkowników z projektowaną aplikacją edukacyjną typu escape room w środowisku rzeczywistości wirtualnej. W systemie wyróżniono dwóch aktorów: gracza oraz administratora systemu.

Administrator systemu jest osobą nadzorującą przebieg rozgrywki z poziomu komputera sterującego, znajdującego się poza środowiskiem wirtualnym. Do zadań administratora należy uruchamianie i kończenie aplikacji, monitorowanie przebiegu gry oraz reagowanie na sytuacje awaryjne. Aby uniknąć jakichkolwiek problemów w aplikacji, administratorowi dodano możliwość pomijania zagadek w trakcie gry. Po otrzymaniu prośby lub wystąpieniu jakiegokolwiek błędu administrator może uznać poziom za zaliczony i przenieść gracza do następnej zagadki.
To pozwala na zachowanie stabilności działania aplikacji w środowisku CAVE.

Gracz przebywa fizycznie w przestrzeni Laboratorium Zanurzonej Wizualizacji Przestrzennej i korzysta z aplikacji w środowisku VR. Podczas rozgrywki porusza się po wirtualnym pomieszczeniu, wchodzi w interakcję z obiektami poprzez ich naciskanie, przemieszczanie czy wybieranie. Gracz odczytuje informacje tekstowe i komunikaty systemowe wyświetlane w przestrzeni trójwymiarowej oraz rozwiązuje zagadki matematyczne. Także dla wygody użytkowania gracz za pomocą wybranego przycisku może obracać pokój i wszystkie  obiekty w nim na 90 stopni, co pozwala dopasować pokój dla wygodniejszego korzystania.Na każdym etapie system na bieżąco informuje gracza o poprawności wprowadzanych rozwiązań, wyświetlając odpowiednie komunikaty w przestrzeni wirtualnej. Interakcja z aplikacją odbywa się bez wykorzystania klasycznego interfejsu graficznego, a wszystkie działania realizowane są bezpośrednio w przestrzeni trójwymiarowej. Po poprawnym rozwiązaniu zagadki system automatycznie przechodzi do kolejnego etapu rozgrywki, aż do ukończenia wszystkich trzynastu zagadek.

Zaprojektowany model przypadków użycia odpowiada rzeczywistym funkcjonalnościom zaimplementowanym w aplikacji i zapewnia czytelny podział ról pomiędzy użytkownikiem końcowym a osobą nadzorującą system.
\section{Projekt architektury systemu(Andrii Demyshyn)}
Architektura projektowanej aplikacji została zaprojektowana z myślą o realizacji edukacyjnej gry typu escape room w środowisku rzeczywistości wirtualnej. System został zaimplementowany z wykorzystaniem silnika Unreal Engine i przystosowany do działania w środowisku CAVE, wykorzystywanym w Laboratorium Zanurzonej Wizualizacji Przestrzennej.
Struktura aplikacji opiera się na kilku głównych komponentach odpowiedzialnych za poszczególne aspekty działania systemu. Centralnym elementem architektury jest moduł zarządzania rozgrywką zaimplementowanej jako obiekt typu Blueprint Actor MergeBP. MergeBP kontroluje aktualny etap gry, kolejność zagadek oraz postęp gracza. Moduł ten odpowiada również za przejścia pomiędzy kolejnymi etapami po poprawnym rozwiązaniu zagadek matematycznych.
Logika zagadek matematycznych została podzielona na trzynaście niezależnych poziomów gry, każdy z których odpowiada jednej zagadce i jednemu działowi matematyki. Każda zagadka posiada własny mechanizm interakcji oraz weryfikacji poprawności rozwiązania, co umożliwia ich łatwą modyfikację lub rozbudowę bez ingerencji w pozostałą część systemu. Takie podejście upraszcza również proces testowania poszczególnych etapów gry.
Istotnym elementem architektury aplikacji jest sposób realizacji środowiska gry. Cała rozgrywka odbywa się w jednej wspólnej przestrzeni wirtualnej, która dynamicznie zmienia się w zależności od aktualnego etapu gry. Poszczególne elementy pomieszczenia, obiekty zagadek oraz elementy interaktywne są ukrywane, usuwane, aktywowane lub teleportowane w odpowiednie miejsca po ukończeniu danego poziomu. Rozwiązanie to pozwala na zachowanie spójności środowiska oraz ograniczenie konieczności ładowania nowych scen.
Moduł interakcji odpowiada za obsługę kontrolerów ruchu oraz systemów śledzenia pozycji użytkownika w przestrzeni CAVE. Umożliwia on graczowi poruszanie się po wirtualnym pomieszczeniu, manipulowanie obiektami oraz wykonywanie akcji wymaganych do rozwiązania zagadek. Równolegle system obsługuje tryb administratora, który działa z poziomu komputera sterującego i umożliwia nadzorowanie przebiegu rozgrywki oraz ingerencję w jej przebieg w sytuacjach awaryjnych.
Zaprojektowana architektura systemu zapewnia stabilne działanie aplikacji w środowisku rzeczywistości wirtualnej oraz umożliwia jej dalszą rozbudowę, na przykład poprzez dodanie nowych zagadek matematycznych lub rozszerzenie scenariusza gry.


\section{Projekt interfejsu użytkownika i środowiska gry(Andrii Demyshyn)}
Projekt interfejsu użytkownika oraz środowiska gry został opracowany z myślą o specyfice rzeczywistości wirtualnej oraz warunkach pracy w systemie CAVE. Głównym założeniem było zapewnienie wysokiego poziomu immersji oraz intuicyjnej obsługi aplikacji przy jednoczesnym ograniczeniu elementów mogących rozpraszać uwagę gracza podczas rozwiązywania zagadek matematycznych.
W projektowanej aplikacji zrezygnowano z klasycznego, dwuwymiarowego interfejsu graficznego w postaci menu czy stałych elementów HUD. Wszystkie informacje przekazywane użytkownikowi, takie jak komunikaty systemowe, treści zadań czy informacja o poprawności rozwiązania, prezentowane są bezpośrednio w przestrzeni trójwymiarowej jako elementy świata gry. Takie rozwiązanie pozwala na zachowanie spójności wizualnej oraz zwiększa poczucie obecności w środowisku wirtualnym.
Środowisko gry zostało zaprojektowane jako jedna wspólna przestrzeń wirtualna, której wygląd i zawartość zmieniają się w zależności od aktualnego etapu rozgrywki. Poszczególne zagadki matematyczne realizowane są poprzez dynamiczne pojawianie się, ukrywanie lub modyfikowanie obiektów znajdujących się w pomieszczeniu. Dzięki temu możliwe było zachowanie jednolitej przestrzeni przy jednoczesnym wyraźnym rozróżnieniu kolejnych etapów gry.
Interakcja gracza z otoczeniem realizowana jest poprzez kontrolery ruchu, umożliwiające wybieranie, naciskanie oraz przemieszczanie obiektów. Dodatkowo w aplikacji zaimplementowano możliwość obrotu całego pomieszczenia wraz z obiektami o 90 stopni, co pozwala na dostosowanie orientacji przestrzeni do preferencji użytkownika. Rozwiązanie to zwiększa komfort użytkowania aplikacji i ułatwia rozwiązywanie zagadek.
Projekt interfejsu oraz środowiska gry został przetestowany w rzeczywistych warunkach Laboratorium Zanurzonej Wizualizacji Przestrzennej. Przeprowadzone testy pozwoliły na dopasowanie skali obiektów, czytelności komunikatów oraz sposobu interakcji do potrzeb użytkowników, zapewniając płynne i komfortowe korzystanie z aplikacji w środowisku VR.
\chapter{Technologie i narzędzia}
Realizacja projektu aplikacji edukacyjnej typu escape room w środowisku rzeczywistości wirtualnej wymagała doboru odpowiednich narzędzi oraz technologii, które umożliwiłyby stworzenie interaktywnej, atrakcyjnej wizualnie i funkcjonalnej aplikacji kompatybilnej z systemami dostępnymi w Laboratorium Zanurzonej Wizualizacji Przestrzennej.
\label{chap:research}

\section{Silnik gry (Konrad Czarnecki)}
Do stworzenia aplikacji wirtualnego escape roomu zdecydowano się na wykorzystanie silnika Unreal Engine 5, który jest jednym z najpopularniejszych środowisk do tworzenia gier komputerowych oraz aplikacji wirtualnej rzeczywistości. Unreal Engine, rozwijany przez firmę Epic Games, umożliwia tworzenie zaawansowanych wizualnie, interaktywnych projektów 3D oraz VR dzięki nowoczesnemu systemowi renderowania, rozbudowanemu edytorowi oraz wsparciu dla technologii immersyjnych.

Silnik ten oferuje użytkownikom możliwość programowania logiki aplikacji zarówno w języku C++, jak i z wykorzystaniem wizualnego systemu skryptowego Blueprint, co znacząco przyspiesza proces tworzenia prototypów i ułatwia implementację interakcji w środowisku VR. Unreal Engine zapewnia również bogaty zestaw narzędzi do tworzenia animacji, efektów specjalnych, obsługi dźwięku oraz integracji z zewnętrznymi bibliotekami.

Podczas analizy możliwych rozwiązań rozważano również wykorzystanie silnika Unity, który podobnie jak Unreal Engine jest szeroko stosowany w branży gier i aplikacji VR. Unity charakteryzuje się dużą elastycznością, wsparciem dla wielu platform oraz dostępnością licznych wtyczek i rozszerzeń. W porównaniu z Unreal Engine, Unity posiada mniej rozbudowany natywny system graficzny oraz wymaga większego nakładu pracy przy tworzeniu zaawansowanych efektów wizualnych.

Ostatecznym argumentem przemawiającym za wyborem Unreal Engine była kwestia kompatybilności — Laboratorium Zanurzonej Wizualizacji Przestrzennej, w którym aplikacja miała zostać wdrożona, nie wspiera nowszych wersji Unity, natomiast Unreal Engine zapewniał pełną zgodność z istniejącą infrastrukturą sprzętową i programową laboratorium.





\section{Laboratorium Zanurzonej Wizualizacji Przestrzennej (Konrad Czarnecki)}
Laboratorium Zanurzonej Wizualizacji Przestrzennej (LZWP) to specjalistyczne środowisko badawczo-edukacyjne wyposażone w systemy rzeczywistości wirtualnej, umożliwiające tworzenie i testowanie aplikacji immersyjnych w warunkach kontrolowanych. W skład laboratorium wchodzą między innymi systemy typu CAVE, czyli pomieszczenia projekcyjne z ekranami ściennymi i podłogowymi, które otaczają użytkownika obrazem 3D wyświetlanym z kilku projektorów.

LZWP wyposażone jest również w systemy śledzenia pozycji i ruchu użytkownika oraz kontrolery umożliwiające interakcję z wirtualnym środowiskiem. Dzięki temu laboratorium stanowi doskonałe zaplecze do testowania aplikacji edukacyjnych VR oraz prowadzenia badań nad efektywnością i ergonomią rozwiązań immersyjnych.




\section{Środowisko 3D (Konrad Czarnecki)}
W procesie tworzenia aplikacji niezbędne było opracowanie modeli 3D reprezentujących obiekty, elementy wystroju oraz interaktywne przedmioty pojawiające się w wirtualnym escape roomie. Do tego celu wykorzystano Blender — darmowe oprogramowanie do modelowania trójwymiarowego, animacji, teksturowania oraz renderowania.

Blender oferuje szeroki zakres narzędzi umożliwiających tworzenie szczegółowych modeli 3D, generowanie animacji oraz przygotowywanie materiałów i tekstur kompatybilnych z silnikami do tworzenia gier. Dzięki wsparciu dla formatów eksportowych takich jak FBX i GLTF, modele stworzone w Blenderze mogły zostać bezproblemowo zaimportowane do Unreal Engine i wykorzystane w aplikacji VR.



\section{System kontroli wersji (Konrad Czarnecki)}
Dla zapewnienia bezpieczeństwa danych oraz efektywnego zarządzania projektem zastosowano system kontroli wersji Git wraz z usługą hostingową GitHub. GitHub umożliwia przechowywanie kodu źródłowego, modeli 3D, plików dźwiękowych oraz dokumentacji projektowej w repozytorium zdalnym z możliwością współdzielenia zasobów pomiędzy członkami zespołu.

Dzięki systemowi kontroli wersji możliwe było śledzenie historii zmian, zarządzanie gałęziami projektowymi oraz szybkie przywracanie poprzednich wersji w przypadku wystąpienia błędów. W projekcie GitHub pełnił również rolę platformy do przechowywania backupów.
\chapter{Organizacja pracy (Jan Walczak)}
\label{organizacja_pracy}

Zarządzanie zespołem i dobra organizacja pracy to dwa kluczowe aspekty podczas pracy grupowej.
Metodyczne działanie pozwala usprawnić i uporządkować pracę projektową i komunikację 
w zespole \cite{Zarzadzanie_Projektami_IT_Przewodnik_po_Metodykach}. Zespół, który implementował omawiane w tej pracy rozwiązanie, składał się 
z~trzech osób: Jana Walczaka, Konrada Czarneckiego i Andriego Demyshyna.

\section{System kontroli wersji (Jan Walczak)}

\subsection{Zastosowane rozwiązanie}
System kontroli wersji pomaga śledzić historię zmian dokonywanych przez uczestników projektu.
Ułatwia zarządzanie procesem jego tworzenia, zapewnia integralność danych oraz gwarantuje,
że dokonywane zmiany są na bieżąco zapisywane i udostępniane uczestnikom oraz użytkownikom \cite{github_about_git}. 

Do wdrożenia systemu kontroli wersji w omawianym projekcie użyto internetowej platformy GitHub.
Organizacja pracy sprowadza się do opracowania zasad dotyczących tzw. \glqq Flow\grqq.
\glqq Flow \grqq pochodzi od angielskiego słowa \glqq workflow \grqq (\textit{ang. przepływu pracy}).
Taki przepływ ma na celu opisanie i~wdrażanie zmian zachodzących w danym projekcie tak, aby ułatwić 
ich zrozumienie nie tylko deweloperom, ale także użytkownikom, którzy śledzą zmiany zachodzące w repozytorium \cite{github_about_git}.  

\subsection{Kontrola przepływu}

Przepływ zaczyna się od zdefiniowania problemu, poprzez opisanie problemu lub funkcjonalności,
których będzie dotyczyć implementowana zmiana. Na platformie GitHub odbywa się to poprzez tworzenie
\verb|Issues| (\textit{zagadnień}). Pozwalają one również na planowanie i prowadzenie dyskusji
na temat danego zagadnienia z innymi uczestnikami projektu lub użytkownikami \cite{github_about_issues}.
Dodatkową zaletą jest możliwość bezpośredniego powiązania zmian dokonanych w kodzie z unikalnym numerem,
który jest nadawany każdej takiej dyskusji przez omawianą platformę.
W omawianym projekcie zdecydowano się na wprowadzanie zasad dotyczących formułowania tytułu i treści
\verb|Issue|. Tytuł powinien być napisany w języku angielskim, tak żeby każda osoba przeglądająca
repozytorium mogła powiązać problem i dyskusję ze zmianą w aplikacji. Dodatkowo powinien być zwięzły i dotyczyć
wyłącznie jednego zagadnienia. Opis i dyskusja mogą być napisane w języku polskim, tak żeby maksymalnie usprawnić komunikację zespołu.
\verb|Issue| może znajdować się w dwóch stanach: zamkniętym -- czyli stan, w którym praca została zakończona
i otwartym -- czyli stan, w którym praca nad danym zagadnieniem nadal trwa.

\begin{figure}[htbp]
    \centering
    \includegraphics[width=0.7\textwidth]{images/git_issues.png}
    \caption{Zrzut ekranu z platformy GitHub -- tablica z przykładowymi Issues do projektu.}
    \label{git:issues}
\end{figure}

\FloatBarrier

W repozytorium znajduje się tak zwana główna linia (\textit{ang. main line of development}).
Jest to liniowa historia zmian, które zostały wprowadzone do projektu. Deweloperzy mogą wprowadzać swoje zmiany,
które są definiowane tym, w jaki sposób (w którym punkcie w historii zmian) odłączyły się od tej linii.
W ten sposób można tworzyć rozgałęzienia (\textit{ang. branches}) względem głównej linii \cite{git_branches}. 
Główna gałąź (\textit{ang. main branch}), może umożliwić deweloperom zorientowanie się czy zmiana, którą chcą wprowadzić, 
będzie mogła być połączona z aktualnie istniejącą wersją aplikacji. Ważnym elementem przepływu jest więc zdefiniowanie 
tego w jaki sposób każda zmiana będzie dołączana do głównej gałęzi. Podczas implementacji projektu zdecydowano się na
bezpośrednie powiązanie tworzonych gałęzi z \verb|Issues|, tj. nazywanie ich tymi samymi tytułami oraz wiązanie ich przez linkowanie
unikalnego identyfikatora nadawanego każdemu \verb|Issue|.

\subsection{Łączenie gałęzi}

Domyślnie gałęzie mogą być wiązane bez żadnych ograniczeń. GitHub umożliwia tworzenie zbioru zasad (\textit{ang. rulesets})
dotyczących wiązania gałęzi deweloperów z innymi gałęziami. Zasady te pozwalają na określenie, które grupy są upoważnione
do jakich czynności związanych z daną gałęzią oraz definiują jakie zasady muszą spełnić, aby powiązać swoją gałąź,
czyli wprowadzaną zmianę, do danej linii. \cite{git_rulesets}. W projekcie zastosowano następujące zasady,
dotyczące głównej linii:
\begin{itemize}
    \item \verb|Restrict deletions| -- nie pozwalaj na usuwanie powiązanych referencji.
    \item \verb|Require a pull request before merging| -- powiąż z gałęzią przez \glqq pull request\grqq.
    \item \verb|Block force pushes| -- nie pozwól na wymuszenie nadpisania zmian.
\end{itemize}

W celu spełnienia zasady \verb|Require a pull request before merging|
każda wprowadzana zmiana musi zostać przygotowana do połączenia w odpowiedni sposób. Odbywa się to przez mechanizm
\verb|Pull request|. Umożliwia on wgląd we wprowadzane modyfikacje innym uczestnikom projektu przed ich finalnym zatwierdzeniem i wdrożeniem \cite{git_about_pull_requests}.
W projekcie określono, że żeby zmiana została połączona z główną gałęzią, musi być zatwierdzona przez minimalnie 
jednego uczestnika projektu, niebędącego autorem. Dodatkowo, zmiana musi być możliwa do powiązania bez występowania
konfliktów, czyli fragmentów plików, których system kontroli wersji nie mógł samodzielnie scalić ze zmianami \cite{git_about_pull_requests}.

Zmiany wprowadzane przez deweloperów składają się z serii migawek \verb|Commit|. \verb|Commit| zawiera wiadomość,
unikalny identyfikator oraz listę plików i zmian, które są z nim powiązane. Pozwala na
wielokrotny zapis zmian podczas pracy w ramach jednej gałęzi. W projekcie zostało nałożone ograniczenie, wymuszające
na uczestnikach scalanie migawek w jedność za pomocą mechanizmu \verb|rebase| i \verb|squash| \cite{git_about_pull_requests}.

Po poprawnym utworzeniu i scaleniu \verb|Pull request| utworzona wcześniej gałąź może zostać usunięta. Jest to bezpieczne,
ponieważ gałąź zostaje dołączona do głównej linii jako \verb|Commit| i historia zachowuje swój liniowy charakter. Deweloper
powinien oznaczyć \verb|Issue| jako zamknięte. \verb|Pull request| otrzyma status \verb|Merged| (scalone) automatycznie.

\begin{figure}[htbp]
    \centering
    \includegraphics[width=0.7\textwidth]{images/git_merged.png}
    \caption{Zrzut ekranu z platformy GitHub -- tablica z przykładowym Pull Request}
    \label{git:merged}
\end{figure}

\FloatBarrier

\subsection{Przechowywanie dużych plików -- Git LFS}
Projekt był realizowany w środowisku Unreal Engine, przez co w projekcie znajdowało się dużo plików o dużym rozmiarze.
Co do zasady Git nie jest przystosowany do przechowywania dużych plików (takich jak wideo, audio, grafiki), a jedynie
małych plików binarnych takich jak tekst czy pliki z kodem.

W projekcie zastosowano mechanizm \verb|Git LFS| czyli \glqq Git Large File Storage\grqq.  Jest to system, który
umożliwia tworzenie wskaźników kierujących do danego pliku, który jest umieszczany poza repozytorium (na zewnętrznym
serwerze) \cite{git_lfs}. Repozytorium wymaga wyspecyfikowania, które pliki powinny być przechowywane nie jako pliki
binarne w repozytorium, a jako dowiązania. W~tym celu należało zastosować odpowiednie filtry wskazujące na duże
obiekty tworzone w Unreal Engine: rozszerzenia \verb|*.uasset| oraz \verb|*.umap|.
W przypadku omawianej platformy zagwarantowany jest dysk o pojemności \textit{10~GB} znajdujący
się bezpośrednio po stronie platformy.

\section{Tablica Kanban (Jan Walczak)}
Aby dobrze zarządzać projektem potrzebna jest jasna komunikacja i wyznaczanie celów już od
najwcześniejszych etapów rozwoju projektu. Praktyki definiujące wymagania i przydzielające uczestników
projektu do poszczególnych zadań znacząco zwiększają ich zaangażowanie i gwarantują, że aplikacja będzie spełniać
potrzeby zdefiniowane przez interesariuszy \cite{rational-unified-process}.  

Żeby zapewnić powyższe wymagania zdecydowano się na zastosowanie zwinnej metodyki Kanban. Jednym z jej głównych zadań jest
wizualizacja działań podejmowanych w projekcie \cite{kanban}. Metodyka jest często stosowana w formie fizycznej i występuje
wtedy jako tablica podzielona na odpowiednie kolumny. Standardowy, trzykolumnowy podział prezentuje się następująco:
\begin{itemize}
    \item ToDo -- zadania do zrobienia,
    \item Doing -- zadania, nad którymi trwa praca,
    \item Done -- zadania zakończone.
\end{itemize}
Uczestnicy umieszczają samoprzylepne karteczki z krótko opisanymi zadaniami. W opisywanym projekcie zastosowano narzędzie
internetowe, służące do wizualizacji wykonywanych zadań. Każdy uczestnik otrzymał przydział zadań według deklarowanych przez niego
preferencji. Dodano dodatkową kolumnę \glqq testing\grqq, w której były umieszczane zadania, które były w danej chwili
testowane.

\begin{figure}[htbp]
    \centering
    \includegraphics[width=0.7\textwidth]{images/kanban_nasze.png}
    \caption{Zrzut ekranu z platformy Trello -- wizualizacja tablicy Kanban}
    \label{git:merged}
\end{figure}

\FloatBarrier

Kolejnym, charakterystycznym elementem metodyki Kanban jest kontrola przepływem \cite{kanban}. Dzięki sekwencyjnemu
przemieszczaniu karteczek z zadaniami na tablicy, zgodnie z zasadą, że można przesunąć tylko jedną karteczkę na raz
(od lewej do prawej), uczestnicy mają możliwość śledzenia postępu prac. Aby kontrola przepływem działała sprawnie
konieczna jest samodyscyplina uczestników w uaktualnianiu postępów wykonywanych zadań. 

Kontrola przepływem wiąże się bezpośrednio z zasadą limitu WIP (\textit{ang. Work In Progress}) określoną dla metodyki, 
która odróżnia ją od innych, zwinnych metodyk \cite{kanban}. W przypadku opisanym w projekcie określała ona nad iloma zagadkami
na raz może pracować każdy uczestnik projektu. Zdecydowano się na ustalenie tego limitu na jedną zagadkę na raz. W takim przypadku
każdy uczestnik projektu mógł umieścić tylko jedną karteczkę, przypisaną do niego, w kolumnie \glqq Doing\grqq.
%\chapter{Analiza dydaktyczna i potencjał edukacyjny}
\label{chap:an}

\section{Motywacja uczniów do nauki matematyki}

\section{Możliwość personalizacji i różnicowania poziomu trudności}

\section{Wpływ imersji na naukę}
\chapter{Wstępny projekt zagadek}
\label{chap:theory}


W poniższym rozdziale znajduje się teoretyczny opis zagadek, który stanowi punkt wejściowy do implementacji
projektu inżynierskiego. Stanowi on podstawę teoretyczną i nakreśla charakter pracy. Opisy poszczególnych zagadek
zostały zredagowane na podstawie ogólnodostępnej podstawy programowej dla szkół średnich z przedmiotu matematyka.
Jest to kluczowy aspekt całego projektu inżynierskiego, gdyż aplikacja końcowa ma być skierowana do uczniów
szkół średnich, w związku z czym wymagany zakres wiedzy nie może wykraczać poza podstawę programową.











\section{Zadanie 1 – Wzory skróconego mnożenia (Andrii Demyshyn)}
\subsection{Cel zadania}
Celem zadania jest utrwalenie wiedzy z zakresu wzorów skróconego mnożenia oraz rozwijanie umiejętności rozpoznawania poprawnych zależności algebraicznych. Zadanie ma charakter wprowadzający i pozwala uczestnikowi na przypomnienie  sobie podstawowych wzorów omawianych w programie nauczania matematyki na poziomie szkoły średniej. Dodatkowo zadanie wspiera rozwój logicznego myślenia oraz umiejętność analizy struktury wyrażeń algebraicznych.
\subsection{Zasady działania}
Zadanie polega na dopasowaniu początków wzorów skróconego mnożenia do ich poprawnych zakończeń. Uczestnik otrzymuje zestaw rozpoczętych wyrażeń algebraicznych, w których brakują prawe strony równań. Równocześnie dostępny jest zbiór możliwych zakończeń wzorów, wśród których znajdują się zarówno poprawne, jak i niepoprawne zapisy algebraiczne (rys.~\ref{fig:wzory}).
\begin{figure}[H]
    \centering
    \includegraphics[width=0.8\textwidth]{images/wzory.png}
    \caption{Przykładowy wygląd pierwszego zadania – wzory skróconego mnożenia}
    \label{fig:wzory}
\end{figure}
Celem użytkownika jest wybranie właściwych elementów i połączenie ich w taki sposób, aby utworzyć kompletne i matematycznie poprawne wzory skróconego mnożenia. Zadanie wymaga znajomości podstawowych własności działań algebraicznych oraz umiejętności odróżniania poprawnych wzorów od błędnych.
\subsection{Zakończenie zadania}
Zadanie uznaje się za zakończone w momencie poprawnego uzupełnienia wszystkich wzorów skróconego mnożenia. Prawidłowe rozwiązanie potwierdza, że uczestnik posiada wymaganą wiedzę teoretyczną oraz potrafi ją zastosować w praktyce poprzez rozpoznawanie i kompletowanie wyrażeń algebraicznych.














\section{Zadanie 2 – Planimetria (Jan Walczak)}
\label{subsec:planimetria_teoria}
Po ukończeniu pierwszego zadania pojawia się tabela z trzema kolumnami.
Dwie skrajne kolumny zawierają nazwy figur geometrycznych. Środkowa kolumna jest pusta i będzie uzupełniana przez gracza 
odpowiednimi symbolami.  
\subsection{Cel zadania}
Celem gracza jest poprawne ułożenie relacji między figurami w kolejnych rzędach. 
Przykładowo, relacją taką jest to, że „każdy kwadrat jest rombem” lub „każdy kwadrat jest prostokątem”.
\subsection{Zasady działania}
Gracze przechodzą przez kolejne rzędy sekwencyjnie (rys. \ref{fig:przyklad}). Dopóki nie ułożą pierwszej relacji poprawnie,
to nie mogą ułożyć kolejnej. Po poprawnym ułożeniu rzędu (wstawieniu odpowiedniego symbolu) podświetla się on na zielono, 
sygnalizując graczowi, że może przejść do kolejnego punktu. Symbole, które układają gracze są następujące:
\begin{itemize}
    \item > to <
    \item > to nie <
\end{itemize}
\begin{figure}[htbp]
    \centering
    \begin{tabular}{|c|c|c|}
        \hline
        Kwadrat & > to < & Prostokąt \\ \hline
        Prostokąt & > to nie < & Kwadrat \\ \hline
        Romb & [      ] & Kwadrat \\ \hline
        ... & ... & ... \\ \hline
    \end{tabular}
    \caption{Przykładowy diagram zależności między figurami geometrycznymi.}
    \label{fig:przyklad}
\end{figure}

\medskip
  
Gracze kierują się swoją wiedzą matematyczną oraz następującymi własnościami figur:

\begin{itemize}[left=1.5em, label=\textbullet, topsep=0pt, itemsep=0pt]
    \item liczba boków,
    \item długości boków,
    \item kąty (proste lub nie),
    \item cechy charakterystyczne danej figury.
\end{itemize}
\subsection{Zakończenie zadania}
Po poprawnym uzupełnieniu wszystkich relacji zadanie uznaje się za zakończone. Gracz może przejść
do kolejnego zadania.













\section{Zadanie 3 – Nierówności (Konrad Czarnecki)}
\label{subsec:nierownosci_teoria}
Po wejściu do kolejnego pomieszczenia gracz widzi przed sobą most zawieszony nad przepaścią.
Po drugiej stronie znajduje się zamknięte przejście, do którego gracz musi się dostać.
Na bocznych ścianach wypisany jest układ nierówności:
    $2x - 5 < 9$;
    $x + 1 \ge 4$.
Na płytkach mostu umieszczone są różne liczby z zakresu liczb całkowitych (np.: $1$, $2$, $3$, $4$, $5$, $6$, $8$, $7$, $10$).
\subsection{Cel zadania}
Gracz musi rozwiązać układ nierówności i wyznaczyć przedział, który spełnia oba warunki. Następnie powinien przeprowadzić
niewielką figurkę przez most, odpowiednio nią poruszając, tak aby przesuwała się wyłącznie po płytkach z wartościami należącymi
do tego przedziału. Poruszanie figurką odbywa się za pomocą czterech przycisków (w prawo, w lewo, do przodu i do tyłu) znajdujących
się na jednej ze ścian.
\medskip
Rozwiązanie układu
\begin{enumerate}[left=1.5em, topsep=0pt, itemsep=0pt]
    \item Rozwiązanie pierwszej nierówności:
    \begin{itemize}[left=1.5em, label=\textbullet, topsep=0pt, itemsep=0pt]
        \item $2x - 5 < 9$
        \item $2x < 14$
        \item $x < 7$
    \end{itemize}
    \item Rozwiązanie drugiej nierówności:
    \begin{itemize}[left=1.5em, label=\textbullet, topsep=0pt, itemsep=0pt]
        \item $x + 1 \ge 4$
        \item $x \ge 3$
    \end{itemize}
    \item Wspólny przedział:
    \begin{itemize}[left=1.5em, label=\textbullet, topsep=0pt, itemsep=0pt]
        \item \textit{$x \in [3; 7)$}
    \end{itemize}
\end{enumerate}
\medskip
Gracz musi wybrać wyłącznie płytki z wartościami większymi lub równymi $3$ i mniejszymi od $7$, np. $3$, $4$, $5$, $6$.
\subsection{Zasady działania}
\begin{itemize}[left=1.5em, label=\textbullet, topsep=0pt, itemsep=0pt]
    \item Gracz poruszając postacią, przechodzi przez kolejne płytki.
    \item Poprawna płytka – postać przechodzi dalej.
    \item Błędna płytka – płytka zapada się lub podświetla na czerwono, a postać wraca na początek mostu. Gracz może próbować dowolną liczbę razy, aż do skutecznego przejścia na drugą stronę.
\end{itemize}
\subsection{Zakończenie zadania}
Zadanie zostaje uznane za zakończone, gdy graczowi uda się przejść na drugą stronę mostu.



















\section{Zadanie 4 – Funkcje (Konrad Czarnecki)}
\label{subsec:funkcje_teoria}
Zadania 4 i 5 są realizowane w jednym pomieszczeniu. Gracz widzi dwie duże tablice, umieszczone na ścianach. Na obu naniesiona jest siatka układu współrzędnych. W układzie współrzędnych pojawiają się trzy punkty, np.:
\begin{itemize}[left=1.5em, label=\textbullet, topsep=0pt, itemsep=0pt]
    \item Punkt $A (1, 2)$
    \item Punkt $B (3, 4)$
    \item Punkt $C (5, 6)$
\end{itemize}
oraz wzór funkcji kwadratowej ze wszystkimi współczynnikami domyślnie ustawionymi na $0$.
Poniżej widoczne są elementy służące do sterowania wykresem funkcji. Dają one możliwość zmiany współczynników wylosowanego wzoru.
\subsection{Cel zadania}
Celem gracza jest dobranie odpowiednich współczynników funkcji krawdatowej za pomocą wirtualnego suwaka tak, aby przeszła ona przez wszystkie wyświetlone punkty. 
\subsection{Zasady działania}
Podczas wybierania współczynników funkcji przez gracza, jej wykres na tablicy zmienia się na bieżąco zgodnie z ustawionym wzorem. Wykres funkcji powinien być losowany z wcześniej zdefiniowanej puli par typu funkcja – punkty, tak aby po każdorazowym uruchomieniu zadania gracz czuł, że ma przed sobą nowe wyzwanie.
\subsection{Zakończenie zadania}
Rozwiązanie jest sprawdzane na bieżąco i gdy gracz dobierze współczynniki funkcji kwadratowej poprawnie, tj. przetnie ona wszystkie wybrane punkty, zadanie zostaje uznane za rozwiązane. W takim przypadku tablica się blokuje i podświetla na zielono.

















\section{Zadanie 5 – Geometria analityczna (Konrad Czarnecki)}
\label{subsec:geometria_analityczna_teoria}
W układzie współrzędnych pojawiają się dwa wykresy funkcji liniowych. Są one losowane z pewnej określonej puli, podobnie jak w zadaniu 4. 
\subsection{Cel zadania}
Zadaniem gracza jest znalezienie punktu przecięcia dwóch funkcji liniowych wyświetlanych w układzie współrzędnych. Określenie puli wyklucza możliwość prostych równoległych, które nie mają ze sobą żadnych punktów wspólnych lub w całości się pokrywają. W tych przypadkach zadanie byłoby niemożliwe do rozwiązania. Gracz będzie wpisywał swoją odpowiedź (współrzędne punktu przecięcia) na klawiaturze znajdującej się pod tablicą.
\subsection{Zasady działania}
Skala osi współrzędnych jest dobrana tak, by gracz nie mógł odczytać z niej rozwiązania; musi rozwiązać układ równań dysponując dwoma wzorami funkcji. Przykładowe dwa wykresy przecinających się funkcji (rys. \ref{fig:przykladowy_wykres_funkcji}): 
\begin{itemize}[left=1.5em, label=\textbullet, topsep=0pt, itemsep=0pt]
    \item $y = 3x + 4$ (czerwony)
    \item $y = 5x + 8$ (niebieski)
\end{itemize}
\begin{figure}[htbp]
    \centering
    \includegraphics{images/przykładowy_wykres_funkcji.png}
    \caption{Przykładowy wygląd tablicy z przecinającymi się wykresami}
    \label{fig:przykladowy_wykres_funkcji}
\end{figure}
Gracz dysponuje tymi wzorami i na ich podstawie musi rozwiązać układ równań. Wykresy powinny być dobrane tak, aby obie współrzędne punktu przecięcia były liczbami całkowitymi. 
\subsection{Zakończenie zadania}
Rozwiązanie jest sprawdzane na bieżąco i gdy gracz dobierze współrzędne punktu przecięcia funkcji poprawnie, zadanie zostaje uznane za rozwiązane. W takim przypadku tablica się blokuje i podświetla na zielono. Jeśli gracz wykonał wcześniej zadanie czwarte (znajdujące się w tym samym pokoju), przechodzi do następnego zadania.
















\section{Zadanie 6 – Kombinatoryka (Andrii Demyshyn)}

\subsection{Cel zadania}
Gracz zostaje poinformowany, że musi wyliczyć, ile jest możliwych do ułożenia trzycyfrowych kodów do sejfu bez powtarzających się cyfr.

\subsection{Zasady działania}
Na środku pokoju znajduje się zamknięty sejf (rys.~\ref{fig:sejf}). Rozwiązaniem jest kod, a nie liczba – oznacza to, że początkową cyfrą w trzycyfrowym kodzie może być zero. Informacja ta powinna być jasno zakomunikowana graczowi, np. na tabliczce umieszczonej nad sejfem.

Rozwiązaniem zadania jest:
\[
10 \cdot 9 \cdot 8 = 720
\]

\subsection{Zakończenie zadania}
Po poprawnym wpisaniu kodu sejf automatycznie się otwiera, co utwierdza gracza w przekonaniu, że poprawnie wykonał zadanie. W środku znajdują się elementy potrzebne do wykonania następnego zadania.

\begin{figure}[H]
    \centering
    \includegraphics[width=0.8\textwidth]{images/sejf.png}
    \caption{Przykładowy wygląd sejfu występującego w zadaniu}
    \label{fig:sejf}
\end{figure}















\section{Zadanie 7 – Liczby rzeczywiste i działania na zbiorach liczbowych \\(Jan Walczak)}
\label{subsec:liczby_rzeczywiste}

W następnym zadaniu gracze znajdują skrzynię, na której powierzchni – z każdej strony – narysowane są liczby:
\textit{-3; 7; 12; 2; 5; 0; 10; 3,75; √2}.

\subsection{Cel zadania}
Skrzynia wyposażona jest w mechaniczny zamek z pięcioma polami na cyfry, obok których znajdują się symbole zbiorów:
\begin{itemize}
    \item \textit{N} – liczby naturalne,
    \item \textit{Z} – liczby całkowite,
    \item \textit{R} – liczby rzeczywiste,
    \item \textit{Q} – liczby wymierne,
    \item \textit{R\textbackslash Q} – liczby niewymierne,
\end{itemize}
Gracz musi uważnie obejrzeć skrzynię i policzyć ile liczb, zapisanych na skrzyni, należy do jakiego zbioru. 
Po wpisaniu odpowiednich liczb, obok symboli zbiorów, skrzynia otwiera się, a gracz otrzymuje kilka symboli: 
\begin{itemize}
    \item zawieranie się zbiorów – $\subset$,
    \item zbiór pusty – $\varnothing$,
    \item iloczyn zbiorów – $\cap$.
\end{itemize}
Tym samym gracz przechodzi do drugiego etapu zadania. W drugim etapie w pokoju ukazuje się plansza z symbolami zbiorów, takimi jak na skrzyni. 
Celem gracza jest ułożenie poprawnej relacji między nimi tj. zbiór liczb naturalnych zawiera się w zbiorze liczb całkowitych, zbiór liczb całkowitych zawiera się w zbiorze liczb rzeczywistych itd.

\subsection{Zasady działania}
W pierwszym etapie zadania gracz powinien policzyć, ile liczb ze skrzyni należy do danego zbioru 
i wpisać odpowiednią odpowiedź na kłódce. Przykładowo:
\begin{itemize}
    \item \textit{N} (naturalne): \textit{ 7; 12; 2; 5; 10} – 5 liczb,
    \item \textit{C} (całkowite):  \textit{-3; 7; 12; 2; 5; 0; 10} – 7 liczb,
    \item \textit{R} (rzeczywiste): wszystkie liczby – 9 liczb,
    \item \textit{Q} (wymierne): \textit{-3; 7; 12; 2; 5; 0; 10; 3,75} – 8 liczb,
    \item \textit{R\textbackslash Q} (niewymierne): \textit{√2} – 1 liczba.
\end{itemize}
Kombinacja do ustawienia na kłódce: \textit{5, 7, 9, 8, 1}.

W drugim etapie gracz powinien ustawić posiadane symbole między literami symbolizującymi kolejne zbiory i odpowiednio je obrócić, tak aby powstała między nimi poprawna relacja tj.

\begin{itemize}
    \item $N \subset Z \subset R$,
    \item $R\textbackslash Q \cap Q = \varnothing$,
    \item $R\textbackslash Q \cap R = R\textbackslash Q$.
\end{itemize}

\subsection{Zakończenie zadania}
Zadanie zostaje uznane za zakończone, gdy gracz poprawnie przejdzie przez oba etapy. 
Gracz nie może przejść do etapu drugiego, bez zakończenia etapu pierwszego – jest to wymuszone 
przez wymaganie otwarcia przez niego skrzyni.
















\section{Zadanie 8 – Znaki funkcji trygonometrycznych (Konrad Czarnecki)}
\label{subsec:znaki_funkcji_trygonometrycznych_teoria}
Po ukończeniu zadania ze zbiorami gracz kieruje się do ściany z czterema dużymi okręgami jednostkowymi, oznaczonymi nazwami funkcji:
\begin{itemize}[left=1.5em, label=\textbullet, topsep=0pt, itemsep=0pt]
    \item $sin(\alpha)$
    \item $cos(\alpha)$
    \item $tg(\alpha)$
    \item $ctg(\alpha)$
\end{itemize}
Każdy z okręgów podzielony jest na cztery ćwiartki, oznaczone jako $I$, $II$, $III$, $IV$.
Przy każdej ćwiartce znajduje się puste miejsce, które gracz musi wypełnić odpowiednim znakiem (rys. \ref{fig:cwiartki_ukladow_jednostkowych}):
\begin{itemize}[left=1.5em, label=\textbullet, topsep=0pt, itemsep=0pt]
    \item „+” (plus) — funkcja przyjmuje wartości dodatnie
    \item „−” (minus) — funkcja przyjmuje wartości ujemne
\end{itemize}
Gracz zmienia znak naciskając na niego. Domyślnie wszystkie znaki są ustawione na puste, dopiero po pierwszym naciśnięciu zmieniają się na znak „+”, po kolejnym na „−”, następnie znów na „+”, itd.
\begin{figure}[htbp]
    \centering
    \includegraphics[width=\textwidth]{images/cwiartki_ukladow_jednostkowych.png}
    \caption{Ćwiartki układow jednostkowych ze znakami funkcji trygonometrycznych}
    \label{fig:cwiartki_ukladow_jednostkowych}
\end{figure}
\subsection{Cel zadania}
Celem gracza jest poprawne ustawienie znaków funkcji trygonometrycznych w każdej ćwiartce układu współrzędnych.

\subsection{Zasady działania}
Gracz modyfikuje znaki „+” i „−” w odpowiednich miejscach na planszy naciskając na nie. Może dowolnie poprawiać swój wybór, dopóki nie zatwierdzi odpowiedzi.

\subsection{Zakończenie zadania}
Po poprawnym ułożeniu wszystkich znaków ściana rozświetla się na zielono.















\section{Zadanie 9 – Ciągi liczbowe (Jan Walczak)}
\label{sec:sequence}
Poniżej opisane zadanie będzie podobne do gry wyprodukowanej przez Nintendo na platformę Pegasus (rys. \ref{duckhunt}).

\subsection{Cel zadania}
Zadaniem gracza będzie odpowiednio szybko obliczyć kolejne wyrazy ciągu arytmetycznego 
z podanego wzoru. Będzie on wyposażony w wirtualny, laserowy pistolet, którym będzie musiał strzelać
w odpowiednio oznaczone kaczki.
\subsection{Zasady działania}
Na jednej ze ścian pokoju pojawia się formuła ciągu arytmetycznego, np.: $a_n = 2n + 1$ wybranego z predefiniowanej puli. 
Gracz będzie musiał wybrać i strzelić do kaczki oznaczonej odpowiednią wartością kolejnych wyrazów ciągu. Gra przyspiesza (kaczki lecą coraz szybciej) razem z postępem w zadaniu. 
Aby ułatwić graczowi zadanie, na górze jednej ze ścian pokoju będzie podany nie tylko wzór ciągu, ale też aktualny numer wyrazu tego ciągu. 
Przykładowo, dla wzoru $a_n = 2n + 1$:
\begin{itemize}
    \item $n = 1$ – gracz musi trafić w kaczkę z liczbą 3,
    \item $n = 2$ – gracz musi trafić w kaczkę z liczbą 5
\end{itemize}

\subsection{Zakończenie zadania}
Gracz będzie zdobywał punkty za każdy poprawnie wybrany wyraz ciągu. 
Gra zakończy się po upływie określonego czasu lub jeśli gracz pomyli się trzy razy.

\begin{figure}[htbp]
    \centering
    \includegraphics{images/duckhunt.jpg}
    \caption{gra Duck Hunt}
    \label{duckhunt}
\end{figure}







\section{Zadanie 10 – Prawdopodobieństwo (Andrii Demyshyn)}
Gracze podchodzą do stołu ustawionego w rogu pokoju, gdzie siedzi postać Morfeusza. Na stole stoją dwa identyczne pojemniki, a obok leży 100 tabletek — 50 czerwonych i 50 niebieskich (rys.~\ref{fig:Morfeus}).

Pojawia się komunikat:
„Pomóż Morfeuszowi zwiększyć jego szanse na powrót do rzeczywistości.
Podziel tabletki między dwa pojemniki tak, aby miał jak największą szansę na wybranie czerwonej.”

\subsection{Cel zadania}
Gracze muszą znaleźć optymalny układ, czyli:
w pierwszym pojemniku umieścić jedną czerwoną tabletkę,
w drugim pojemniku umieścić 49 czerwonych i 50 niebieskich tabletek (lub odwrotnie. Rozwiązanie działa niezależnie od tego, który pojemnik traktujemy jako „pierwszy”, a który jako „drugi”).
Taki układ osiągnie maksymalne prawdopodobieństwo wylosowania czerwonej tabletki – 74,75\%.

\subsection{Zasady działania}
Gracze mogą:
przeciągać tabletki do pojemników w dowolny sposób,
testować różne rozkłady,
sprawdzać procent szans na wygraną, który wyświetla się po każdym rozłożeniu.

\begin{figure}[H]
    \centering
    \includegraphics[width=0.8\textwidth]{images/morfeus.png}
    \caption{Przykładowy wygląd w zadaniu}
    \label{fig:Morfeus}
\end{figure}

Po każdym rozkładzie system oblicza i wyświetla szansę na wygraną.
Jeśli gracz nie osiągnie 74,75\%, pojawia się komunikat:
„Możesz to zrobić lepiej! Spróbuj jeszcze raz.”
Jeśli gracz osiągnie 74,75\% system gratuluje i zalicza zadanie.

\subsection{Zakończenie zadania}
Po poprawnym rozłożeniu tabletek i osiągnięciu maksymalnej szansy 74,75\%, Morfeusz wstaje od stołu, uśmiecha się i mówi:
„Dziękuję. Dzięki Wam mam szansę wrócić do rzeczywistego świata”.
Po chwili Morfeusz znika, rozpuszczając się w powietrzu niczym hologram lub efekt teleportacji, symbolizujące jego powrót do rzeczywistości.










\section{Zadanie 11 – Optymalizacja i rachunek różniczkowy (Autor)}
a
\subsection{Cel zadania}
a
\subsection{Zasady działania}
a
\subsection{Zakończenie zadania}


\section{Zadanie 12 – Układy równań (Andrii Demyshyn)}

\subsection{Cel zadania}
Na środku pomieszczenia znajduje się stół z trzema przezroczystymi pojemnikami, oznaczonymi kolorami: czerwonym (x), zielonym (y) oraz niebieskim (z). Bezpośrednio nad stołem, na ścianie, umieszczona jest duża żarówka, która zapala się, gdy gracze prawidłowo uzupełnią pojemniki (rys.~\ref{fig:Lamp}). Obok znajduje się instrukcja techniczna:
„Aby uruchomić instalację świetlną, należy załadować dokładnie 12 kulek. Wszystkie kulki razem dają 12 jednostek energii. Czerwona kulka jest dwa razy bardziej wydajna niż zielona. Czerwona i dwie zielone razem dają o 9 jednostek energii więcej niż niebieska”.

\subsection{Zasady działania}
\begin{itemize}
    \item Gracze wkładają kolorowe kulki do pojemników.
    \item System na bieżąco sprawdza poprawność ustawienia.
    \item W razie błędu wyświetlany jest komunikat: „Niepoprawne ustawienie. Spróbuj ponownie”.
\end{itemize}

\[
\begin{cases}
x = 2y \\
x + 2y - z = 9 \\
x + y + z = 12
\end{cases}
\quad
\begin{aligned}
&x &&\text{– energia generowana przez kulki czerwone}, \\
&y &&\text{– energia generowana przez kulki zielone}, \\
&z &&\text{– energia generowana przez kulki niebieskie}.
\end{aligned}
\]

\begin{itemize}
    \item Czerwony (x) = 6,
    \item Zielony (y) = 3,
    \item Niebieski (z) = 3.
\end{itemize}

\begin{figure}[H]
    \centering
    \includegraphics[width=0.4\textwidth]{images/Lamp.png}
    \caption{Przykładowy wygląd zadania}
    \label{fig:Lamp}
\end{figure}

\subsection{Zakończenie zadania}
Po poprawnym ułożeniu żarówka zapala się nad stołem, sygnalizując zakończenie zadania.









\section{Zadanie 13 – Stereometria (Jan Walczak)}
\label{stereometria:teoria}
Bryły przestrzenne mogą sprawić uczniom trudność, ponieważ wymagają przejścia od rysunku 
dwuwymiarowego tj. takiego na kartce papieru, do wyobrażenia sobie ich w przestrzeni trójwymiarowej.
\subsection{Cel zadania}
Celem zadania jest zapoznać uczniów z podstawowymi własnościami brył geometrycznych poprzez ich samodzielne odkrywanie.
\subsection{Zasady działania}
Gracz wchodzi do ciemnego pokoju i jest wyposażony w źródło światła np. pochodnię lub latarkę. 
Na środku pokoju znajduje się jedna z brył, wylosowanych z danej puli: prostopadłościan, ostrosłup, graniastosłup, kula lub sześcian.
Na ścianie pokoju znajduje się panel z pytaniami np.:
\begin{itemize}
    \item Jak nazywa się ta bryła?
    \item Jaki jest wzór na obliczenie jej pola powierzchni?
    \item Jaki jest wzór na obliczenie jej objętości?
\end{itemize}
Gracz obchodząc bryłę dookoła i rozświetlając ją latarką powinien móc udzielić odpowiedzi na te pytania. 
\subsection{Zakończenie zadania}
Gdy gracz udziela poprawnych odpowiedzi na kolejne pytania zostają one podświetlone na zielono, sygnalizując graczowi, że wykonał zadanie poprawnie. 
Jeżeli się pomyli to losowana jest kolejna bryła, inna niż poprzednia. Gra kończy się w momencie kiedy gracz udzieli poprawnych
odpowiedzi na pytania dotyczące dwóch różnych brył.
\chapter{Implementacja zagadek}
\label{chap:development}

\section{Zadanie 1 – wzory skróconego mnożenia (Autor)}
a
\section{Zadanie 2 – Planimetria (Jan Walczak)}
\label{sec:planimetria_praktyka}

\subsection{Problemy i różnice w realizacji zadania w praktyce}
Podczas implementacji zadania, zgodnie z teorią opisaną w podrozdziale \ref{subsec:planimetria_teoria}
napotkałem się z kilkoma problemami. Przede wszystkim zauważyłem, że odpowiedzi są zero--jedynkowe.
Przykładowo: uczeń ma do uzupełniania relację typu \glqq każdy kwadrat ... prostokątem\grqq. Jeśli 
odpowie niepoprawnie tj. zaznaczy odpowiedź \glqq nie jest\grqq, a zostanie od razu zapytany ponownie o tę
samą relację, to od razu wyklucza jedną z odpowiedzi. Tym samym zadanie zatraca swoją wartość edukacyjną
-- uczeń może rozwiązać całe zadanie stosując jedynie metodę eliminacji.

Rozwiązanie tego problemu, które zostało zaimplementowane, to zdefiniowanie puli takich relacji i po udzieleniu
przez ucznia odpowiedzi, każdorazowe losowanie relacji innej niż ta poprzednia.
W ten sposób uniemożliwia się uczniowi stosowania zasady eliminacji i wymusza na nim prawidłowe podejście.

Kolejnym problemem była prezentacja zadania. Zwykła tabela, którą uczeń miałby uzupełniać mogłaby wydać mu się
mało ciekawa. Tym samym postanowiłem wizualnie usprawnić zagadkę. Na środku zostały umieszczone dwa przyciski:
\begin{itemize}
    \item każdy
    \item nie każdy
\end{itemize}
Uczeń zostaje poinstruowany, że po obu ścianach pokoju zostaną wyświetlone różne figury geometryczne. Po lewej stronie,
patrząc od przycisków -- figury oznaczone kolorem czerwonym, po prawej stronie -- figury oznaczone kolorem zielonym.
Liczba figur jest stała, każdorazowo typ wyświetlanych figur jest wybierany z określonej puli i wyświetlany w pseudolosowej 
konfiguracji -- losowane jest ich położenie, obrót oraz rozmiar. Zadaniem ucznia jest uzupełnianie kolejnych relacji poprzez
wybieranie odpowiednich przycisków. Relacja wyświetlana na ścianie pokoju przedstawia się jako:
\glqq każdy typ figury, narysowany kolorem czerwonym, jest równocześnie typem figury oznaczonym kolorem zielonym\grqq.
Dodatkowo, tekst jest odpowiednio pokolorowany, tak aby uczeń nie miał wątpliwości, że chodzi o typ figury, wyświetlane tymże kolorem
na ścianach.

\subsection{Implementacja struktury danych przechowującej relację}
Na początku pracy należało zdefiniować strukturę, przechowującą relację, czyli innymi słowami, pytanie na które uczeń
będzie odpowiadał. Relacja taka została zdefiniowana jako aktor. Zawiera pola:

\begin{itemize}
    \item \verb|Every| -- wartość logiczna
    \item \verb|FigureA| -- ciąg tekstowy, reprezentujący pierwszą figurę w relacji
    \item \verb|FigureB| -- ciąg tekstowy, reprezentujący drugą figurę w relacji
\end{itemize}

Wartość zmiennej \verb|Every| odpowiada na pytanie, czy każda figura typu pierwszego (\verb|FigureA|) jest równocześnie
figurą typu drugiego (\verb|FigureB|). Struktura zawierała również funkcję słownikową, czyli taką, która tłumaczy wartości
tekstowe na liczbowe.

\subsection{Implementacja kontrolerów}
Aby zarządzać zagadką została zaimplementowana seria kontrolerów i menadżerów.
\begin{itemize}
    \item \verb|ControlerFigures| (główny kontroler)
    \item \verb|FiguresTextManager|
    \item \verb|WallFigures|
\end{itemize}

\verb|WallFigures| to najprostszy z kontrolerów. Wykonuje polecenia głównego kontrolera. Jego zadaniem jest 
wyświetlanie żądanych figur i ich losowe ustawianie (obracanie, skalowanie, rozmieszczanie). W projekcie występują
jego dwie instancje: kontroler lewy i prawy. Odpowiadają za odpowiednie ściany, na których wyświetlane są figury.

\verb|FiguresTextManager| jest odpowiedzialny za zarządzanie tekstem wyświetlanym na ekranie. Wykonuje polecenia
głównego kontrolera. Jego zadaniem jest odpowiednie wyświetlanie i kolorowanie tekstu.

\verb|ControlerFigures| czyli kontroler główny jest najważniejszym elementem zagadki. Zawiera referencję do pozostałych
kontrolerów, zarządza obiektami wyświetlanymi na scenie i kontroluje przebieg zadania. Dodatkowo zawiera dwustronną
referencję z instancjami przycisków \verb|Button|, które wysyłają do niego powiadomienia o tym, że zostały naciśnięte.
Kontroler zawiera też predefiniowaną tablicę relacji typów figur opisanych wcześniej.

\begin{figure}[htbp]
    \centering
    \includegraphics[width=0.7\textwidth]{images/planimetreia_UML.png}
    \caption{Diagram UML zawierający najważniejsze elementy kontrolerów dla zadania 2 -- planimetria.}
    \label{planimetria:uml}
\end{figure}

\FloatBarrier

\subsection{Przebieg zadania}
Główny kontroler jest odpowiedzialny za inicjalizację zadania. Uruchamia on funkcję 
\verb|InitBlueprint|, odpowiedzialną za wywołanie odpowiednich funkcji, z podrzędnych 
elementów, takich jak: ustawienie tekstu, inicjalizacja ścian i przycisków. Ustawia także początkowe wartości zmiennych dla zadania.

Kontroler oczekuje na otrzymanie powiadomienia od przycisku odpowiedzialnego za uruchomienie gry.
Kiedy otrzyma dane powiadomienie, za pośrednictwem odpowiedniej funkcji, uruchamia grę tj. wywołuje
funkcje podrzędnych komponentów ustawiając im odpowiednią widoczność na scenie w grze.

Z predefiniowanej tablicy relacji zostaje wybrana jedna, która zostaje wyświetlona graczowi za pośrednictwem
wywołania odpowiednich funkcji.
Główny kontroler czeka na informacje od przycisków. Kiedy zostaje powiadomiony o tym, że jeden z nich, został naciśnięty
sprawdza poprawność odpowiedzi. Jeśli odpowiedź jest poprawna, usuwa relację z tablicy i  losuje kolejną. Jeśli nie jest poprawna,
losuje kolejną relację do wyświetlenia graczowi i nie zalicza punktu. Gra kończy się po poprawnym uzupełnieniu wszystkich relacji.

\begin{figure}[htbp]
    \centering
    \includegraphics[width=0.7\textwidth]{images/planimetria_FLOW.png}
    \caption{Diagram przepływu dla zadania 2 -- planimetria}
    \label{planimetria:flow}
\end{figure}

\FloatBarrier

\section{Zadanie 3 – Nierówności (Autor)}
\section{Zadanie 4 – Funkcje (Autor)}
\section{Zadanie 5 – Geometria analityczna (Autor)}
\section{Zadanie 6 – Kombinatoryka (Autor)}
\section{Zadanie 7 – Liczby rzeczywiste i działania na zbiorach liczbowych 
\texorpdfstring{\\}{ } (Jan Walczak)}
\label{sec:rzeczywiste_praktyka}
\subsection{Problemy i różnice w realizacji zadania w praktyce}
\label{subsec:rzeczywiste_problemy}
Podczas implementacji zadania, zgodnie z teorią opisaną w podrozdziale \ref{subsec:liczby_rzeczywiste}
musiałem wprowadzić, względem niej, kilka poprawek. Pierwotny mechanizm zakładał, że zamek, umieszczony na skrzyni, będzie
wyposażony w pola na umieszczenie odpowiednich cyfr -- liczby elementów danego podzbioru znajdujących się na niej. W praktyce
okazało się to niemożliwe. Maksymalna liczba jednocyfrowa to 9, tym samym zadanie jest ograniczone do wyświetlania maksymalnie 9 liczb
rzeczywistych w pokoju. 

Postanowiłem zmienić zamysł i zamiast kłódki umieścić skrzynię z elektronicznym zamkiem i wyposażyć go w klawiaturę.
W ten sposób dostajemy możliwość umieszczenia więcej niż 9 liczb w pokoju. Uczeń staje równiez przed dodatkowym zadaniem -- 
wywnioskować kolejność wpisywanych liczb zgodnie z podpowiedzią, umieszczoną na ścianie (patrz rysnuek \ref{rzeczywiste:screen}).

\begin{figure}[htbp]
    \centering
    \includegraphics[width=0.7\textwidth]{images/liczby_rzeczywiste_screen.png}
    \caption{Zrzut ekranu -- skrzynia z zamkiem oraz podpowiedź na ścianie dla zadania 7.}
    \label{rzeczywiste:screen}
\end{figure}
\FloatBarrier

Zmienione zostały również liczby, które są rozmieszczone w pokoju. Aktualnie pojawiają się liczby: 
\textit{15; π; 4,5; √2; ¼; -3; 1; -7; ⅓ ; 9} Tym samym kombinacja prezentuje się następująco (kolejność zgodna z podpowiedzią 
na rysunku \ref{rzeczywiste:screen})
\begin{itemize}
    \item \textit{N} (naturalne): \textit{15; 1; 9} – 3 liczby,
    \item \textit{C} (całkowite):  \textit{15; 1; 9; -3; -7} – 5 liczb,
    \item \textit{R} (rzeczywiste): wszystkie liczby – 10 liczb,
    \item \textit{W} (wymierne): \textit{15; 1; 9; -3; -7; 4,5; ¼; ⅓} – 8 liczb,
    \item \textit{NW} (niewymierne): \textit{π; √2} – 2 liczby.
\end{itemize}

Tym samym kombinacja do otwarcia sejfu to \textit{351082}.

Ostatnią rzeczą, która została zmieniona jest koncepcja drugiej części zadania. Niepotrzebny jest mechanizm obracania bloczków,
ponieważ wybrane relacje są jednoznaczne -- są czytane zawsze od lewej do prawej strony. Niepoprawny obrót bloczka od razu sugeruje
błędną odpowiedź -- tym samym jest to mechanizm bezcelowy.

Spośród wszystkich relacji zostały wybrane dwie:
\begin{itemize}
    \item $N \subset Z \subset R$,
    \item $NW \cap W = \varnothing$,
\end{itemize}
Zostały one wybrane przez swoją charakterystykę. Pierwsza z nich pokazuje, jak kolejne zbiory zawierają się w sobie nazwajem,
a druga pokazuje rozłączność zbiorów liczb wymiernych i niewymiernych.
Gracz ma do dyspozycji 6 bloczków:
\begin{itemize}
    \item dwa bloczki z symbolem $\subset$,
    \item dwa bloczki z symbolem $\varnothing$,
    \item dwa bloczki z symbolem $\cap$
\end{itemize}
Mają mu posłużyć do uzupełnienia dwóch relacji wymienionych wcześniej. Po ułożeniu relacji poprawnie bądź nie bloczki pozostają
do dyspozycji gracza.

\subsection{Implementacja kontrolera}

W przeciwieństwie do zadania drugiego, opisanego w \ref{sec:planimetria_praktyka}, został zaimplementowany jedynie jeden, główny 
kontroler \verb|RealNumbersController|. Zarządza on przebiegiem zadania, położeniem elementów na scenie oraz wyświetlaniem tekstu.
Zawiera odpowiednie referencje do komponentów i odpowiada za poprawną inicjalizację.

Komponent \verb|Element| jest bloczkiem z odpowiednim symbolem relacji, \verb|Sejf| jest skrzynią, która przetrzymuje bloczki,
a \verb|WallQuestion| to płaska tablica (ścianka), na której wyświetlane są pytania w drugiej części zadania.

\verb|Sejf| posiada jednostronną referencję do głównego kontrolera. Kontroler może zostać przez niego powiadomiony, że gracz
poprawnie wpisał kod, tak aby kontroler mógł sterować dalszym przebiegiem zadania. \verb|WallQuestion| zawiera dwustronną referencję
do głównego kontrolera. Mechanizm powiadomień działa tutaj podobnie ale kontroler może odpowiadać na otrzymywane powiadomienia poprzez
wywoływanie odpowiednich funkcji.

\begin{figure}[htbp]
    \centering
    \includegraphics[width=0.7\textwidth]{images/real_UML.png}
    \caption{Diagram UML zawierający najważniejsze elementy dla zadania 7 -- liczby rzeczywiste i działania na zbiorach liczbowych.}
    \label{rzeczywiste:UML}
\end{figure}
\FloatBarrier

\subsection{Przebieg zadania}
Zadanie rozpoczyna się od pierwszej części -- wpisania kodu do skrzyni. Kontroler inicjalizuje komponenty, a  na ścianach i na skrzyni 
zostają wyświetlone liczby opisane w \ref{subsec:rzeczywiste_problemy}. Gracz powinien policzyć ile liczb i z jakiego zbioru jest 
wyświetlanych w pokoju i poprawnie wpisać kod, według podpowiedzi znajdującej się na ścianie.
Po poprawnym wpisaniu kodu, kontroler główny zostaje powiadomiony przez instancję skrzyni, że zadanie może przejść do następnego etapu.
Drzwi się otwierają a kontroler inicjalizuje 6 bloczków, opisanych w \ref{subsec:rzeczywiste_problemy} i ściankę z pytaniami \verb|WallQuestion|.
Zapisuje pozycję oraz obrót bloczków w świecie, tak aby możliwy był powrót do stanu początkowego, między udzielanymi odpowiedziami.

Ścianka wykrywa kolizję z bloczkami, które gracz na niej umieszcza. W momencie kiedy zostaje uzupełniona w całości tj. bloczki wypełniają 
miejsca na odpowiedzi, to powiadamia kontroler o rozwiązaniu zadania. Kontroler decyduje czy odpowiedź gracza jest poprawna. W przypadku gdy 
jest, to przesyła do \verb|WallQuestion| powiadomienie o zmianie pytania na kolejne i zalicza punkt. Jeżeli odpowiedź nie jest poprawna, to
również powiadamia o tym ściankę, ale nie zalicza punktu. W obu przypadkach bloczki wracają na wcześniej zapisaną pozycję.
W zależności od poprawności udzielonej odpowiedzi ścianka rozświetla się kolorem zielonym lub czerwonym, tak żeby gracz miał pewność, że udzielił
dobrej odpowiedzi. Podczas sprawdzania stan ścianki tj. ustawienie bloczków i pytanie zostaje zablokowany i gracz nie może go modifykować (dokładać
lub zabierać bloczków).

Kiedy gracz odpowie na oba pytania poprawnie poziom zostaje uznany za zaliczony.

\begin{figure}[htbp]
    \centering
    \includegraphics[width=1.0\textwidth]{images/real_FLOW.png}
    \caption{Diagram przepływu dla zadania 7 -- liczby rzeczywiste i działania na zbiorach liczbowych.}
    \label{rzeczywiste:FLOW}
\end{figure}
\FloatBarrier

\subsection{Implementacja mechanizmu uzupełniania ścianki}
Ścianka, czyli \verb|WallQuestion|, wykrywa kolizję z bloczkami, czyli \verb|Element| w dwóch miejscach, w których będą one docelowo
umieszczone. Kiedy gracz przytrzyma bloczek nad jednym z miejsc uruchamiana jest odpowiednia funkcja \verb|StartCollision()|. W zależności,
od tego gdzie został umieszczony bloczek, sprawdzany jest aktualny stan ścianki -- czy w tym miejscu gracz zdążył już umieścić bloczek.
Jeśli tak, to ścianka nie pozwoli mu umieścić w tym miejscu kolejnego bloczka, dopóki nie wyjmie poprzedniego. Dzieje się tak za sprawą dwóch
zmiennych wartości logicznych: \verb|ActorHovered| i \verb|ActorLockedIn|. Oznaczają one kolejno: stan, w którym gracz trzyma bloczek nad ścianką
i stan, w którym gracz umieścił bloczek na ściance. W przypadku gdy \verb|ActorLockedIn| jest ustawione na wartość \verb|0|, to zapisywana
jest referencja bloczku, który został właśnie umieszczony. 

Dodatkowo informacja o tym, że gracz puścił bloczek, czyli zamierza go umieścić w danym miejscu, jest odpowiednio zapisywana w bloczku.
Kiedy bloczek znajduje się w ręce gracza otrzymuje sygnaturę \verb|held|. Kiedy go puści, sygnatura jest odbierana. Wyżej wymienione wartości
logiczne są ustawiane dopiero wtedy, kiedy sygnatura najpierw istniała w instancji bloczka, a potem została z niej usunięta.

W przypadku, gdy gracz chce usunąć bloczek ze ścianki sytuacja jest analogiczna. Tym razem jednak, sprawdzamy czy bloczek, który zabiera gracz,
jest tym znajdującym się na ściance, poprzez porównanie referencji. 

\section{Zadanie 8 – Znaki funkcji trygonometrycznych (Autor)}
\section{Zadanie 9 – Ciągi liczbowe (Jan Walczak)}
\label{sec:ciągi_praktyka}
\subsection{Problemy i różnice w realizacji zadania w praktyce}
W przeciwieństwie do zadań opisanych w podroździałach \ref{sec:planimetria_praktyka} i \ref{sec:rzeczywiste_praktyka} w tym zadaniu nie pojawiło się 
dużo rozbieżności między teorią, a praktyczną implementacją.

Miejsce, w którym wprowadziłem modyfikację, to warunki zakończenia gry. Zadanie, przez swój projekt, stawia bardziej na szybką analizę, niż
na długie zastanawianie się nad odpowiedzią. Jest bezpośrednio inspirowane prawdziwą grą, w której największą wartością, wyróżniającą gracza, jest jego
czas reakcji. Z tego powodu zdecydowałem się na usunięcie limitu czasowego. Gra przyspiesza liniowo, razem z postępem tj. liczbą zestrzelonych przez
gracza kaczek.

Dodatkowo zwiększyłem liczbę żyć tj. możliwości na pomyłkę gracza do 4 zamiast 3. Jest to spowodowane wprowadzeniem dodatkowej mechaniki gry -- jeśli
kaczka wyleci poza zasięg gracza (za ścianę) gracz traci życie. Gra kończy się jedynie w momencie utraty wszsyskich żyć -- co musi nastąpić, ponieważ
prędkość kaczek będzie rosła tak długo, jak długo toczy się rozgrywka.

\subsection{Implementacja pistoletu laserowego}
Implementacja pistoletu lasterowego była kluczowym punktem, potrzebnym do stworzenia całego poziomu. 

Aby poprawnie zaimplementować taki laser potrzebne są w sumie trzy wartości:
\begin{itemize}
    \item punkt początkowy, od którego zaczynamy rysowanie lasera
    \item punkt końcowy, do którego laser będzie rysowany
    \item wektor kierunkowy, według którego laser będzie rysowany
\end{itemize}

Za punkt początkowy został obrany koniec ręki gracza tj. kontrolera którym steruje. Kolejno, z ustawienia gracza można wyznaczyć wektor kierunkowy.
Wystarczy pobrać z kontrolera jego wartość przy użyciu systemowej funkcji Unreal Engine \verb|Get Forward Vector|. Należało go zapisać,
razem z punktem początkowym. Obliczenie punktu końcowego jest nieco trudniejsze. Polega na początkowym pomnożeniu wektora kierunkowego przez inny wektor,
o dużych wartościach poszczególnych współrzędnych, a następnym dodaniu wyniku tego działania do współrzędnych punktu początkowego. 
W projekcie zastosowano wektor $(1000,0; 1000,0; 1000,0)$. Tak obliczony punkt należało chwilowo zapisać jako punkt końcowy.

Następnie, przy użyciu systemowej funkcji \verb|Line Trace by Channel|, należało wyznaczyć punkt, w którym linia, poprowadzona od punktu
początkowego do końcowego, przecina się ze światem przedstawionym w symulacji. Funkcja ta zwraca wynik w postaci wartości typu struktury \verb|Hit Result|, 
który można rozbić na składowe i pozyskać z niego \verb|Impact Point|, czyli rzeczywisty punkt końcowy oraz \verb|Distance|, c
zyli wartość zmiennoprzecinkową reprezentującą długość wyznaczonej linii.

Z obliczonymi wartościami można narysować laser, przebiegający od kontrolera gracza, do pierwszego napotkanego obiektu w grze. W tym celu kontroler ma
na stałe przypisany obiekt \verb|Static Mesh| z siatką statyczną w kształcie cylindra, który jest rozciągany według obliczonych wartości. Nałożony jest
na niego materiał emitujący czerwone światło. Na końcu, rozciągniętej siatki statycznej, umieszczona jest niewidzialna kulka, z włączonym systemem kolizji. 
Dzięki niej można wykrywać, w jaki komponent na scenie, aktualnie celuje gracz.

Na rysunku \ref{fig:sequence_laser} widać, że laser faktycznie wykrywa otoczenie, w które celuje gracz i zmienia odpowiednio swoją długość. Zmianie
ulega również kierunek, w którym celuje gracz.
\begin{figure}[htbp]
    \centering
    \includegraphics[width=0.7\textwidth]{images/sequence_laser.png}
    \caption{Porównanie rysowania lasera w przypadku kontaktu z przykładowym elementem otoczenia i ścianą.}
    \label{fig:sequence_laser}
\end{figure}

\subsection{Implementacja kontrolera}

Podobnie jak w zadaniu opisanym w podroździale \ref{sec:rzeczywiste_praktyka} zaimplementowany został jeden, główny kontroler \verb|SequenceControler|.
Jego zadaniem jest inicjalizacja gry, zarządzenie przebiegiem gry, wyświetlanie tekstu i zarządzanie instancjami kaczek. Kontroler wybiera pseudolosowo
jeden z trzech dostępnych wzorów na kolejny wyraz ciągu arytmetycznego, według którego będzie wybierał kolejne elementy:
\begin{itemize}
    \item $a_n = 3n + 4$
    \item $a_n = 2n + 3$
    \item $a_n = 5n + 7$
\end{itemize}

Komponent \verb|Duck| implementuje funkcjonalności związane ze sterowaniem pojedynczej kaczki. Zawiera dwustronną referencję do obiektu głównego kontrolera
i przesyła powiadomienia, kiedy zostanie zestrzelony przez gracza, przy użyciu pistoletu laserowego. W jej instancji zapisywana jest pozycja początkowa, numer,
który aktualnie jest na niej wyświetlany i wartość zmiennoprzecinkowa, czyli droga, którą pokona kaczka w każdej klatce wykonywania się programu.

Gracz zostaje również zapoznany z zasadami działania gry. Kiedy jest gotowy, do rozpoczęcia gry, naciska przycisk podobny do tego opisanego w podroździale
\ref{sec:planimetria_praktyka}. Działa na tej samej zasadzie i również posiada dwustronną referencję do instancji kontrolera.

W zagadce występują jeszcze dwa komponenty: \verb|HitWall| i \verb|Life|. Główny kontroler zawiera tablicę referencji do 4 
instancji obiektów \verb|Life|. Reprezentują ilość pozostałych żyć gracza i nie pełnią w logice działania programu żadnej 
istotnej roli, poza tą wizualną. \verb|HitWall| jest obiektem niewidocznym, znajdującym się poza zasięgiem gracza. Lecące 
kaczki będą wykrywać ten obiekt, 
po przeleceniu przez ścianę i sygnalizować to głównemu kontrolerowi tak, jakby zostały trafione. Jeżeli kaczka uderzy w 
ten komponent to wyśle specjalny typ powiadomienia, mówiący o tym, że graczowi nie udało się jej zestrzelić. Jeśli kaczka 
zawierała poprawny wyraz ciągu to kontroler oznaczyto jako błąd i odejmie graczowi jedno życie. W takim przypadku, jeżeli 
gracz ma jeszcze pozostałe życia, wyświetli się kolejny wyraz ciągu, a kaczki zaczną lecieć od początku.

\begin{figure}[htbp]
    \centering
    \includegraphics[width=0.7\textwidth]{images/sequence_UML.png}
    \caption{Diagram UML zawierający najważniejsze elementy dla zadania 9 -- Ciągi liczbowe.}
    \label{fig:sequence_UML}
\end{figure}

\subsection{Przebieg zadania}
Zadanie rozpoczyna się od inicjalizacji komponentów przez główny kontroler.  Początkowo graczowi zostaje
wyświetlona instrukcja do zadania. Kiedy gracz będzie gotowy może rozpocząć grę naciskając przycisk \glqq
Start\grqq. Kontroler zostaje w takiej sytuacji powiadomiony przez instancję przycisku i rozpoczyna grę. W sposób 
pseudolsowy zostaje wybrany jeden z trzech wzorów na kolejny \textit{n-ty} wyraz ciągu arytmetycznego. Ustawia również
poprawny wyraz ciągu, początkowo $n = 0$, na jednej z trzech kaczek. Pozostałe kaczki otrzymują wyrazy, większe
lub mniejsze od poprawnego wyrazu, również w sposób pseudolosowy, biorąc pod uwagę to, że wybrane wyrazy mogą być
odległe jedynie o 2 jednostki na osi współrzędnych, od poprawnego wyrazu.  

Kaczki pokonują stałą jednostkę odległości w każdej jednostce czasu wykonywania się programu, wyrażoną za pomocą
liczby zmiennoprzecinkowej. Jednostka ta jest tym większa, im większy jest aktualny numer wyrazu ciągu
arytmetycznego. Oznacza to, że będą przyspieszać razem z przebiegiem gry. Do tak obliczonej, stałej wartości,
dodawana jest pseudolosowa wartość zmiennoprzecinkowa wybrana z przedziału $\left[0,5; 0,7\right] $. 

Kaczki poruszają się od lewej do prawej strony ekranu, aż do zderzenia się z instancją obiektu \verb HitWall|. Kaczki mogą zostać zestrzelone przez gracza, zanim zderzą się z wyżej wymienionym obiektem. W obu przypadkach, do kontrolera wysyłane jest powiadomienie odpowiedniego rodzaju: kaczka została zestrzelona lub kaczka  zderzyła się z obiektem \verb|HitWall|. W powiadomieniu przekazana jest również wartość stałoprzecinkowa, reprezentująca liczbę, która znajdowała sie na danej kaczce. 

Kontroler po otrzymaniu w powiadomieniu liczby oraz informacji, czy dana kaczka została zestrzelona, 
podejmuje decyzję:
\begin{enumerate}
  \item Jeśli liczba jest poprawnym wyrazem ciągu -- zalicza punkt i ustawia ponownie kaczki.
  \item Jeśli liczba nie jest poprawna -- odejmuje jedno życie, ale nie ustawia ponownie kaczek.
  \item Jeśli kaczka, przechowująca poprawny wyraz ciągu, uderzy w obiekt \verb|HitWall| (wyleci poza 
  zasięg gracza) -- odejmuje jedno życie i ponownie ustawia kaczki.
\end{enumerate}
Ponowne ustawienie kaczek polega na obliczeniu kolejnego wyrazu ciągu arytmetycznego, ustawieniu go na 
jednej z nich, ustawieniu dwóm pozostałym kaczkom błędnego wyrazu oraz na cofnięciu ich do stanu 
początkowego. Gra kończy się w momencie, gdy gracz straci wszystkie 4 życia.

\begin{figure}[htbp]
    \centering
    \includegraphics[width=1.0\textwidth]{images/sequence_FLOW.png}
    \caption{Diagram przepływu dla zadania 9 -- Ciągi liczbowe }
    \label{sequence:flow}
\end{figure}

\section{Zadanie 10 – Prawdopodobieństwo (Autor)}
\section{Zadanie 11 – Optymalizacja i rachunek różniczkowy (Autor)}
\section{Zadanie 12 – Układy równań (Autor)}
\section{Zadanie 13 – Stereometria (Jan Walczak)}

\chapter{Raport z testowania (Jan Walczak)}
\label{Raport_z_testowania}

\section{Przebieg testów}

Podczas implementacji aplikacji można wyróżnić trzy iteracje testowania, z których każda skupiała się na innej
funkcjonalności. Głównym celem testów było sprawdzenie, jak zachowuje się przygotowana aplikacja w~warunkach laboratoryjnych
w~LZWP.

Pierwsza iteracja obejmowała testy podstawowych mechanizmów dotyczących poruszania się gracza. 
Trwała ona dwa dni, podczas których odbyły się dwa spotkania w LZWP. Sprawdzono:
\begin{itemize}
    \item śledzenie ruchu gracza w LZWP,
    \item poprawność wczytywania wejścia poprzez naciskanie odpowiednich przycisków na kontrolerze,
    \item podstawowe interakcje z otoczeniem, takie jak używanie wskaźnika laserowego oraz chwytanie wirtualnych przedmiotów.
\end{itemize}

Druga iteracja testów miała na celu sprawdzenie funkcjonalności wszystkich opracowanych zagadek.
Skupiała się na identyfikacji miejsc, w których aplikacja generowała błędy w środowisku docelowym.
Zidentyfikowane błędy były poprawiane poza LZWP. Iteracja ta obejmowała trzy dni spotkań w LZWP.

Ostatnia iteracja testów koncentrowała się na wielokrotnym budowaniu aplikacji oraz analizie różnic w działaniu pomiędzy
wersją edytorską, testowaną w edytorze Unreal Engine, a wersją skompilowaną aplikacji (ang.~\textit{build}),
która działała w~CAVE w~LZWP. Była to największa część testowania i zajęła dwa tygodnie.

\section{Napotkane problemy}

Najwięcej problemów napotkano podczas trzeciej iteracji testów. Okazało się, że skompilowana wersja aplikacji zachowywała się
inaczej niż wersja edytorska. Wszelkie napotkane błędy wymagały wprowadzenia poprawek oraz ponownej kompilacji aplikacji, co
było procesem czasochłonnym i znacząco wydłużyło tę część testowania.

Głównym problemem, którego nie można było wykryć na etapie testów w edytorze, było pozycjonowanie kamery gracza oraz
odpowiednia synchronizacja pozycji kontrolera w LZWP. Różnice te wynikały z nieprawidłowego nałożenia przesunięcia
(ang.~\textit{offset}) oraz skali kamery, która reprezentowała gracza w wirtualnej przestrzeni. W konsekwencji dochodziło do
niepoprawnego wyświetlania wirtualnego pokoju w~CAVE -- jego podłoga częściowo nachodziła na ściany, a ruch
kontrolera nie był poprawnie przekazywany do aplikacji, przez co gracz nie był w stanie skutecznie poruszać się po
wirtualnym pomieszczeniu. Rozwiązaniem było eksperymentalne ustawienie przesunięcia i skali kamery, tak aby granice
ścian pokoju były poprawnie wyświetlane.

Kolejnym problemem była synchronizacja obiektów generowanych w sposób pseudolosowy. Okazało się, że klaster komputerowy,
na którym uruchamiana była aplikacja w LZWP, obliczał liczby pseudolosowe niezależnie na każdym komputerze. Skutkowało to
desynchronizacją tych obiektów, a w konsekwencji ich niepoprawnym wyświetlaniem. Rozwiązaniem okazało się ustalenie
wspólnego ziarna (ang.~\textit{seed}) dla wszystkich komputerów działających w klastrze.

Ostatnim istotnym problemem była desynchronizacja obiektów w grze, na które oddziaływała grawitacja. Problem ten miał
podobną naturę do poprzedniego — położenie takich obiektów było obliczane niezależnie na każdym komputerze klastra,
przez co nie było wyznaczane jednoznacznie. Rozwiązaniem było ustawienie w edytorze Unreal Engine symulacji fizyki
w tryb deterministyczny.

\section{Wykryte błędy}
Oprócz wyżej wymienionych problemów pojawiły się również błędy, które nie były związane z~działaniem aplikacji
w~LZWP, lecz wynikały bezpośrednio z błędów implementacyjnych.

Pierwszym wykrytym problemem była funkcja chwytania przedmiotów przez gracza. Kontroler, z którego korzysta użytkownik
w~LZWP, wyposażony jest w analogowy spust, który po naciśnięciu zwraca wartości zmiennoprzecinkowe. W aplikacji mechanizm
chwytania przedmiotów został natomiast zaprojektowany z myślą o wejściu zero-jedynkowym, odpowiadającym wciśnięciu
klawisza na klawiaturze. Błąd ten okazał się na tyle istotny, że uniemożliwiał poprawne chwytanie obiektów.
Rozwiązaniem było zmodyfikowanie obsługi wejścia w taki sposób, aby przyjmowała wartości zmiennoprzecinkowe.

Drugim znaczącym błędem był sposób generowania figur w zadaniu opisanym w podrozdziale
nr.~\ref{sec:planimetria_praktyka}. Figury były tworzone podczas inicjalizacji zadania w sposób nieprawidłowy.
Obiekty należące do poszczególnych zadań były oznaczane odpowiednimi etykietami, co umożliwiało aplikacji
rozpoznanie ich przynależności do konkretnego zadania. Generowane figury nie posiadały jednak takich etykiet,
w wyniku czego nigdy nie znikały z wirtualnego pokoju. Kontroler odpowiedzialny za ich położenie zawierał
referencje do instancji tych obiektów, przez co ukrywana była jedynie ich tekstura. Błąd ten okazał się szczególnie problematyczny, ponieważ niewidoczne figury wchodziły w kolizję z innymi obiektami
znajdującymi się na scenie. Prowadziło to m.in. do niezamierzonego naciskania przycisków lub blokowania dostępu
gracza do pozostałych elementów interaktywnych. Rozwiązaniem było dodanie odpowiednich etykiet do generowanych figur.

Ostatnim istotnym błędem było niespójne oznaczanie przedmiotów, które mogły być chwytane przez gracza. Funkcjonalność
chwytania opierała się na przyczepianiu chwytanych elementów do kontrolera, co realizowano poprzez dołączanie całych
obiektów reprezentowanych w aplikacji przez aktorów. Część elementów została jednak zaprojektowana w sposób nieprawidłowy,
tzn. podczas chwytania przenoszona była jedynie ich tekstura, natomiast obiekt aktora pozostawał w miejscu, z którego został
zabrany. W konsekwencji obiekty te mogły zostać złapane tylko jeden raz, po czym ich tekstura ulegała oddzieleniu od obiektu aktora.
Prowadziło to do niespójnego zachowania aplikacji oraz utrudniało dalszą interakcję z otoczeniem. Rozwiązaniem było
ujednolicenie systemu chwytania przedmiotów oraz oparcie go na przyczepianiu do ręki pełnych obiektów aktorów.


\chapter{Podsumowanie}
\label{chap:summary}

\section{Ocena realizacji celów}

\section{Propozycje rozwoju projektu}

\section{Wnioski końcowe}
\include{chapters/8_wykazy}
\end{document}
